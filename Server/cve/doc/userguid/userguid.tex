% This file was converted to LaTeX by Writer2LaTeX ver. 1.0.2
% see http://writer2latex.sourceforge.net for more info
\documentclass[letterpaper]{article}
\usepackage[latin1]{inputenc}
\usepackage[T1]{fontenc}
\usepackage[english]{babel}
\usepackage{amsmath}
\usepackage{amssymb,amsfonts,textcomp}
\usepackage{color}
\usepackage{array}
\usepackage{supertabular}
\usepackage{hhline}
\usepackage{hyperref}
\hypersetup{pdftex, colorlinks=true, linkcolor=blue, citecolor=blue, filecolor=blue, urlcolor=blue, pdftitle=, pdfauthor=Clinton Jeffery, pdfsubject=, pdfkeywords=}
\usepackage[pdftex]{graphicx}
% Text styles
\newcommand\textstyleTeletype[1]{\texttt{#1}}
\newcommand\textstyleStrongEmphasis[1]{\textbf{#1}}
% Outline numbering
\setcounter{secnumdepth}{0}
\makeatletter
\newcommand\arraybslash{\let\\\@arraycr}
\makeatother
% List styles
\newcounter{saveenum}
\newcommand\liststyleLi{%
\renewcommand\labelitemi{${\bullet}$}
\renewcommand\labelitemii{${\circ}$}
\renewcommand\labelitemiii{${\blacksquare}$}
\renewcommand\labelitemiv{${\bullet}$}
}
\newcommand\liststyleLii{%
\renewcommand\labelitemi{${\bullet}$}
\renewcommand\labelitemii{${\circ}$}
\renewcommand\labelitemiii{${\blacksquare}$}
\renewcommand\labelitemiv{${\bullet}$}
}
\newcommand\liststyleLiii{%
\renewcommand\theenumi{\arabic{enumi}}
\renewcommand\theenumii{\arabic{enumii}}
\renewcommand\theenumiii{\arabic{enumiii}}
\renewcommand\theenumiv{\arabic{enumiv}}
\renewcommand\labelenumi{\theenumi.}
\renewcommand\labelenumii{\theenumii.}
\renewcommand\labelenumiii{\theenumiii.}
\renewcommand\labelenumiv{\theenumiv.}
}
\newcommand\liststyleLiv{%
\renewcommand\labelitemi{${\bullet}$}
\renewcommand\labelitemii{${\circ}$}
\renewcommand\labelitemiii{${\blacksquare}$}
\renewcommand\labelitemiv{${\bullet}$}
}
\newcommand\liststyleLv{%
\renewcommand\labelitemi{${\bullet}$}
\renewcommand\labelitemii{${\circ}$}
\renewcommand\labelitemiii{${\blacksquare}$}
\renewcommand\labelitemiv{${\bullet}$}
}
\newcommand\liststyleLvi{%
\renewcommand\theenumi{\arabic{enumi}}
\renewcommand\theenumii{\arabic{enumii}}
\renewcommand\theenumiii{\arabic{enumiii}}
\renewcommand\theenumiv{\arabic{enumiv}}
\renewcommand\labelenumi{\theenumi.}
\renewcommand\labelenumii{\theenumii.}
\renewcommand\labelenumiii{\theenumiii.}
\renewcommand\labelenumiv{\theenumiv.}
}
\newcommand\liststyleLvii{%
\renewcommand\labelitemi{${\bullet}$}
\renewcommand\labelitemii{${\circ}$}
\renewcommand\labelitemiii{${\blacksquare}$}
\renewcommand\labelitemiv{${\bullet}$}
}
\newcommand\liststyleLviii{%
\renewcommand\theenumi{\arabic{enumi}}
\renewcommand\theenumii{\arabic{enumii}}
\renewcommand\theenumiii{\arabic{enumiii}}
\renewcommand\theenumiv{\arabic{enumiv}}
\renewcommand\labelenumi{(\theenumi)}
\renewcommand\labelenumii{\theenumii.}
\renewcommand\labelenumiii{\theenumiii.}
\renewcommand\labelenumiv{\theenumiv.}
}
\newcommand\liststyleLix{%
\renewcommand\theenumi{\arabic{enumi}}
\renewcommand\theenumii{\arabic{enumii}}
\renewcommand\theenumiii{\arabic{enumiii}}
\renewcommand\theenumiv{\arabic{enumiv}}
\renewcommand\labelenumi{(\theenumi)}
\renewcommand\labelenumii{\theenumii.}
\renewcommand\labelenumiii{\theenumiii.}
\renewcommand\labelenumiv{\theenumiv.}
}
\newcommand\liststyleLx{%
\renewcommand\theenumi{\arabic{enumi}}
\renewcommand\theenumii{\arabic{enumii}}
\renewcommand\theenumiii{\arabic{enumiii}}
\renewcommand\theenumiv{\arabic{enumiv}}
\renewcommand\labelenumi{(\theenumi)}
\renewcommand\labelenumii{\theenumii.}
\renewcommand\labelenumiii{\theenumiii.}
\renewcommand\labelenumiv{\theenumiv.}
}
\newcommand\liststyleLxi{%
\renewcommand\labelitemi{${\bullet}$}
\renewcommand\labelitemii{${\circ}$}
\renewcommand\labelitemiii{${\blacksquare}$}
\renewcommand\labelitemiv{${\bullet}$}
}
\newcommand\liststyleLxii{%
\renewcommand\theenumi{\arabic{enumi}}
\renewcommand\theenumii{\arabic{enumii}}
\renewcommand\theenumiii{\arabic{enumiii}}
\renewcommand\theenumiv{\arabic{enumiv}}
\renewcommand\labelenumi{\theenumi.}
\renewcommand\labelenumii{\theenumii.}
\renewcommand\labelenumiii{\theenumiii.}
\renewcommand\labelenumiv{\theenumiv.}
}
\newcommand\liststyleLxiii{%
\renewcommand\theenumi{\arabic{enumi}}
\renewcommand\theenumii{\arabic{enumii}}
\renewcommand\theenumiii{\arabic{enumiii}}
\renewcommand\theenumiv{\arabic{enumiv}}
\renewcommand\labelenumi{(\theenumi)}
\renewcommand\labelenumii{\theenumii.}
\renewcommand\labelenumiii{\theenumiii.}
\renewcommand\labelenumiv{\theenumiv.}
}
\newcommand\liststyleLxiv{%
\renewcommand\labelitemi{${\bullet}$}
\renewcommand\labelitemii{${\circ}$}
\renewcommand\labelitemiii{${\blacksquare}$}
\renewcommand\labelitemiv{${\bullet}$}
}
\newcommand\liststyleLxv{%
\renewcommand\labelitemi{${\bullet}$}
\renewcommand\labelitemii{${\circ}$}
\renewcommand\labelitemiii{${\blacksquare}$}
\renewcommand\labelitemiv{${\bullet}$}
}
\newcommand\liststyleLxvi{%
\renewcommand\labelitemi{${\bullet}$}
\renewcommand\labelitemii{${\circ}$}
\renewcommand\labelitemiii{${\blacksquare}$}
\renewcommand\labelitemiv{${\bullet}$}
}
\newcommand\liststyleLxvii{%
\renewcommand\labelitemi{${\bullet}$}
\renewcommand\labelitemii{${\circ}$}
\renewcommand\labelitemiii{${\blacksquare}$}
\renewcommand\labelitemiv{${\bullet}$}
}
\newcommand\liststyleLxviii{%
\renewcommand\labelitemi{${\bullet}$}
\renewcommand\labelitemii{${\circ}$}
\renewcommand\labelitemiii{${\blacksquare}$}
\renewcommand\labelitemiv{${\bullet}$}
}
\newcommand\liststyleLxix{%
\renewcommand\labelitemi{${\bullet}$}
\renewcommand\labelitemii{${\circ}$}
\renewcommand\labelitemiii{${\blacksquare}$}
\renewcommand\labelitemiv{${\bullet}$}
}
\newcommand\liststyleLxx{%
\renewcommand\theenumi{\arabic{enumi}}
\renewcommand\theenumii{\arabic{enumii}}
\renewcommand\theenumiii{\arabic{enumiii}}
\renewcommand\theenumiv{\arabic{enumiv}}
\renewcommand\labelenumi{\theenumi.}
\renewcommand\labelenumii{\theenumii.}
\renewcommand\labelenumiii{\theenumiii.}
\renewcommand\labelenumiv{\theenumiv.}
}
\newcommand\liststyleLxxi{%
\renewcommand\theenumi{\arabic{enumi}}
\renewcommand\theenumii{\arabic{enumii}}
\renewcommand\theenumiii{\arabic{enumiii}}
\renewcommand\theenumiv{\arabic{enumiv}}
\renewcommand\labelenumi{\theenumi.}
\renewcommand\labelenumii{\theenumii.}
\renewcommand\labelenumiii{\theenumiii.}
\renewcommand\labelenumiv{\theenumiv.}
}
\newcommand\liststyleLxxii{%
\renewcommand\labelitemi{${\bullet}$}
\renewcommand\labelitemii{${\circ}$}
\renewcommand\labelitemiii{${\blacksquare}$}
\renewcommand\labelitemiv{${\bullet}$}
}
\newcommand\liststyleLxxiii{%
\renewcommand\labelitemi{${\bullet}$}
\renewcommand\labelitemii{${\circ}$}
\renewcommand\labelitemiii{${\blacksquare}$}
\renewcommand\labelitemiv{${\bullet}$}
}
\newcommand\liststyleLxxiv{%
\renewcommand\labelitemi{${\bullet}$}
\renewcommand\labelitemii{${\circ}$}
\renewcommand\labelitemiii{${\blacksquare}$}
\renewcommand\labelitemiv{${\bullet}$}
}
\newcommand\liststyleLxxv{%
\renewcommand\theenumi{\arabic{enumi}}
\renewcommand\theenumii{\arabic{enumii}}
\renewcommand\theenumiii{\arabic{enumiii}}
\renewcommand\theenumiv{\arabic{enumiv}}
\renewcommand\labelenumi{\theenumi.}
\renewcommand\labelenumii{\theenumii.}
\renewcommand\labelenumiii{\theenumiii.}
\renewcommand\labelenumiv{\theenumiv.}
}
\newcommand\liststyleLxxvi{%
\renewcommand\theenumi{\arabic{enumi}}
\renewcommand\theenumii{\arabic{enumii}}
\renewcommand\theenumiii{\arabic{enumiii}}
\renewcommand\theenumiv{\arabic{enumiv}}
\renewcommand\labelenumi{\theenumi.}
\renewcommand\labelenumii{\theenumii.}
\renewcommand\labelenumiii{\theenumiii.}
\renewcommand\labelenumiv{\theenumiv.}
}
\newcommand\liststyleLxxvii{%
\renewcommand\theenumi{\arabic{enumi}}
\renewcommand\theenumii{\arabic{enumi}.\arabic{enumii}}
\renewcommand\theenumiii{\arabic{enumi}.\arabic{enumii}.\arabic{enumiii}}
\renewcommand\theenumiv{\arabic{enumi}.\arabic{enumii}.\arabic{enumiii}.\arabic{enumiv}}
\renewcommand\labelenumi{\theenumi.}
\renewcommand\labelenumii{\theenumii.}
\renewcommand\labelenumiii{\theenumiii.}
\renewcommand\labelenumiv{\theenumiv.}
}
\newcommand\liststyleLxxviii{%
\renewcommand\labelitemi{[E008?]}
\renewcommand\labelitemii{{\textbullet}}
\renewcommand\labelitemiii{{\textbullet}}
\renewcommand\labelitemiv{{\textbullet}}
}
\newcommand\liststyleLxxix{%
\renewcommand\labelitemi{${\bullet}$}
\renewcommand\labelitemii{${\circ}$}
\renewcommand\labelitemiii{${\blacksquare}$}
\renewcommand\labelitemiv{${\bullet}$}
}
\newcommand\liststyleLxxx{%
\renewcommand\labelitemi{[E008?]}
\renewcommand\labelitemii{{\textbullet}}
\renewcommand\labelitemiii{{\textbullet}}
\renewcommand\labelitemiv{{\textbullet}}
}
\newcommand\liststyleLxxxi{%
\renewcommand\theenumi{\arabic{enumi}}
\renewcommand\theenumii{\arabic{enumii}}
\renewcommand\theenumiii{\arabic{enumiii}}
\renewcommand\theenumiv{\arabic{enumiv}}
\renewcommand\labelenumi{\theenumi.}
\renewcommand\labelenumii{\theenumii.}
\renewcommand\labelenumiii{\theenumiii.}
\renewcommand\labelenumiv{\theenumiv.}
}
\newcommand\liststyleLxxxii{%
\renewcommand\labelitemi{[E008?]}
\renewcommand\labelitemii{{\textbullet}}
\renewcommand\labelitemiii{{\textbullet}}
\renewcommand\labelitemiv{{\textbullet}}
}
\newcommand\liststyleLxxxiii{%
\renewcommand\labelitemi{[E008?]}
\renewcommand\labelitemii{{\textbullet}}
\renewcommand\labelitemiii{{\textbullet}}
\renewcommand\labelitemiv{{\textbullet}}
}
\newcommand\liststyleLxxxiv{%
\renewcommand\theenumi{\arabic{enumi}}
\renewcommand\theenumii{\arabic{enumii}}
\renewcommand\theenumiii{\arabic{enumiii}}
\renewcommand\theenumiv{\arabic{enumiv}}
\renewcommand\labelenumi{\theenumi.}
\renewcommand\labelenumii{\theenumii.}
\renewcommand\labelenumiii{\theenumiii.}
\renewcommand\labelenumiv{\theenumiv.}
}
\newcommand\liststyleLxxxv{%
\renewcommand\theenumi{\arabic{enumi}}
\renewcommand\theenumii{\arabic{enumii}}
\renewcommand\theenumiii{\arabic{enumiii}}
\renewcommand\theenumiv{\arabic{enumiv}}
\renewcommand\labelenumi{\theenumi.}
\renewcommand\labelenumii{\theenumii.}
\renewcommand\labelenumiii{\theenumiii.}
\renewcommand\labelenumiv{\theenumiv.}
}
\newcommand\liststyleLxxxvi{%
\renewcommand\labelitemi{${\bullet}$}
\renewcommand\labelitemii{${\circ}$}
\renewcommand\labelitemiii{${\blacksquare}$}
\renewcommand\labelitemiv{${\bullet}$}
}
\newcommand\liststyleLxxxvii{%
\renewcommand\theenumi{\arabic{enumi}}
\renewcommand\theenumii{\arabic{enumii}}
\renewcommand\theenumiii{\arabic{enumiii}}
\renewcommand\theenumiv{\arabic{enumiv}}
\renewcommand\labelenumi{\theenumi.}
\renewcommand\labelenumii{\theenumii.}
\renewcommand\labelenumiii{\theenumiii.}
\renewcommand\labelenumiv{\theenumiv.}
}
\newcommand\liststyleLxxxviii{%
\renewcommand\labelitemi{[E008?]}
\renewcommand\labelitemii{{\textbullet}}
\renewcommand\labelitemiii{{\textbullet}}
\renewcommand\labelitemiv{{\textbullet}}
}
\newcommand\liststyleLxxxix{%
\renewcommand\labelitemi{${\bullet}$}
\renewcommand\labelitemii{${\circ}$}
\renewcommand\labelitemiii{${\blacksquare}$}
\renewcommand\labelitemiv{${\bullet}$}
}
\newcommand\liststyleLxl{%
\renewcommand\labelitemi{${\bullet}$}
\renewcommand\labelitemii{${\circ}$}
\renewcommand\labelitemiii{${\blacksquare}$}
\renewcommand\labelitemiv{${\bullet}$}
}
\newcommand\liststyleLxli{%
\renewcommand\labelitemi{${\bullet}$}
\renewcommand\labelitemii{${\circ}$}
\renewcommand\labelitemiii{${\blacksquare}$}
\renewcommand\labelitemiv{${\bullet}$}
}
% Page layout (geometry)
\setlength\voffset{-1in}
\setlength\hoffset{-1in}
\setlength\topmargin{0.7874in}
\setlength\oddsidemargin{0.7874in}
\setlength\textheight{9.062033in}
\setlength\textwidth{6.9251995in}
\setlength\footskip{26.148pt}
\setlength\headheight{0cm}
\setlength\headsep{0cm}
% Footnote rule
\setlength{\skip\footins}{0.0469in}
\renewcommand\footnoterule{\vspace*{-0.0071in}\setlength\leftskip{0pt}\setlength\rightskip{0pt plus 1fil}\noindent\textcolor{black}{\rule{0.25\columnwidth}{0.0071in}}\vspace*{0.0398in}}
% Pages styles
\makeatletter
\newcommand\ps@Standard{
  \renewcommand\@oddhead{}
  \renewcommand\@evenhead{}
  \renewcommand\@oddfoot{}
  \renewcommand\@evenfoot{}
  \renewcommand\thepage{\arabic{page}}
}
\newcommand\ps@FirstPage{
  \renewcommand\@oddhead{}
  \renewcommand\@evenhead{}
  \renewcommand\@oddfoot{}
  \renewcommand\@evenfoot{}
  \renewcommand\thepage{\arabic{page}}
}
\newcommand\ps@RightPage{
  \renewcommand\@oddhead{}
  \renewcommand\@evenhead{}
  \renewcommand\@oddfoot{\thepage{}}
  \renewcommand\@evenfoot{\@oddfoot}
  \renewcommand\thepage{\arabic{page}}
}
\newcommand\ps@LeftPage{
  \renewcommand\@oddhead{}
  \renewcommand\@evenhead{}
  \renewcommand\@oddfoot{\thepage{}}
  \renewcommand\@evenfoot{\@oddfoot}
  \renewcommand\thepage{\arabic{page}}
}
\makeatother
\pagestyle{Standard}
\setlength\tabcolsep{1mm}
\renewcommand\arraystretch{1.3}
\title{CVE User's Guide}
\author{Gustav Verhulsdonck \and Hani Bani-Salameh \and Iyad Abu Doush \and Iyad Al-Sharif \and Clinton Jeffery}
\date{2010-09-19T00:26:28.60}
\begin{document}
\clearpage\setcounter{page}{1}\pagestyle{Standard}
\thispagestyle{FirstPage}

\ \\ \ \\ \ \\ \ \\ \ \\ \ \\ \ \\ \ \\ \ \\ \ \\ \ \\ \ \\ \ \\ \ \\ \ \\

{\centering\sffamily\bfseries\itshape\Huge\color[rgb]{0.0,0.0,0.5019608}
CVE User's Guide
\par}

\ \\ \ \\ \ \\

{\centering\sffamily
Gustav Verhulsdonck \\
Hani Bani-Salameh \\
Iyad Abu Doush \\
Ziad Al-Sharif \\
Clinton Jeffery
\par}

\ \\ \ \\

{\centering\sffamily
Document version 0.2, for CVE version 0.5 \\
October 4, 2010
\par}


\bigskip

\bigskip

\bigskip

\bigskip

\bigskip

\bigskip

\bigskip

\bigskip

{\centering\sffamily
University of Idaho \\
Moscow, ID \\
and \\
New Mexico State University \\
Las Cruces, NM
\par}

\clearpage
\bigskip

\ \\ \ \\ \ \\ \ \\ \ \\ \ \\ \ \\ \ \\ \ \\ \ \\ \ \\ \ \\ \ \\ \ \\ \ \\
\ \\ \ \\ \ \\ \ \\ \ \\ \ \\ \ \\ \ \\ \ \\ \ \\ \ \\

{\centering\sffamily
Copyright � 2006-2010 by the authors
\par}

{\centering\sffamily
This document is made available under the GNU Open Documentation
License\newline
 and may be freely copied and modified under the terms of that license.
\par}

\clearpage

\setcounter{tocdepth}{10}
\renewcommand\contentsname{Table of Contents}
\tableofcontents

\bigskip

\clearpage\setcounter{page}{1}\pagestyle{LeftPage}
\thispagestyle{RightPage}
\subsection[Introduction]{\sffamily\bfseries\itshape\color[rgb]{0.0,0.0,0.5019608}
Introduction}

\bigskip

{\sffamily
Welcome to CVE. CVE stands for \textbf{C}ollaborative \textbf{V}irtual
\textbf{E}nvironment, a joint initiative of \ University of Idaho and
New Mexico State University that aims to help interested students
everywhere gain more experience in the area of computer science. CVE is
a virtual world in which students can collaborate on computer science
projects with other users, or work on their individual projects with
the help of an instructor. }


\bigskip

\noindent {\sffamily\bfseries\color[rgb]{0.0,0.0,0.5019608}
How to use this manual}

\noindent {\sffamily
This manual is comprised of 13 chapters. Each chapter is divided into
different sections that explain a particular aspect of CVE. It is told
primarily from the perspective of a Microsoft Windows user, as they are
the most numerous kind of users.}

\bigskip

\noindent {\sffamily\bfseries\color[rgb]{0.0,0.0,0.5019608}
About Sections}

\noindent {\sffamily
\textcolor{black}{Each section}\textbf{\textcolor{black}{
}}\textcolor{black}{is indicated by}\textbf{\textcolor{black}{
}}\textcolor{black}{a dark-blue font heading \ as you see above.
\ }Each chapter starts with a general overview of the chapter that
explains what the chapter is about. The section of the chapter is
followed by a bullet list explaining the main steps and/or the main
components that will be discussed in the chapter. This is followed by
the items outlined in the overview.}

\bigskip

\noindent {\sffamily\bfseries\color[rgb]{0.0,0.0,0.5019608}
About Instructions}

\noindent {\sffamily
\textbf{\textit{\textcolor[rgb]{0.0,0.2784314,1.0}{Each instruction}}}
is indicated by\textcolor{black}{ a}\textcolor{black}{ light-blue
text.} Some instructions in this manual require that you follow a set
order. In this case these steps are numbered. }

\bigskip

\noindent {\sffamily
An example of a step in an instruction:}


\noindent {\sffamily\bfseries\itshape\color[rgb]{0.0,0.2784314,1.0}
1. Click on the Windows CVE Logo on your desktop or in your Programs
menu to start CVE}

\noindent {\sffamily
Below each step, \textit{background information} will be offered in a
regular black font. While you are encouraged to read through the entire
manual, it is possible to go through certain steps quickly by reading
the highlighted light-blue text. \ }

{\sffamily
\textbf{\textit{\textcolor[rgb]{0.0,0.0,0.5019608}{\newline
About NOTE: }}}\textbf{\textcolor[rgb]{0.0,0.0,0.5019608}{ Boxes}}}

\noindent {\sffamily
At times important information will be offered of which an example is
given below:}

\noindent {\sffamily
\textbf{\textit{NOTE}}\textbf{:} }

\noindent {\sffamily
Important information will be offered in this box. It is important that
you read this information since it involves subsequent steps. }

\noindent {\sffamily
These boxes are there to remind you off important information as well as
serve as visual reminders for when you revisit that section.}

\bigskip

\noindent {\sffamily\bfseries\color[rgb]{0.0,0.0,0.5019608}
About Quicklist sections}

\noindent {\sffamily
Each Chapter ends with a Quicklist which gives a quick overview of the
keys used for specific purposes. For those that do not want to revisit
the information in the chapter, but simply want to remind themselves of
particular key combination used for specific tasks, these quicklists
are a valuable resource in allowing you to revisit some of the specific
concepts and actions you want to remember in the chapter.}

\bigskip

\noindent {\sffamily\bfseries\color[rgb]{0.0,0.0,0.5019608}
Questions, Suggestions, Comments?}

\noindent {\sffamily
If you have any questions, suggestions that would make this a better
manual, please email the authors at: \linebreak
\href{mailto:jeffery@cs.uidaho.edu}{jeffery@cs.uidaho.edu}.}

\clearpage\subsection{Overview of User Guide}
{\sffamily
In order to use CVE, you{\textquotesingle}ll first need to familiarize
yourself with this user guide to get a sense of all of the different
things you can do in the virtual world. This guide is set up in the
following manner:\newline
}


\bigskip

\liststyleLi
\begin{itemize}
\item {\sffamily
\textbf{Chapter 1:} Creating a New Account \newline
(Registering as a New User, Creating a New Avatar \& Avatar creation
options)}
\item {\sffamily
\textbf{Chapter 2:} Logging in \ \newline
(How to log in to CVE after You{\textquotesingle}ve Registered and
created Your Avatar)}
\item {\sffamily
\textbf{Chapter 3:} Using CVE \newline
(What various areas of the CVE interface will do) \ \ }
\item {\sffamily
\textbf{Chapter 4: }\ Exploring the 3d environment}

{\sffamily
(How to navigate your avatar through the environment)}
\item {\sffamily
\textbf{Chapter 5: \ }Interacting with Other Users}

{\sffamily
(How to interact and communicate in various ways with other users)}
\item {\sffamily
\textbf{Chapter 6: }Using Text Chat}

{\sffamily
(How to use text chat to communicate with other)}
\item {\sffamily
\textbf{Chapter 7:} Using the Voice Chat To Talk To Other Users}

{\sffamily
(How to use audio conversation tools in order to communicate with
others)}
\item {\sffamily
\textbf{Chapter 8: }Using the File Menu Bar to Create Projects}

{\sffamily
(How to use the File Menu Bar for various options)}
\item {\sffamily
\textbf{Chapter 9: }Creating a Project Using the Integrated Development
Environment}

{\sffamily
(How to create programs using the IDE)}
\item {\sffamily
\textbf{Chapter 10:} Creating Unicon Projects in CVE}

{\sffamily
(How to create, run and debug Unicon programs)}
\item {\sffamily
\textbf{Chapter 11:} Creating Java Projects in CVE}

{\sffamily
(How to create, run and debug Java programs)}
\item {\sffamily
\textbf{Chapter 12:} Creating C/C++ Projects in CVE}

{\sffamily
(How to create, run and debug C/C++ programs)}
\item {\sffamily
\textbf{Chapter 13:} Using the IDE to Share Documents With Others}

{\sffamily
(How to use the IDE sharing function to collaborate on programs)}
\end{itemize}

\bigskip


\bigskip

{\sffamily
Knowing all of these elements will \ make your experience more
successful as you work and interact with others in CVE. }


\bigskip


\bigskip

\clearpage\subsection{Chapter 1: Creating a New Account}
{\sffamily
This chapter details how \ to open CVE and create a New account in CVE.
In addition, since each new account also means that you have to create
your own avatar, this chapter will also describe the process of how to
create your own avatar. An avatar is a representation of yourself that
will appear to others in the CVE environment. It is important that you
spend some time creating your avatar, since you will be known through
your avatar in CVE.}


\bigskip

{\sffamily
The following things will be discussed in this chapter:}

\liststyleLii
\begin{itemize}
\item {\sffamily\bfseries\color[rgb]{0.0,0.0,0.5019608}
Step 1: Registering as a new User}
\item {\sffamily\bfseries\color[rgb]{0.0,0.0,0.5019608}
Step 2: Creating a New Avatar}
\item {\sffamily\bfseries\color[rgb]{0.0,0.0,0.5019608}
Avatar Creation Options}
\end{itemize}

\bigskip

{\sffamily\bfseries
\textit{NOTE:} }

{\sffamily
Since CVE requires each new user to create a new avatar, these steps are
mentioned in the order the user will have to do them. }


\bigskip

{\sffamily
To ensure quick access to those people familiar with the process of
creating an avatar, the steps to create a new avatar are given
separately from background information regarding this process. }


\bigskip

{\sffamily
Users wanting to have more background information about avatar creation
options are advised to read the \textbf{\textit{Avatar Creation
Options}} below first before commencing with individual steps below.}


\bigskip

\subsubsection{Registering as a New User}
{\sffamily
In order to access CVE, you{\textquoteright}ll need to create a Username
and Password. The following steps describe how you can create an
account that will let you log in to CVE.}


\bigskip

{\sffamily\bfseries\itshape\color[rgb]{0.0,0.2784314,1.0}
\ \ 1. Click on the Windows CVE Logo on your desktop to start CVE}

\begin{figure}[h]
\centering
\includegraphics[width=2.8984in,height=2.7736in]{userguid-img1.jpg}
\end{figure}
{\sffamily
The CVE software at various points was codenamed Unicron (after an
artificial planet from the Transformers television series and toys),
NSH (for New Science Hall at NMSU), and VIEW. You may at times still
see those names in various places in the software.}

\liststyleLiii
\begin{enumerate}
\item {\sffamily\bfseries\itshape\color[rgb]{0.0,0.2784314,1.0}
Click on {\textquotedblleft}REGISTER as a new user{\textquotedblright}}


\begin{figure}
\centering
\includegraphics[width=5.5in,height=3.2in]{userguid-img2.png}
\end{figure}
\end{enumerate}

\bigskip

{\sffamily
When you register as a new user, you will be taken to a screen that will
let you create your own avatar. Each new user will need to create a new
avatar as well. Follow the instructions for Creating a New Avatar in
CVE as described below. }


\subsubsection[Creating a New Avatar]{Creating a New Avatar}
{\sffamily
There are three steps to the process of Creating a New Avatar: }

\liststyleLiv
\begin{itemize}
\item {\sffamily
In the \textbf{first step}, you{\textquoteright}ll be asked to give some
personal information such as your first and last name, and by what name
you want to be identified in CVE, as well as your password. }
\item {\sffamily
In the \textbf{second step}, you{\textquoteright}ll define your body
type, head shape, and gender. You can also upload an image of your own
face so that others can see who you are. By default, your avatar has a
smiley face. In this step you{\textquoteright}ll also configure the
body, shirt, pant and shoe color of your avatar.}
\item {\sffamily
In the \textbf{third step}, you will be given a review of your answers
in STEP 1 and 2, and will be allowed to make any corrections before
creating your final avatar.}
\end{itemize}

\bigskip


\bigskip

\noindent {\sffamily\bfseries\itshape\color[rgb]{0.0,0.2784314,1.0}
1. Fill in your personal information and Click
{\textquotedblleft}Next{\textquotedblright}\newline
}

\liststyleLv
\begin{itemize}
\item {\sffamily
Fill in your \textbf{first} and \textbf{last name} in the upper two text
fields, as well as the \textbf{login id }(your user name in CVE) below.
\ }
\item {\sffamily
Fill in your \textbf{password} twice to ensure (once in the Password
field and once in the Re-enter field) to confirm that this is the
correct password you want to use. }


\hspace{0.5in}
\includegraphics[width=5.5in,height=3.8in]{userguid-img4.jpg}

\item {\sffamily
Fill in your \textbf{email address }and your \textbf{affiliation} (what
school you are attending or what organization is sponsoring you), as
well as your \textbf{gender}. When you have filled in all of this
information and have read the License agreement, click
\textbf{{\textquotedblleft}Next{\textquotedblright} }to proceed to the
Second Step of the Avatar Creation process, in which
you{\textquoteright}ll define how your avatar looks. }


\bigskip
\end{itemize}
{\sffamily\bfseries\itshape\color[rgb]{0.0,0.2784314,1.0}
2. Create your Avatar by selecting the height, body, color, and head
shape for your avatar}


\bigskip

{\sffamily\bfseries
\textit{Note:} }

{\sffamily\bfseries
\textmd{For a more in-delth explanation of your options in creating an
avatar, read the section below entitled }\textit{Avatar Creation
Options.}}

%\begin{figure}[h]
%\centering
\hspace{0.5in}
\includegraphics[width=5.0in,height=3.5in]{userguid-img5.jpg}
%\end{figure}

\bigskip

\noindent {\sffamily\bfseries\itshape\color[rgb]{0.0,0.2784314,1.0}
3. Press {\textquotedblleft}Done{\textquotedblright} button after
completing STEPS 1-2 to have CVE create your personal avatar with the
settings you just specified}

{\sffamily
You will see a review of what you just specified, as well as a current
view of your avatar in the right-hand side of the window. The following
image, for instance, will create a white-colored avatar with the
user.gif image for a blockhead shaped avatar.}

\bigskip

\hspace{0.6in}
\includegraphics[width=5.0in,height=3.5in]{userguid-img6.jpg}


\bigskip

\noindent {\sffamily\bfseries\itshape\color[rgb]{0.0,0.2784314,1.0}
4. After reviewing your information in Step 3, press
{\textquotedblleft}Submit{\textquotedblright} to create your avatar or
{\textquotedblleft}Make correction{\textquotedblright} to change your
avatar before submitting it.}




\bigskip

\noindent {\sffamily\bfseries
That{\textquotesingle}s it! After submitting, you will be taken
automatically into CVE with your new avatar.}


\bigskip

\subsubsection{\color[rgb]{0.0,0.0,0.5019608}Avatar Creation Options}

{\sffamily
In the preceding steps you are asked to create your avatar. An avatar is
the representation of you so that other users can recognize you when
you are in CVE. \ CVE lets you determine the following things about
your avatar:}

\liststyleLvii
\begin{itemize}
\item {\sffamily
\textbf{height} of your avatar }
\item {\sffamily
\textbf{body} \textbf{type} of your avatar }
\item {\sffamily\bfseries
color of your avatar{\textquotesingle}s body, shirt, pants and shoes }
\item {\sffamily
\textbf{head shape} of your avatar}
\item {\sffamily
\textbf{image} that you can use for your avatar{\textquoteright}s face.}
\end{itemize}

\bigskip

{\sffamily
At all times in this process, you can see what your avatar looks like in
the right hand side of the window, so any changes you make on the left
will be reflected in the way your avatar looks on the right.}

{\sffamily
\ It is important to spend some time creating an avatar that you like,
because others will recognize you only through your avatar. As they
say, first impressions are important! Below are some examples of the
various choices that you have in creating your avatar.}

\bigskip

\bigskip

{\centering\bfseries\itshape
Indicating the Height of your Avatar
\par}

{\sffamily
You{\textquoteright}ll need to give the height of your avatar. This is
done in feet and inches. Make sure to fill out both feet and inches. An
average height is about 6 feet, so you would fill out 6 feet a 0
inches. }

\vspace{0.16in}

{\centering\bfseries\itshape
Choosing a Body Type for your avatar
\par}

{\sffamily
There are a variety of sizes for your avatar: Very slim, Moderately
Slim, Average, Built, Big and Large. Below are three examples that
display the difference between the smallest and the largest avatar size
relative to each other. \ For instance, if you want your avatar to have
an average size, you would select
\textbf{{\textquotedblleft}Average{\textquotedblright}} in the
drop-box. If you want a large avatar, select
\textbf{{\textquotedblleft}Large{\textquotedblright}}.}

{\hspace{1.4in}
\begin{tabular}{c c c}
{\sffamily\bfseries\itshape Very Slim} &
{\sffamily\bfseries\itshape Average} &
{\sffamily\bfseries\itshape Large} \\

\includegraphics[width=1.028in,height=1.9583in]{userguid-img7.jpg} &
\includegraphics[width=1.222in,height=1.9445in]{userguid-img8.jpg} &
\includegraphics[width=1.5555in,height=1.9583in]{userguid-img9.jpg} \\
\end{tabular}
}

\vspace{0.24in}

{\centering\bfseries\itshape
Different Avatar Colors
\par}

{\sffamily
Next to the standard white color, you can choose between blue, green,
red, yellow, black, purple, pink and brown colors for your avatar}

{\hspace{1.5in}
\begin{tabular}{c c c c}
{\sffamily\bfseries\itshape Blue} &
{\sffamily\bfseries\itshape Green} &
{\sffamily\bfseries\itshape Red} &
{\sffamily\bfseries\itshape Yellow} \\

\includegraphics[width=0.8882in,height=0.8693in]{userguid-img10.jpg} &
\includegraphics[width=0.8398in,height=0.8508in]{userguid-img11.jpg} &
\includegraphics[width=0.7945in,height=0.8791in]{userguid-img12.jpg} &
\includegraphics[width=0.8902in,height=0.8602in]{userguid-img13.jpg}
\\

{\sffamily\bfseries\itshape Black} &
{\sffamily\bfseries\itshape Purple} &
{\sffamily\bfseries\itshape Pink} &
{\sffamily\bfseries\itshape Brown} \\

\includegraphics[width=0.8661in,height=0.978in]{userguid-img14.jpg} &
\includegraphics[width=0.8547in,height=0.9819in]{userguid-img15.jpg} &
\includegraphics[width=0.8453in,height=0.978in]{userguid-img16.jpg} &
\includegraphics[width=0.9043in,height=0.9398in]{userguid-img17.jpg}
\\

\end{tabular}
}

\bigskip \bigskip

{\centering\bfseries\itshape
Configuring the color for the body,
shirt, pants and shoes
\par}

{\sffamily
Below is an example of the different options that you have in
configuring different colors for the different body parts of your
avatar.}


\bigskip

{\hspace{0.6in}
\begin{tabular}{c c c c}
{\sffamily\bfseries\itshape Body color}  &
{\sffamily\bfseries\itshape Shirt color} &
{\sffamily\bfseries\itshape Pants color} &
{\sffamily\bfseries\itshape Shoes color} \\

\includegraphics[width=1.2555in,height=2.0063in]{userguid-img18.jpg} &
\includegraphics[width=1.3335in,height=2.0366in]{userguid-img19.jpg} &
\includegraphics[width=1.222in,height=2.0063in]{userguid-img20.jpg}  &
\includegraphics[width=1.2535in,height=1.9866in]{userguid-img21.jpg}
\\
\end{tabular}
}

\bigskip


{\centering\bfseries\itshape
Choosing the Head Shape of the Avatar
\par}

{\sffamily
You can choose between two different head shapes for your avatar: a
\textbf{\textit{Blockhead}} or an \textbf{\textit{Egghead}}. The
Blockhead will place a 128x128 (or 256x256) GIF image in front of your
avatar, whereas the Egghead will shape a GIF image around the head. The
Egghead requires that you create a specific image that will fit around
the Egghead, whereas the Blockhead can be any 128x128 (or 256x256)
image of you.}

\begin{center}
\tablehead{\multicolumn{1}{c}{\sffamily\bfseries\itshape \centering
\includegraphics[width=2.5in,height=2.5in]{userguid-img22.jpg}
} & \multicolumn{1}{c}{\sffamily\bfseries\itshape \centering
\includegraphics[width=2.5in,height=2.5in]{userguid-img23.jpg}
}\\}
\begin{supertabular}{ll}

\end{supertabular}
\end{center}
{\centering\bfseries\itshape
\ Uploading the Image for your avatar
\par}

{\sffamily
The \textbf{\textit{Blockhead}} image can be easily created from a
128x128 (or 256x256) image; The \textbf{\textit{Egghead }}image
requires a bit more work, but will give your avatar a more realistic
feel. If you want to get started right away, the blockhead image is the
easiest option, since it will only require a simple image, whereas the
Egghead image will need to be created in an image editor such as
Windows Paint, the GIMP, or Adobe Photoshop. In the Egghead you want
your face to occupy the lower two-thirds of the middle third of the
image. Most of the rest should be hair, but you could experiment with
putting on ears, and so forth.}


\bigskip

{\centering\bfseries\itshape
Choosing a Blockhead shape
\par}

{\sffamily
In order to choose the Blockhead shape follow the following steps:In
Step 2 of the Avatar Creation process, select
\textbf{{\textquotedblleft}Block{\textquotedblright}} in the Head shape
field. In the Select your square-head picture field, select
\textbf{{\textquotedblleft}Browse{\textquotedblright}} and locate the
image you would like to use for your avatar{\textquoteright}s face.
Click on {\textquotedblleft}Okay{\textquotedblright} to select the
image.}

{\sffamily
%\newline
This will put the image in front of your avatar{\textquoteright}s face.
If want to cancel selecting an image, press
{\textquotedblleft}Cancel{\textquotedblright} and select another
image.}

\bigskip

{\centering\bfseries\itshape
%\newline
Choosing the Egghead shape
\par}

{\sffamily
In order to choose the Egghead shape, select
\textbf{{\textquotedblleft}Egg{\textquotedblright}} in Head shape
field. If you are using an Egghead image, you{\textquoteright}ll need
to create one image that depicts your head as well as the back of your
head. If you have already created an egghead shape to be used, select
\textbf{{\textquotedblleft}Browse{\textquotedblright}} and select the
image that you would like to use. Select the image, then click
\textbf{{\textquotedblleft}Okay{\textquotedblright}}.}

\clearpage\subsection{Chapter 2: Logging In}
{\sffamily
This chapter details how to log into CVE. After you have created your
avatar, you can log in at any time using the userid and password that
you specified when you were creating your avatar. \newline
}

{\sffamily\textbf\textit{NOTE:}}

{\sffamily
If you do not yet have a User name and Password, click on
{\textquotedblleft}\textbf{\textit{REGISTER as a new
user}}{\textquotedblright} and follow the instructions for Creating a
New Account/Avatar in CVE. If you have already created one, and
forgotten your password and email, please \textbf{\textit{contact the
CVE system administrator at jeffery@cs.uidaho.edu}}}

\subsubsection{Logging in to CVE}

\bigskip

{\sffamily
In order to access CVE, you{\textquoteright}ll need to have create an
account with CVE. These steps are described in the next chapter\textbf{
Creating a New Account/Avatar in CVE}. The following steps describe how
you \ log in to CVE.}


\bigskip

\noindent {\sffamily\bfseries\itshape\color[rgb]{0.0,0.2784314,1.0}
1. Click on the Windows CVE Logo to start CVE}



\begin{figure}[h]
\centering
\includegraphics[width=2.5in,height=1.8in]{userguid-img24.jpg}
\end{figure}

\noindent {\sffamily\bfseries\itshape\color[rgb]{0.0,0.2784314,1.0}
2. Fill in your User name and Password and click
{\textquotedblleft}Go{\textquotedblright} }

\bigskip

{\sffamily
After clicking on Windows CVE, the CVE login screen will appear, and you
will need to enter your \textbf{User name} and \textbf{Password} in
order to access CVE.}


\bigskip

{\sffamily
By default, you do not need to select a server to be able to enter CVE.
CVE automatically selects a server for you. However, you can choose a
different server by clicking on the dropbox and selecting a specific
server. In case you are not able to log in to CVE, and if you are
wondering which server to select, select
\texttt{Idaho.}}


%\begin{figure}[h]
%\centering
\hspace{0.6in}
\includegraphics[width=5.1in,height=3.2in]{userguid-img25.jpg}
%\end{figure}

\bigskip

{\sffamily
%\newline
If you successfully log in, you will see the CVE interface and it will
visibly indicate that you are online. Otherwise it will say something
like "Login Failed".

\bigskip

Depending on where your avatar is located, you might see something like
the image below:}

\vspace{0.1in}

\hspace{0.25in}
\includegraphics[width=6in,height=4.14in]{userguid-img26.jpg}


\bigskip

\noindent {\sffamily\bfseries
Congratulations, you have now successfully logged into CVE! }

\subsubsection{Exiting CVE}
{\sffamily
At any time, you can also exit CVE by clicking
\textbf{\textit{{\textquotedblleft}File {\textgreater}
Exit{\textquotedblright}}}. Make sure to save your files first by
clicking \textbf{\textit{{\textquotedblleft}File {\textgreater}
Save{\textquotedblright}}} or \textbf{\textit{{\textquotedblleft}File
{\textgreater} Save As{\textquotedblright}}}}


\bigskip

{\sffamily\bfseries\itshape\color[rgb]{0.0,0.2784314,1.0}
To Exit CVE at any time simply click {\textquotedblleft}File
{\textgreater} Exit{\textquotedblright}}

\begin{center}
\tablehead{}
\begin{supertabular}{|m{3.38516in}|m{3.38446in}|}
\hline
{\sffamily\itshape Action} &
{\sffamily\itshape File Menu Bar Item}\\\hline
{\sffamily\itshape Exit CVE} &
{\sffamily\bfseries\itshape\color[rgb]{0.0,0.2784314,1.0} File
{\textgreater} exit}\\
\hline
\end{supertabular}
\end{center}

{\sffamily\bfseries
Make sure to familiarize yourself with the various tasks you can do in
CVE by reading the following chapters.}

\pagebreak

\subsection{Chapter 3: Using CVE }

{\sffamily
CVE is designed to let you utilize a variety of different options to
collaborate with others. Knowing these functions \ will allow you to
make use of these collaborative functions, which will enhance your
learning experience as you interact and work with others in CVE. This
chapter offers you quick descriptions of the main elements of CVE.
Rather than feeling like you should remember all of this information,
you should see (and read) \ this chapter as a first-time opportunity to
get to know how to use CVE in general, with more in-depth and detailed
instructions offered in individual chapters after this chapter.}


\bigskip

{\sffamily
CVE is equipped with the following capabilities:}

\liststyleLii
\begin{itemize}
\item {\sffamily\bfseries\color[rgb]{0.0,0.0,0.5019608}
A \ 3d environment that can be navigated }
\item {\sffamily\bfseries\color[rgb]{0.0,0.0,0.5019608}
Text chat}
\item {\sffamily\bfseries\color[rgb]{0.0,0.0,0.5019608}
User avatar movement and user-to-user avatar interaction}
\item {\sffamily\bfseries\color[rgb]{0.0,0.0,0.5019608}
Voice chat using Voice-Over-Internet Protocols (VOIP) area}
{\sffamily\bfseries\color[rgb]{0.5019608,0.0,0.0}
(disabled pending portability improvements)}
\item {\sffamily\bfseries\color[rgb]{0.0,0.0,0.5019608}
A File Menu Bar that lets you create and open Unicon/C/C++ and Java
\ projects}
\item {\sffamily\bfseries\color[rgb]{0.0,0.0,0.5019608}
Collaborative programming and program sharing using the CVE
Integrated Development Environment (IDE)}
\item {\sffamily\bfseries\color[rgb]{0.0,0.0,0.5019608}
File and class management \newline
}
\end{itemize}
{\sffamily
All of these capabilities are represented in the interface in specific
areas. Here is an overview: }

\begin{figure}[h]
\centering
\includegraphics[width=6.311in,height=3.4555in]{userguid-img27.jpg}
\end{figure}
{\sffamily
For your convenience, above is an example of what you might see after
you{\textquotesingle}ve logged in to CVE, as well as what function is
represented in each area. Each of these areas will be explained
separately below, \ with more in-depth instructions in individual
chapters following this chapter.}


\bigskip

\subsubsection{\color[rgb]{0.0,0.0,0.5019608}Using the 3D Environment}

{\sffamily
This part of CVE serves a variety of different purposes:}

\liststyleLviii
\begin{enumerate}
\item {\sffamily
By clicking on the \textbf{\textit{3D View Tab}}, you can navigate the
virtual environment. The 3D View tab generally has the name of the
virtual building or zone you are in, such as {\textquotedblleft}Janssen
Engineering Building{\textquotedblright}.}
\item {\sffamily
By clicking on the \textbf{\textit{Map Tab}}, \ you see where you are on
a 2D map.}
\end{enumerate}

\bigskip

{\sffamily\bfseries
\textit{NOTE:} }

{\sffamily
By default, the \textbf{\textit{3D View }}and \textbf{\textit{Map Tab}}
\textbf{\textit{are only displayed}} in this part of your screen. }


\bigskip

{\sffamily
When utilizing other functions, such as \textbf{\textit{opening a new
file}} or \textbf{\textit{starting a voice chat}}, other tabs will be
displayed in this area when these functions are used. }

{\sffamily
In addition to functioning as a navigation area, there are also multiple
other functions that can be accessed by clicking on the
\textbf{\textit{Tabs}} at the top of the 3D view area, allowing you to
switch between for example, the 3D navigation or watching a map where
you currently are. }

\begin{figure}[h]
\centering
\includegraphics[width=4.0492in,height=3.0882in]{userguid-img28.jpg}
\end{figure}

{\centering\bfseries\itshape
3D View Tab
\par}

{\sffamily
Using the \textrm{\textbf{\textcolor{black}{$\uparrow
$}}}\textcolor{black}{ }\textbf{Up},
\textrm{\textbf{\textcolor{black}{$\downarrow $}}} \textbf{Down},
\textrm{\textbf{\textcolor{black}{$\leftarrow $ \ }}}\textbf{Left} and
\textrm{\textbf{\textcolor{black}{$\rightarrow $ \ \ }}}\textbf{Right}
arrow keys \ will let you navigate throughout the virtual
representation of the world you are in, such as Janssen Engineering
Building at the University of Idaho, or Science Hall building at New
Mexico State University. \ In navigating through the virtual
environment, you{\textquotesingle}ll meet various other users, which
will lead you to get to know these users and collaborate with some of
the people that you meet.}

\bigskip

{\centering\bfseries\itshape
Map Tab
\par}

{\sffamily
%\newline
The \textbf{\textit{Map Tab}} displays where you are in the virtual
world, allowing you to see where other users are so that you may
interact with them.}

%\begin{figure}[h]
%\centering
\hspace{1.4in}
\includegraphics[width=3.6075in,height=2.2929in]{userguid-img29.jpg}
%\end{figure}


\bigskip

{\centering\bfseries\itshape
File Tab
\par}

{\sffamily
When you open a new or an existing project or file using the
\textbf{\textit{File Menu Bar}}, a new \textbf{File Tab }will be opened
above the 3D view area, allowing you to program your file using the 3D
view area. }

\bigskip

{\centering\bfseries\itshape
Voice Chat Tab
\par}

{\sffamily
When you engage in voice chat using the virtual cell phone, a
\textbf{Voice Tab} will be opened and a new \textbf{virtual cell phone
screen organizer} will be displayed in the 3D view area. You can hide
the area or turn off all calls to resume navigating the 3D view area. }


\bigskip

\subsubsection{Using the Text Chat Function}

{\sffamily
You can text chat with multiple users as well as a single user in CVE by
using the chat area. The chat interface contains two elements: }

\liststyleLix
\begin{enumerate}
\item {\sffamily
a \textbf{\textit{chat area /message window}} that will display chat
messages from others, as well as general program messages }
\item {\sffamily
\ a \textbf{\textit{chat input \ area}} where you can type messages to
others}



\begin{tabular}{l r}
\includegraphics[width=3.0in,height=2.25in]{userguid-img30.jpg} &
\includegraphics[width=3.0in,height=2.25in]{userguid-img31.jpg}
\end{tabular}
\end{enumerate}
{\sffamily
CVE uses chat commands to distinguish between messages intended for a
single user and multiple users. \ For this reason, CVE uses different
chat commands for chat messages intended for single and multiple
users.}

%\subsubsection{Using Voice Chat with the VOIP Function}

%\bigskip

%{\sffamily
%The Voice Over Internet Protocol (VOIP) function lets users voice chat
%with other users. \ Each user will need to have working speakers and a
%microphone in order to use voice chat. Just like a phone, it is
%possible to talk one-on-one, or to have conference calls with multiple
%users talking to each other. }

%\begin{figure}[h]
%\centering
%\includegraphics[width=4.2874in,height=3.4965in]{userguid-img32.jpg}
%\end{figure}

%\bigskip

%{\sffamily\color{black}
%The Voice Chat Area has four different functions:}

%\liststyleLx
%\begin{enumerate}
%\item {\sffamily\color{black}
%A \textbf{\textit{Turn Off Function}}}
%\item {\sffamily\color{black}
%A \textbf{\textit{Quick Phone Call function}} to quickly connect
%privately with other online users}
%\item {\sffamily\color{black}
%A \textbf{\textit{Public Conference Call Function}} to talk to whoever
%is present in the same room \ as you}
%\item {\sffamily\color{black}
%A \textbf{\textit{Virtual Cell Phone Function}} with the ability to have
%\textit{private} and \textit{public }calls by letting you select the
%users that you want to talk to regardless of their location }
%\end{enumerate}

%\bigskip

%{\sffamily\color{black}
%Using the Voice Chat, you can either talk to the people who are
%currently in the same room as your avatar by using the public
%conference call function, or have private calls or public conference
%calls with other users \ through the virtual cell phone. }

\bigskip

\subsubsection{Using the File Menu Bar}

\bigskip

{\sffamily\color{black}
You can open or create source code files, especially Java, C/C++ and
Unicon files and projects using the File Menu Bar. The File Menu Bar is
generally used to open and close files that you have created using
these programming languages. The File Menu Bar also lets you configure
the size of various elements of the screen.}


\bigskip

{\sffamily\color{black}
The File Menu Bar lets you do the following things:}


\bigskip

\liststyleLxi
\begin{itemize}
\item {\sffamily\bfseries\color{black}
Create/Open/Close/Save Files and Projects}
\item {\sffamily\bfseries\color{black}
Exit CVE}
\item {\sffamily\bfseries\color{black}
Configure the size of various elements of the screen}
\item {\sffamily\bfseries\color{black}
Create, Run, Compile Java/C/C++ and Unicon Files}
\item {\sffamily\bfseries\color{black}
Add pre-created Unicon code into your files/projects}
\item {\sffamily\bfseries\color{black}
Connect/Disconnect to the Server}
\item {\sffamily\bfseries\color{black}
Configure Account/Avatar Options}
\item {\sffamily\bfseries\color{black}
Consult Help Guides and Guidelines}
\end{itemize}

\bigskip

\begin{figure}[h]
\centering
\includegraphics[width=6.0272in,height=4.3689in]{userguid-img33.jpg}
\end{figure}

{\sffamily\color{black}
The following functions are contained in each menu item on the File Menu
Bar:}


\bigskip

\liststyleLxii
\begin{enumerate}
\item {\sffamily\color{black}
\textbf{FILE:} Open/ Close/ Save Files / Save Chat Transcripts / Take
Screenshot/ Exit}
\item {\sffamily
\textbf{\textcolor{black}{VIEW: }}Change the size of various screen
windows}
\item {\sffamily
\textbf{\textcolor{black}{EDIT: }}Cut/ Copy/ Paste/ Select All/ Find and
Replace/ Go to Line}
\item {\sffamily
\textbf{\textcolor{black}{INSERT:}} Insert pre-created Procedure/ Class/
Method Unicon Code }
\item {\sffamily
\textbf{\textcolor{black}{COMPILE:}} Make Executable/ Compile
Only/Compiler Options}
\item {\sffamily
\textbf{\textcolor{black}{RUN: }}Run Program }
\item {\sffamily
\textbf{\textcolor{black}{PROJECT: }}Create A New/ Open an existing
C/C++/Java/Unicon Project/ Compile a Project / Make Clean / Make Clean
All}
\item {\sffamily
\textbf{\textcolor{black}{NETWORK: }}Connect/Disconnect From the Server}
\item {\sffamily
\textbf{\textcolor{black}{ACCOUNT: }}Change
Password/Account/Avatar/Face}
\item {\sffamily
\textbf{\textcolor{black}{HELP: }}User Guide /Key Commands / About
CVE Version}
\end{enumerate}

\bigskip

\subsubsection{Using the Class/File Management and Collaboration
Functions}

\bigskip

{\sffamily
An important part of CVE is located in the class/file management and
collaboration area, which has three different functions: }

\liststyleLxiii
\begin{enumerate}
\item {\sffamily
The \textbf{\textit{Class Browser Tab}} lets you know what users are
online in your class}
\item {\sffamily
The \textbf{\textit{Users Tab}} lets you contact other users as well as
initiate collaborative sessions}

\begin{figure}[h]
\centering
\includegraphics[width=4.3362in,height=3.1791in]{userguid-img34.jpg}
\end{figure}

\item {\sffamily
The \textbf{\textit{Courses Tab}} informs you about which courses you
are signed up for}
\end{enumerate}

\bigskip

{\sffamily
This area is central to your experience in CVE, because it lets you see
which users are }

{\sffamily
present in the virtual environment so that you can work, chat or
collaborate with them. Typically, when you are using CVE, you will be
looking over here to see who you can chat with.}


\bigskip

{\sffamily
In addition, this area can display a tree showing files that are
available for editing using CVE{\textquotesingle}s
\textbf{\textit{Integrated Development Environment}} (IDE). The
collaborative area lets you \ create your programs using the IDE and
share your programs with other users in CVE, who can watch your screen
as you program, and give you feedback through text or voice chat.}

\clearpage\subsection[Chapter 4: \ Exploring the 3D Environment]{Chapter
4: \ Exploring the 3D Environment}
{\sffamily
This chapter describes how to move your avatar around in CVE. Moving
your avatar is easy. Generally, the \textbf{Up, Down, Left, Right} keys
are used for moving around CVE.}


\bigskip

{\sffamily
This chapter contains the following sections:}

\liststyleLxiv
\begin{itemize}
\item {\sffamily\bfseries\color[rgb]{0.0,0.0,0.5019608}
Moving Forward \& Backwards/ Turning Left \& Right/ Moving Sideways and
Looking Up \& Down}
\item {\sffamily\bfseries\color[rgb]{0.0,0.0,0.5019608}
Teleporting one{\textquotesingle}s avatar to meet other users}
\item {\sffamily\bfseries\color[rgb]{0.0,0.0,0.5019608}
Quicklist Avatar Movement}
\end{itemize}

\bigskip

{\sffamily\bfseries
\textit{NOTE}: }

{\sffamily
Next to moving around, it is also important to learn how to
\textbf{\textit{teleport}} one{\textquotesingle}s avatar, since this
will facilitate you meeting other users{\textquotesingle} avatars in a
quick fashion without having to look for each other first.}


\subsubsection{Moving, Turning, Moving Sideways and Looking Up \& Down}
{\sffamily
When you want to move your avatar throughout CVE, simply use the
\textrm{\textbf{\textcolor{black}{$\uparrow $}}}\textcolor{black}{
}\textbf{Up}, \textrm{\textbf{\textcolor{black}{$\downarrow $}}}
\textbf{Down}, \textrm{\textbf{\textcolor{black}{$\leftarrow $
\ }}}\textbf{Left} and \textrm{\textbf{\textcolor{black}{$\rightarrow $
\ \ }}}\textbf{Right} arrow keys to move forward, backwards and to turn
left and right. In addition, use \textbf{Pg Up / Numpad 9} and
\textbf{Pg Dn/Numpad 3} to look up and down.}

\begin{center}
\tablehead{ \hline
\multicolumn{1}{|m{1.1462599in}|}{\centering \sffamily\bfseries\itshape
Action} &
\multicolumn{1}{m{2.1441598in}|}{\centering \sffamily\bfseries\itshape
Keys/Alternate Keys} &
\multicolumn{1}{m{3.00456in}|}{\centering\arraybslash
\sffamily\bfseries\itshape Avatar Action}\\ \hline
}

\begin{supertabular}{m{1.1462599in}m{2.1441598in}m{3.00456in}}
\multicolumn{2}{m{3.3691597in}}{\hspace*{-\tabcolsep}
\begin{tabular}{|m{1.1462599in}|m{2.1441598in}}
\hline \\
{\sffamily\itshape\color{black}  Move Forward}
 &
{ \sffamily\bfseries\color{black} (Up Arrow) / \newline \newline
NumPad 8 } % \centering
\includegraphics[width=0.6945in,height=0.5in]{userguid-img35.jpg}
\ \newline
 \\  \hline
{ \sffamily\itshape\color{black} Move Backward} &
\ \newline
{ \sffamily\bfseries (Down Arrow) /  \newline  \newline
NumPad 2 } % \centering
\includegraphics[width=0.611in,height=0.5in]{userguid-img36.jpg}
\ \newline
 \\\hline
{ \sffamily\itshape\color{black} Turn Left } &
\ \newline
{ \sffamily\bfseries (Left Arrow) /  \newline \newline
NumPad 4 } % \centering
\includegraphics[width=0.7362in,height=0.5835in]{userguid-img37.jpg}
\ \newline
 \\\hline

{ \sffamily\itshape\color{black} Turn Right } &
\ \newline
{ \sffamily\bfseries (Right Arrow) /  \newline \newline
NumPad 6 } % \centering
\includegraphics[width=0.7917in,height=0.639in]{userguid-img38.jpg}
\ \newline
 \\\hline

{ \sffamily\itshape  Look up } &

\ \newline
{ \sffamily\bfseries\color{black}  NumPad 9/ \ Ctrl W }
\includegraphics[width=1.0555in,height=0.528in]{userguid-img39.jpg}
 \\\hline


{ \sffamily\itshape  Look down } &
\ \newline
{\sffamily\bfseries\color{black}  NumPad 3 / Ctrl S}
{ \centering
\includegraphics[width=0.9862in,height=0.5972in]{userguid-img40.jpg}}
 \\\hline

{\sffamily\itshape  Move Avatar Sideways Left }  &

\ \newline
\includegraphics[width=0.9902in,height=0.4in]{userguid-img41.jpg}

\\ \hline

{\sffamily\itshape  Move Avatar Sideways Right } &

\ \newline

\includegraphics[width=1in,height=0.4in]{userguid-img42.jpg}

 \\ \hline

\end{tabular} % \hspace*{-\tabcolsep}

} &
%\hspace*{-\tabcolsep}
\begin{tabular}{|m{3.00456in}|}
%\centering
\ \newline
\includegraphics[width=2.7in,height=4.5in]{userguid-img43.jpg}
\ \newline
\\\hline
\ \\
\ \\
{\sffamily\bfseries\color{black}  Look Up }
\ \\
\ \\
\\\hline
\ \\
\ \\
{ \sffamily\bfseries\color{black}  Look Down }
\ \\
\ \\
\\\hline
\ \\
{\centering
\includegraphics[width=1.4in,height=0.6in]{userguid-img44.jpg}}
 \\
\hline
 \\
{\centering
\includegraphics[width=1.4in,height=0.6in]{userguid-img45.jpg}}
\\\hline
\end{tabular}\hspace*{-\tabcolsep}
\\
\end{supertabular}
\end{center}



\subsubsection{Teleporting one{\textquotesingle}s avatar to meet other
users}

\bigskip

{\sffamily
Next to moving one{\textquotesingle}s avatar, it is also possible to
\textbf{\textit{teleport}} to a stable point that is the same for
everybody. By using the teleport function you can quickly meet with
someone you know is online, but who is not in your immediate vicinity.
}


\bigskip

{\sffamily\bfseries
\textit{NOTE:} }

{\sffamily
\textbf{\textit{The Teleport Function}} is especially helpful when you
want to meet another user without having to find each other first.
Instead, you can text chat that user and ask them if they would like to
transport to the beginning of science hall. }


\bigskip

{\sffamily
In doing so, you can immediately start interacting with the
user{\textquotesingle}s avatar or engage in a conference room call.
\textbf{\textit{The Teleport function is an easy way to get in touch
with someone else{\textquotesingle}s avatar}} \textbf{\textit{when you
are looking for each other but cannot find each
other{\textquotesingle}s avatar immediately. }}}


\bigskip


\bigskip

\begin{center}
\tablehead{\hline
\sffamily\bfseries \ \ Teleport: &
\centering
\includegraphics[width=0.85in,height=0.3299in]{userguid-img46.jpg}
\\%\hline
}
%\begin{supertabular}{|m{3.38376in}|m{3.38376in}|}

%\end{supertabular}
\end{center}

{\sffamily
When you utilize the teleport function, you are immediately taken to the
entrance to Science Hall. This is incredibly useful when you want to
meet with other users to have face-to-face avatar interaction or
conference calls.}

\subsubsection[Quicklist Avatar Movement]{Quicklist Avatar Movement}
{\sffamily
Here{\textquotesingle}s quick list of the controls to moving your
avatar,The function of these controls are described in more detail
above.\newline
}

\begin{center}
\tablehead{\hline
\centering \sffamily\bfseries\itshape Action &
\centering \sffamily\bfseries\itshape Keys &
\centering\arraybslash \sffamily\bfseries\itshape Avatar Action\\\hline}
\begin{supertabular}{|m{2.22956in}|m{2.22956in}|m{2.22956in}|}
\sffamily\itshape\color{black} Move Forward &
\sffamily\color{black}  \textbf{(Up Arrow) / NumPad 8}\centering
\includegraphics[width=0.6945in,height=0.5in]{userguid-img47.jpg}
 &
{\sffamily\color{black} Moves avatar forward}

~
\\\hline
\sffamily\itshape\color{black} Move Backward &
\sffamily \textrm{\textbf{\textcolor{black}{ \ }}}\textbf{(Down Arrow) /
NumPad 2}\centering
\includegraphics[width=0.611in,height=0.5in]{userguid-img48.jpg}
 &
{\sffamily\color{black} Moves avatar backward}

~
\\\hline
\sffamily\itshape\color{black} Turn Left &
\sffamily \textrm{\textbf{\textcolor{black}{ \ }}}\textbf{(Left Arrow) /
NumPad 4}\centering
\includegraphics[width=0.7362in,height=0.5835in]{userguid-img49.jpg}
 &
\sffamily\color{black} Turns avatar left\\\hline
\sffamily\itshape\color{black} Turn Right &
\sffamily\bfseries (Right Arrow) / NumPad 6\centering
\includegraphics[width=0.7917in,height=0.639in]{userguid-img50.jpg}
 &
\sffamily\color{black} Turns avatar right\\\hline
\sffamily\itshape\color{black} Move Avatar Sideways Left &
\centering
\includegraphics[width=1.2598in,height=0.4in]{userguid-img51.jpg}
 &
\sffamily\color{black} Avatar moves sideways left\\\hline
\sffamily\itshape Move Avatar Sideways Right &
\centering
\includegraphics[width=1.4402in,height=0.4in]{userguid-img52.jpg}
 &
{\sffamily Avatar moves sideways right}\\\hline
{\sffamily\itshape Look Up} &
{\sffamily\bfseries\color{black} NumPad 9/ \ Ctrl W} \centering
\includegraphics[width=0.8402in,height=0.4in]{userguid-img53.jpg}
 &
{\sffamily Looks up} \newline
(First Person Only)\\\hline
{\sffamily\itshape Look Down} &
{\sffamily\bfseries\color{black}
 NumPad 3 / Ctrl S}
\includegraphics[width=0.8909in,height=0.511in]{userguid-img54.jpg}
 &
{\sffamily Looks down} \newline
(First Person Only)\\\hline

{\sffamily\itshape Teleport} &

\includegraphics[width=1.4902in,height=0.4in]{userguid-img55.jpg}
&

{\sffamily\color{black} Teleport to the beginning of Science Hall}

\\\hline

%}\\
\end{supertabular}
\end{center}

\bigskip


\bigskip


\bigskip

\clearpage

\subsection[Chapter 5: \ Interacting with Other Users]{Chapter
5: \ Interacting with Other Users}
{\sffamily
This chapter explains how to interact with other users in CVE. Because
you will be working collaboratively with other users on a variety of
projects, you will be expected to know how to interact with other
virtual users through your avatar. CVE is created for multiple users to
engage in collaborative interaction, so it is important that you are
aware of your avatar{\textquotesingle}s functions in interacting with
other users. \ A lot of these interactive functions are actived through
key combinations usually involving CTRL and a Letter Key.}


\bigskip

{\sffamily\bfseries\itshape NOTE: }

{\sffamily
A lot of the below actions serve a social purpose: to make other users
aware of you so that you can exchange particular kinds of information,
such as waving at another user so you can initiate a chat session with
them, or pointing at a particular object in CVE. }


\bigskip

{\sffamily
Just like you do in the real world, you can interact with others users
through these kinds of social signals.
This chapter describes the following elements:}

\liststyleLii
\begin{itemize}
\item {\sffamily\bfseries\color[rgb]{0.0,0.0,0.5019608}
Opening/Closing Doors/ Knocking on Doors/ Saying Hello \& Goodbye to
Other Users}
\item {\sffamily\bfseries\color[rgb]{0.0,0.0,0.5019608}
Switching between first and third person views}
\item {\sffamily\bfseries\color[rgb]{0.0,0.0,0.5019608}
Greeting Someone/Getting Someone{\textquotesingle}s Attention/Giving
Thumbs Up}
\item {\sffamily\bfseries\color[rgb]{0.0,0.0,0.5019608}
Pointing at Something/Someone (Up/Down/Left/ Right/Forward)}
\item {\sffamily\bfseries\color[rgb]{0.0,0.0,0.5019608}
How Your Avatar Actions Appear to Other Users}
\item {\sffamily\bfseries\color[rgb]{0.0,0.0,0.5019608}
Quicklist Opening/Closing Doors/ Knocking on Doors/ Saying Hello and
Goodbye}
\item {\sffamily\bfseries\color[rgb]{0.0,0.0,0.5019608}
Quicklist Avatar Interactions }
\end{itemize}
\subsubsection{Opening/Closing Doors/ Knocking on Doors/ Saying Hello \&
Goodbye to Other Users}

\bigskip

{\sffamily
Next to moving around, you{\textquotesingle}ll also be asked to interact
with various objects in the virtual environment. One of the key
components in the virtual environment is being able to open doors, as
well as announcing your presence to other users. While it is relatively
easy to knock on a door in the real world so that others notice you are
there, in the virtual environment a variety of key combinations is used
to open a door as well as announcing your presence to other users. It
is important that you do so, since it will not always be clear to other
users that a new person has entered the room. \ }


\bigskip

{\centering\bfseries\itshape
Opening/Closing a Door
\par}

{\sffamily
It is important to be able to open a door. The following key combination
will open as well as close doors:}


\bigskip


\bigskip

\begin{center}
\tablehead{\hline
\sffamily\itshape \ To open or close a door:  &
\centering
\includegraphics[width=1.0402in,height=0.3598in]{userguid-img56.jpg}
\\\hline}
%\begin{supertabular}{|m{3.38376in}|m{3.38376in}|}

%\end{supertabular}
\end{center}

\bigskip

{\centering\bfseries\itshape
Knocking on a Door
\par}


\bigskip

{\sffamily
It is also possible to knock on the door first to announce your
presence. This lets other users know that there is a new person that
they can interact with, as well as allowing you to join smoothly with
other users in the virtual environment.}


\bigskip


\bigskip

\begin{center}
\tablehead{\hline
\sffamily\itshape To knock on a door: \  &
\centering
\includegraphics[width=0.9701in,height=0.4in]{userguid-img57.jpg}
\\\hline}
%\begin{supertabular}{|m{3.38376in}|m{3.38376in}|}

%\end{supertabular}
\end{center}

\bigskip

{\centering\bfseries\itshape
Saying {\textquotedblleft}Hello{\textquotedblright} and
{\textquotedblleft}Goodbye{\textquotedblright}\newline

\par}

{\sffamily
Next to knocking on a door, after you have opened a door, it is also
possible to acknowledge when you have just entered a room \ by playing
a {\textquotedblleft}Hello{\textquotedblright} sound. This is a good
way of letting other users know that a new person has entered their
room with whom they can interact. In addition, it is also important to
let other people know when you are leaving a room by playing a
{\textquotedblleft}goodbye{\textquotedblright} sound.}

\bigskip

{\sffamily\itshape To say hello:}

\includegraphics[width=1.0402in,height=0.3799in]{userguid-img58.jpg}

\bigskip

{\sffamily \textit{To say goodbye:}}

\includegraphics[width=1.0201in,height=0.35in]{userguid-img59.jpg}


\bigskip

{\sffamily
Knowing these functions will let you enter and exit rooms in a way so
that other will notice.}

\subsubsection{Switching between First and Third Person views}

{\sffamily\color{black}
In CVE it is possible to switch between first and third person views.
Just like saying {\textquotedblleft}I{\textquotedblright} or
{\textquotedblleft}He/She{\textquotedblright} is used to differentiate
between yourself and others, first person and third person views are a
matter of perspective that give you different options in the virtual
environment. Whereas first person is used exclusively to look at
objects from a single person perspective - \ a third person view is a
view in which your own avatar is included, allowing you to make
gestures and point at objects and other users.}


\bigskip

\liststyleLxv
\begin{itemize}
\item {\sffamily\color{black}
A \textbf{first person view} - a subjective view from where your avatar
is standing, giving you the opportunity to examine objects closely}
\begin{figure}[h]
\centering
\includegraphics[width=3.6736in,height=2.8638in]{userguid-img60.jpg}
\end{figure}
\item {\sffamily\color{black}
A \textbf{third person view} - an objective view that includes your
avatar as seen from behind, giving you more contextual information,
such as indicating where your avatar is standing in relation to other
objects and users in CVE, allowing you to make gestures to other users
or point at particular objects.}


\bigskip
\end{itemize}
{\sffamily\bfseries\color{black}
\textit{NOTE}: }

{\sffamily\color{black}
An important function in CVE is knowing how to \textbf{\textit{switch
between}} \textbf{\textit{First person}} \textbf{\textit{and}}
\textbf{\textit{Third person avatar views}} (\textbf{\textit{Ctrl +
V}}), because these allow you to do different things in CVE. \ }

{\sffamily\color{black}
The \textbf{\textit{First person}} view lets you examine things closely,
whereas the \textbf{\textit{Third person}} view lets you make gestures
and point at objects and users.}

\liststyleLxv


\begin{center}
\begin{tabular}{|m{2.85666in}|m{2.85736in}|}
\hline
 First Person Perspective & Third Person Perspective\\
\hline
\includegraphics[width=2.7547in,height=2.098in]{userguid-img61.jpg} &
\includegraphics[width=2.6882in,height=2.0807in]{userguid-img62.jpg}\\
\hline
First person view is a \textbf{subjective} view that
lets you examine objects closely but does not give as wide a
perspective as third person & Third person views is an \textbf{objective} view that
includes your avatar as seen from behind and in relation to other users
and objects\\
\hline
\end{tabular}
\end{center}



\bigskip

{\sffamily\color{black}
The importance of first person and third person views is that the third
person view gives you the opportunity to do specific avatar actions,
which are described in detail below.}


\bigskip

\subsubsection{Greeting Someone/Getting Someone{\textquotesingle}s
Attention/Giving Thumbs Up}
{\sffamily
These movements can be used to signal that you are trying to get another
user{\textquotesingle}s attention by waving at them (pressing
\textbf{Ctrl E}), or simply to give them a thumbs up (\textbf{Ctrl U}).
}

\begin{table}[h]
\centering
\begin{tabular}{|m{2.21436in}|m{1.9781599in}|m{1.7386599in}|}
\hline
\centering \sffamily\bfseries\itshape Action - Connotation &
\centering \sffamily\bfseries\itshape Keys &
\centering\arraybslash \sffamily\bfseries\itshape Third Person
View\\
\hline
{\sffamily\itshape Put left arm up}
%
%~
%
{\sffamily\itshape Greet someone}
%
%~
%
\sffamily\itshape Get their attention &
%\centering
\includegraphics[width=1.1in,height=0.3799in]{userguid-img63.jpg}
 &
%\centering
\includegraphics[width=1.5346in,height=1.2953in]{userguid-img64.jpg}
\\
 \hline
{\sffamily\itshape Put right arm up}

%~

\sffamily\itshape Give someone thumbs up/ Approve &
%\centering
\includegraphics[width=1.15in,height=0.4201in]{userguid-img65.jpg}
 &
%\centering
\includegraphics[width=1.5346in,height=1.3228in]{userguid-img66.jpg}
\\
\hline
\end{tabular}
\end{table}



\subsubsection{Pointing at Something/Someone}
{\sffamily
%\newline
Your right arm has \ the capability of moving in an \textbf{Up/Down
moving direction}, so that you can point at a particular object or
person in CVE. Your avatar{\textquotesingle}s right arm can also move
\textbf{Left, Right and Forward} to allow for pointing to objects or
persons on the left/right/forward side. Below are the function keys for
making your avatar perform these actions.}


\bigskip

\subsubsection[Pointing at Something/Someone Up/Down]{Pointing at
Something/Someone Up/Down}

\begin{table}[h]
\centering
\begin{tabular}{|m{1.94976in}|m{2.2275598in}|m{1.7386599in}|}
\hline
\centering \sffamily\bfseries\itshape Action-Meaning &
\centering \sffamily\bfseries\itshape Keys &
\centering\arraybslash \sffamily\bfseries\itshape Third Person
View\\\hline
{\sffamily\itshape Point up slowly with right arm }

~

\sffamily\itshape Point up to something/someone &
%\centering
\includegraphics[width=1.1in,height=0.4in]{userguid-img67.jpg}
 &
%\centering
\includegraphics[width=1.6055in,height=1.1965in]{userguid-img68.jpg}
\\
\hline
{\sffamily\itshape Point down slowly with right arm }

~

\sffamily\itshape Point down to something/someone &
%\centering
\includegraphics[width=1.0598in,height=0.4in]{userguid-img69.jpg}
 &
%\centering
\includegraphics[width=1.6366in,height=1.1689in]{userguid-img70.jpg}
\\
\hline
\end{tabular}
\end{table}

\bigskip

\subsubsection[Pointing at Something/Someone Left/
Right]{\sffamily\bfseries\color[rgb]{0.0,0.0,0.5019608} Pointing at
Something/Someone Left/ Right}
\begin{table}[h]
\centering
\begin{tabular}{|m{1.9365599in}|m{2.22816in}|m{1.7379599in}|}
\hline
\centering \sffamily\bfseries\itshape Action-Meaning &
\centering \sffamily\bfseries\itshape Keys &
\centering\arraybslash \sffamily\bfseries\itshape Third Person
View\\\hline
{\sffamily\itshape Point Right}

~

\sffamily\itshape Point to something/ someone on the right &
%\centering
\includegraphics[width=1.0402in,height=0.3701in]{userguid-img71.jpg}
 &
%\centering
\includegraphics[width=1.5846in,height=1.1008in]{userguid-img72.jpg}
\\
\hline
{\sffamily\itshape Point Left}

~

\sffamily\itshape Point to something/ someone on the left &
%\centering
\includegraphics[width=1.0402in,height=0.4402in]{userguid-img73.jpg}
 &
%\centering
\includegraphics[width=1.5953in,height=1.1165in]{userguid-img74.jpg}
\\
\hline
\end{tabular}
\end{table}

\bigskip

\subsubsection[Pointing at Something/Someone in front of you]{Pointing
at Something/Someone in front of you}


\begin{tabular}{|m{1.9226599in}|m{2.22816in}|m{1.7386599in}|}
\hline
{\centering \sffamily\bfseries\itshape Action-Meaning} &
{\centering \sffamily\bfseries\itshape Keys} &
{\centering\arraybslash \sffamily\bfseries\itshape Third Person
View}\\\hline
{\sffamily\itshape Point Forward }


{\sffamily\itshape Point to something/someone in front of you}

{\sffamily\itshape Give someone a turn}

 &
{\centering
\includegraphics[width=0.9598in,height=0.3799in]{userguid-img75.jpg}}
 &

\includegraphics[width=1.5744in,height=1.2035in]{userguid-img76.jpg}
\\
\hline
\end{tabular}


\subsubsection{How Your Avatar Actions Appear to Other Users}

\bigskip

{\sffamily
Below is an overview of how the above actions appear to other users. It
is important for you to know, since other users will see your actions
from this perspective. }


\bigskip

{\sffamily\bfseries\itshape
NOTE: }

{\sffamily\itshape
Keep in mind that it takes some time to point at something to another
user, since it will not always be immediately clear to them where you
are pointing. }


\bigskip

{\centering\bfseries\itshape
Greeting/Getting Someone{\textquotesingle}s Attention
\par}

\begin{center}
\tablehead{\hline
{\centering \sffamily\bfseries\itshape Action} &
{\centering \sffamily\bfseries\itshape Keys} &
{\centering\arraybslash \sffamily\bfseries\itshape View by Other
User}\\
\hline}
\begin{supertabular}{|m{1.6927599in}|m{2.2400599in}|m{1.7302599in}|}
{\sffamily\itshape Greet\newline
\newline
Get Someone{\textquotesingle}s Attention} &

\includegraphics[width=1.0799in,height=0.3701in]{userguid-img77.jpg}
 &
\includegraphics[width=1.7in,height=1.7in]{userguid-img78.jpg}
\\
\hline
{\sffamily\itshape Give someone thumbs up

 Approve

 Ask to say something} &
{\centering
\includegraphics[width=1.1299in,height=0.4098in]{userguid-img79.jpg}}
 &
\includegraphics[width=1.7in,height=1.7in]{userguid-img80.jpg}
\\
\hline
\end{supertabular}
\end{center}

\bigskip


\bigskip

{\centering\bfseries\itshape Pointing Up/Down to Something/Someone \par}


\bigskip

\begin{center}
\tablehead{\hline
\centering \sffamily\bfseries\itshape Action &
\centering \sffamily\bfseries\itshape Keys &
\centering\arraybslash \sffamily\bfseries\itshape View by Other
User\\
\hline}
\begin{supertabular}{|m{1.7129599in}|m{2.2268598in}|m{1.7511599in}|}
\sffamily\itshape Point up to something/\newline
someone &
{\centering
\includegraphics[width=0.95in,height=0.3701in]{userguid-img81.jpg}}
 &
{\centering
\includegraphics[width=1.7063in,height=1.6717in]{userguid-img82.jpg}}
\\
\hline
\sffamily\itshape Point down to something/\newline
someone &
{\centering
\includegraphics[width=1.0098in,height=0.4098in]{userguid-img83.jpg}}
 &
{\centering
\includegraphics[width=1.7547in,height=1.7193in]{userguid-img84.jpg}}
\\
\hline
\end{supertabular}
\end{center}

\bigskip


\bigskip

{\centering\bfseries\itshape
%\newline
Pointing Left/Right to Something/Someone
\par}


\bigskip

\begin{center}
\tablehead{\hline
\centering \sffamily\bfseries\itshape Action &
\centering \sffamily\bfseries\itshape Keys &
\centering\arraybslash \sffamily\bfseries\itshape View by Other
User\\
\hline}
\begin{supertabular}{|m{1.7406598in}|m{2.22816in}|m{1.7538599in}|}
{\sffamily\itshape Point to something/}

\sffamily\itshape someone on the left &
\centering
\includegraphics[width=0.9201in,height=0.4in]{userguid-img85.jpg}
 &
%\centering
\includegraphics[width=1.7063in,height=1.7409in]{userguid-img86.jpg}
\\
\hline
{\sffamily\itshape Point to something/}

\sffamily\itshape someone on the right &
\centering
\includegraphics[width=0.9402in,height=0.3598in]{userguid-img87.jpg}
 &
%\centering
\includegraphics[width=1.7063in,height=1.6992in]{userguid-img88.jpg}
\\
\hline
\end{supertabular}
\end{center}

\bigskip

{\centering\bfseries\itshape
Pointing to Something/Someone In Front of You
\par}


\bigskip

\begin{center}
\tablehead{\hline
\centering \sffamily\bfseries\itshape Action &
\centering \sffamily\bfseries\itshape Keys &
\centering\arraybslash \sffamily\bfseries\itshape View by Other
User\\
\hline}
\begin{supertabular}{|m{1.7406598in}|m{2.2268598in}|m{1.7511599in}|}
{\sffamily\itshape Point to something/}

\sffamily\itshape someone in front of you &
\centering
\includegraphics[width=2.2417in,height=0.889in]{userguid-img89.jpg}
 &
%\centering
\includegraphics[width=1.7063in,height=1.7409in]{userguid-img90.jpg}
\\
\hline
\end{supertabular}
\end{center}

\bigskip

\clearpage
\bigskip

\subsubsection{Quicklist Opening/Closing Doors/ Knocking on Doors/
Saying Hello and Goodbye}
{\sffamily
Here{\textquotesingle}s a quicklist for \ opening/closing /knocking on
doors, as well as saying {\textquotedblleft}Hello{\textquotedblright}
and {\textquotedblleft}Goodbye{\textquotedblright} to other users}

\begin{center}
\tablehead{\hline
\centering \sffamily\bfseries\itshape Action &
\centering \sffamily\bfseries\itshape Keys &
\centering\arraybslash \sffamily\bfseries\itshape Action\\
\hline}
\begin{supertabular}{|m{1.1018599in}|m{2.10526in}|m{2.51506in}|}
{\sffamily\itshape Open/\newline
Close door}

~
 &
\centering
\includegraphics[width=2.1071in,height=0.7335in]{userguid-img91.jpg}
 &
{\sffamily\bfseries\color{black} Opens/ Closes Door}\\

\hline

{\sffamily\itshape Knock on Door}
 &
\includegraphics[width=2.0835in,height=0.8472in]{userguid-img92.jpg}
 &
{\sffamily\bfseries\color{black} Plays {\textquotedblleft}Knock on
Door{\textquotedblright} sound}\\

\hline

{\sffamily\itshape Say
{\textquotedblleft}Hello{\textquotedblright}}
&
\includegraphics[width=2.1071in,height=0.7661in]{userguid-img93.jpg}
&
{\sffamily\bfseries\color{black} Plays
{\textquotedblleft}Hello{\textquotedblright} sound so other users know
you are entering the room}

\\

\hline

{\sffamily\itshape Say {\textquotedblleft}Goodbye{\textquotedblright}}
&
	\includegraphics[width=2.1071in,height=0.7244in]{userguid-img94.jpg}
&
{\sffamily\bfseries\color{black} Plays
{\textquotedblleft}Goodbye{\textquotedblright} sound so other users
know you are leaving the room}

\\
\hline


\end{supertabular}
\end{center}

\bigskip


\bigskip


\bigskip


\bigskip

\subsubsection[Quicklist Avatar Interactions ]{Quicklist Avatar
Interactions }
{\sffamily
Here{\textquotesingle}s a quicklist for all of the actions described.
More full descriptions for these functions are described above. }

{\sffamily
\textbf{\textit{NOTE:}} }

{\sffamily
Because many of these \textbf{\textit{social gestures are not visible
when you are in first person mode}}, \textbf{\textit{they can only be
done in third person}}. }

{\sffamily
Hence, the chart below indicates the action, the keys that cause that
action to happen, as well as the view in which these actions are
allowed.}


\bigskip

\begin{center}
\tablehead{\hline
\centering \sffamily\bfseries\itshape Action &
\centering \sffamily\bfseries\itshape Keys &
\centering\arraybslash \sffamily\bfseries\itshape VIEW\\
\hline}
\begin{supertabular}{|m{2.22956in}|m{2.22956in}|m{2.2351599in}|}
\sffamily\itshape Switch First/Third Person Avatar View &


\centering
\includegraphics[width=1.5091in,height=0.4055in]{userguid-img95.jpg}
~
 &
\sffamily First \& Third Person\\
\hline
{\sffamily\itshape Put left arm up }

~

~
 &
\centering
\includegraphics[width=1.4992in,height=0.4709in]{userguid-img96.jpg}
 &
\sffamily Third Person only\\
\hline
\sffamily\itshape Put right arm up (thumbs up) &
\centering
\includegraphics[width=1.511in,height=0.578in]{userguid-img97.jpg}
 &
\sffamily Third Person only\\
\hline
\sffamily\itshape Point right arm up &
\centering
\includegraphics[width=1.428in,height=0.5602in]{userguid-img98.jpg}
 &
\sffamily %\newline
Third Person only\\
\hline
\sffamily\itshape Point right arm down &
\centering
\includegraphics[width=1.4181in,height=0.5047in]{userguid-img99.jpg}
 &
\sffamily Third Person only\\
\hline
\sffamily\itshape Point left with right arm &
\centering
\includegraphics[width=1.3638in,height=0.5327in]{userguid-img100.jpg}
 &
\sffamily Third Person only\\
\hline
\sffamily\itshape Point right with right arm  &
\centering
\includegraphics[width=1.4346in,height=0.5165in]{userguid-img101.jpg}
 &
\sffamily % \newline
Third Person only\\
\hline
\sffamily\itshape Point right arm forward &
\centering
\includegraphics[width=1.4555in,height=0.5028in]{userguid-img102.jpg}
 &
\sffamily Third Person only\\
\hline
\end{supertabular}
\end{center}

\bigskip

\clearpage\subsection{Chapter 6: Using Text Chat}

\bigskip

{\sffamily
This chapter describes the text chat function. In CVE it is possible to
chat with single or multiple users. \ }

{\sffamily
This chapter contains the following:}

\liststyleLii
\begin{itemize}
\item {\sffamily\bfseries\color[rgb]{0.0,0.0,0.5019608}
Creating Single User Chat Messages}
\item {\sffamily\bfseries\color[rgb]{0.0,0.0,0.5019608}
Creating Multiple User Chat Messages}
\item {\sffamily\bfseries\color[rgb]{0.0,0.0,0.5019608}
Seeing Who is Logged In}
\item {\sffamily\bfseries\color[rgb]{0.0,0.0,0.5019608}
Quicklist Chat Commands}
\end{itemize}

\bigskip

{\sffamily\bfseries
\textit{NOTE}: }

{\sffamily
It is easy to use text chat, simply type in a chat command and your
message in the \textbf{\textit{Chat input area}}\textbf{,} and it will
show up in the \textbf{\textit{Chat Area/ Message Window}}. }


\bigskip

{\sffamily
However, since a distinction is made between messages between single and
multiple users, CVE uses the {\textbackslash}tell command to address a
single user. }

\begin{figure}
\centering
\includegraphics[width=4.2772in,height=3.1484in]{userguid-img103.jpg}
\end{figure}


\begin{figure}
\centering
\includegraphics[width=3.1299in,height=3.0516in]{userguid-img104.jpg}
\end{figure}
\subsubsection{Creating Single user Chat Messages}
{\sffamily
It is also possible to have a single user chat by telling an individual
user only. To tell only a single specific user, simply format it like
this:}


\bigskip

{\sffamily\bfseries
\ \ {\textbackslash}tell \texttt{[user name] [type your message here]
\ [Press {\textquotedblleft}Enter{\textquotedblright}]}}


\bigskip

{\sffamily
typing in:}


\bigskip

{\ttfamily
\ \ {\textbackslash}tell jeffery How are you?\newline
}

{\sffamily
in the chat input area would \textit{only} tell user Jeffery this
question. This means that only Jeffery would receive this message. If I
wanted to ask everybody that is currently logged in, I would use the
\textbf{\textit{Multiple User Chat}} format as described below.}

\subsubsection{Creating Multiple User Chat Messages}
{\sffamily
It is possible to send chat messages to multiple users at once.
Generally, a multiple user text chat is formatted like this:}


\bigskip

{\sffamily\bfseries
\ \ \texttt{[type your message here] [Press
{\textquotedblleft}Enter{\textquotedblright}]}}


\bigskip

{\sffamily
An example:}

{\sffamily
typing in: }

{\ttfamily
\ Hello everybody! [Press Enter]}


\bigskip

{\sffamily
In the chat input area would therefore create a text message in the Chat
Area/ Message Window \ from you that reads
{\textquotedblleft}\texttt{Hello everybody!}{\textquotedblright} and
would be received by all users currently logged in.}


\bigskip

\subsubsection{Seeing Who is Logged in}
{\sffamily
One way to check who is on-line is}


\bigskip

{\sffamily\bfseries
\ \ \ {\textbackslash}who}


\bigskip

{\sffamily
At present this does not tell you too much, in future it will be
expanded to include where they are located, what they are doing, and so
forth.}


\bigskip

\subsubsection{Quicklist Chat Commands}

\bigskip

{\sffamily
Here{\textquotesingle}s a quick list of the chat commands described
above. More full descriptions of these commands are found above.}


\bigskip

\begin{center}
\tablehead{\hline
\centering \sffamily\bfseries\itshape Chat Message &
\centering\arraybslash \sffamily\bfseries\itshape Chat Command
Format\\
\hline}
\begin{supertabular}{|m{2.8386598in}|m{2.94486in}|}
\sffamily\itshape Single User &
{\ttfamily {\textbackslash}tell [username] [message]}

\ttfamily [Press {\textquotedblleft}Enter{\textquotedblright}
Key]\\
\hline
\sffamily\itshape Multiple Users &
\ttfamily [message] [Press {\textquotedblleft}Enter{\textquotedblright}
Key]\\
\hline
\sffamily\itshape Check Who is Logged In &
\ttfamily {\textbackslash}who [Press
{\textquotedblleft}Enter{\textquotedblright} Key]\\\hline
\end{supertabular}
\end{center}

\bigskip

\clearpage\subsection{Chapter 7: Using Voice Chat to Talk to Other
Users}

\bigskip

{\sffamily (This chapter is removed pending a portable implementation.
Our Voip facilities were built on a C++ voip library
that did not meet its portability claims.)

% {\sffamily
% This chapter describes how to use the Voice Chat function so that you
% can voice chat in real-time with other users who are logged. Voice Chat
% uses your computer as a telephone, so that anybody equipped with
% speakers and a microphone can have a PC-to-PC phone conversation with
% you via \textbf{V}oice-\textbf{O}ver \textbf{I}nternet
% \textbf{P}rotocol (VOIP). The \textbf{\textit{VOIP area}} is located in
% the left hand corner of your screen.}

% \begin{figure}
% \centering
% \includegraphics[width=3.6819in,height=2.7346in]{userguid-img105.jpg}
% \end{figure}
% {\sffamily
% This chapter describes the following:}

% \bigskip

% \liststyleLxvi
% \begin{itemize}
% \item {\sffamily\bfseries\color[rgb]{0.0,0.0,0.5019608}
% Turning on/off Voice Chat}
% \item {\sffamily\bfseries\color[rgb]{0.0,0.0,0.5019608}
% Making Quick Phone Calls}
% \item {\sffamily\bfseries\color[rgb]{0.0,0.0,0.5019608}
% Making Virtual Cell Phone Calls}
% \item {\sffamily\bfseries\color[rgb]{0.0,0.0,0.5019608}
% Making Room Based Public Conference Calls}
% \item {\sffamily\bfseries\color[rgb]{0.0,0.0,0.5019608}
% Hiding the Virtual Cell Phone or Conference Call Function}
% \item {\sffamily\bfseries\color[rgb]{0.0,0.0,0.5019608}
% Turning Off Phone Calls}
% \end{itemize}

% \bigskip

% {\sffamily\bfseries
% \textit{\textcolor{black}{NOTE:}}\textmd{\textit{\textcolor[rgb]{0.0,0.0,0.5019608}{
% }}}}

% {\sffamily\bfseries
% \textit{In order for Voice Chat to work, you will need to have speakers
% attached to your computer, as well as a microphone connected to your
% computer. \newline
% \newline
% }\textmd{If you do not have these items connected to your computer, the
% VOIP function will remain }\textit{OFF}}

% \bigskip

% {\sffamily\color{black}
% CVE is equipped with the following capabilities:}

% \liststyleLxvii
% \begin{itemize}
% \item {\sffamily\color{black}
% A \textbf{\textit{Turn Off Function}}}
% \item {\sffamily\color{black}
% A \textbf{\textit{Quick Phone Call function}} to quickly connect
% privately with other online users}
% \item {\sffamily\color{black}
% A \textbf{\textit{Public Conference Call Function}} to talk to whoever
% is present in the same room \ as you}
% \item {\sffamily\color{black}
% A \textbf{\textit{Virtual Cell Phone Function}} with the ability to have
% \textit{private} and \textit{public }calls by letting you select the
% users that you want to talk to regardless of their location}
% \end{itemize}
% 
% \bigskip

% {\sffamily\color{black}
% It is important to learn how to make use of the Voice Chat option, since
% you can have direct conversations with other users, or get to know
% these users through these conversations. In addition, it is possible to
% have voice chat while engaged with other tasks, such as programming.
% Since CVE has the capability to share a work-screen, other users could
% offer you suggestions through voice chat. }

% \bigskip

% {\sffamily\color{black}
% Below is a blowup of the VOIP area with each of these areas and their
% functions indicated.}

% \liststyleLxvii

% \begin{figure}
% \centering
% \includegraphics[width=6.2173in,height=5.1929in]{userguid-img106.jpg}
% \end{figure}
% \end{itemize}

% \bigskip

% \subsubsection{Turning on/off Voice Chat}

% \bigskip

% {\sffamily\bfseries\color{black}
% \textit{NOTE}: }

% {\sffamily\color{black}
% \textbf{\textit{By default, Voice Chat is Turned OFF and you cannot talk
% or receive calls.}} \newline
% \newline
% This means that it is assumed that you will turn it on as soon as you
% want to engage in Voice Chat. Hence, to use any Voice Chat function,
% your Voice Chat will need to be turned \textbf{\textit{ON}} first.}

% \bigskip

% {\sffamily\bfseries\itshape\color[rgb]{0.0,0.2784314,1.0}
% 1. To turn on voice chat, click the Turn On/Off Button. To turn off
% Voice Chat, click the Turn On/Off button again. }

% \begin{figure}
% \centering
% \includegraphics[width=2.7453in,height=2.4984in]{userguid-img107.jpg}
% \end{figure}

% \begin{figure}
% \centering
% \includegraphics[width=2.8335in,height=2.5417in]{userguid-img108.jpg}
% \end{figure}
% {\sffamily\color{black}
% If Voice Chat is turned off, the speaker will be crossed out to indicate
% that Voice Chat is off; if Voice Chat is on, the speaker will not be
% crossed out.}

% \bigskip

% {\sffamily\bfseries\color{black}
% \textit{NOTE:} }

% {\sffamily\color{black}
% All of the steps below require that you have \ Voice Chat turned on. }

% \begin{figure}
% \centering
% \includegraphics[width=0.528in,height=0.4583in]{userguid-img109.jpg}
% \end{figure}
% \subsubsection{Making Quick Phone Calls}

% \bigskip

% {\sffamily\color{black}
% Quick phone calls are calls to a single user. In order to make a Quick
% Phone call, simply go the \textbf{\textit{VOIP Area}} in the upper left
% hand corner. Make sure to have \textbf{Voice Chat Turned On}. }

% \bigskip

% \begin{figure}
% \centering
% \includegraphics[width=4.4126in,height=2.711in]{userguid-img110.jpg}
% \end{figure}
% {\sffamily\bfseries\itshape\color[rgb]{0.0,0.2784314,1.0}
% 1. In the Quick Phone Call Function area, select the user that you want
% to talk by selecting a person in the dropbox}

% \bigskip

% {\sffamily\bfseries\itshape\color[rgb]{0.0,0.2784314,1.0}
% 2. Click {\textquotedblleft}Talk{\textquotedblright} on the right hand
% of the dropbox to talk to the user that you have selected }

% \begin{figure}
% \centering
% \includegraphics[width=6.1957in,height=4.2783in]{userguid-img111.jpg}
% \end{figure}

% \bigskip

% {\sffamily\color{black}
% A message will be sent to the other using saying:
% \texttt{\textbf{[username] is calling you!}}}

% \bigskip

% {\sffamily\bfseries\itshape\color[rgb]{0.0,0.2784314,1.0}
% 3. Wait for the other user to accept the phone call by turning on their
% virtual cell phone}

% \bigskip

% {\sffamily\color{black}
% That{\textquotesingle}s it! You can now talk to the user. At any time,
% if you want to end the Quick Phone call, click on the
% \textbf{\textit{End Button}}}

% \bigskip

% \subsubsection[Turning Off \ Quick Phone Calls]{Turning Off \ Quick
% Phone Calls}
% {\sffamily\bfseries\itshape\color[rgb]{0.0,0.2784314,1.0}
% 1. In order to turn off any existing Quick phone calls, simply click the
% {\textquotedblleft}End{\textquotedblright} button located on the left
% of the dropbox.}

% \begin{figure}
% \centering
% \includegraphics[width=0.6807in,height=0.5957in]{userguid-img112.jpg}
% \end{figure}
% {\sffamily
% Clicking the \textbf{{\textquotedblleft}End}{\textquotedblright} button
% \ will end any Quick Phone conversation that you are having.}

% \subsubsection{Making Virtual Cell Phone Calls}

% \bigskip

% {\sffamily\color{black}
% Next to making Quick Phone calls, CVE can also handle private single
% user-to-single user, as well as single user-to-multiple user virtual
% cell phone calls. \ As in real life, when you answer your cell phone,
% the virtual cell phone lets you select who you want to talk to as well
% as who you want to put on hold.}

% \bigskip

% {\sffamily\color{black}
% The virtual cell phone can do the following functions \ in managing your
% calls:}

% \liststyleLxviii
% \begin{itemize}
% \item {\sffamily\color{black}
% It displays who you are currently calling (displayed in the
% \textbf{Outgoing Calls}),}
% \item {\sffamily\color{black}
% It displays who is currently trying to reach you (\textbf{Incoming
% Calls})}
% \item {\sffamily\color{black}
% It displays which people you are currently talking to (\textbf{Online})}
% \item {\sffamily\color{black}
% It lets you \textit{hold}\textbf{ }current conversations when answering
% incoming calls from other users (\textbf{On Hold})}
% \item {\sffamily\color{black}
% It lets you \textit{hide} the virtual cell phone organizer screen so
% that you can carry on with your tasks while continuing Voice Chat
% (\textbf{Hide})}
% \end{itemize}
% 
% \bigskip
% 

% {\sffamily\bfseries\color{black}
% \textit{NOTE}: }

% {\sffamily\bfseries\color{black}
% \textit{When other users are available to talk to, the following message
% will appear in your Chat Area/Message Window: }\newline
% \newline
% {\textquotedblleft}\texttt{[username]: My Virtual Cell Phone is
% ON}{\textquotedblright}\newline
% }

% {\sffamily\color{black}
% This message indicates that this particular user is available to receive
% calls from you and other users.}

% {\sffamily\color{black}
% Below is an overview of the Virtual Cell Phone functions:}

% \begin{figure}
% \centering
% \includegraphics[width=6.8189in,height=5.5398in]{userguid-img113.jpg}
% \end{figure}

% \bigskip

% \clearpage

% {\sffamily\color{black}
% The following Virtual Cell Phone options will be described below:}

% \bigskip

% \liststyleLxix
% \begin{itemize}
% \item {\sffamily\bfseries\color{black}
% Making a virtual cell phone call to a single user}
% \item {\sffamily\bfseries\color{black}
% Making a virtual cell phone call to multiple users}
% \item {\sffamily\bfseries\color{black}
% Answering an incoming call / Putting a current Voice Chat on hold/
% Disconnecting calls}
% \item {\sffamily\bfseries\color{black}
% Hiding the Virtual Cell Phone Organizer Screen}
% \item {\sffamily\bfseries\color{black}
% Turning Off the Virtual Cell Phone }
% \end{itemize}
% {\centering\bfseries\itshape
% Making a virtual cell phone call to a single user
% \par}

% \bigskip

% {\sffamily\bfseries\color{black}
% \textit{NOTE}:\textmd{ }}

% {\sffamily\color{black}
% \textbf{In order to make a}\textbf{\textit{ Virtual Cell Phone
% Call}}\textbf{, both the sender and receiver of the virtual cell-phone
% call will need to have their virtual cell-phones Turned
% }\textbf{\textit{ON}}\textbf{. }\newline
% \newline
% A message will appear in each user{\textquotesingle}s Chat Area/Message
% Window indicating that a specific user{\textquotesingle}s Virtual Cell
% Phone is Turned On. Whenever a call is made, the receiver will be
% notified that the sender is trying to reach them, at which point the
% receiver can answers the call by turning on their virtual cell phone
% and accepting the call, or declining the call by not answering.}
% 
% 
% \bigskip
% 
% {\sffamily\bfseries\itshape\color[rgb]{0.0,0.2784314,1.0}
% 1. In order to make a single user phone call, turn on your virtual cell
% phone by clicking on the virtual cell phone icon. The Virtual Cell
% Phone Organizer Screen should now appear in the 3d View Area.}

% \liststyleLxx
% \begin{enumerate}
% \item[] 
% \begin{figure}
% \centering
% \includegraphics[width=6.9252in,height=2.9016in]{userguid-img114.jpg}
% \end{figure}
% \end{enumerate}

% \bigskip

% {\sffamily\color{black}
% A message will appear in both the virtual cell phone
% user{\textquotesingle}s screen and that of other users that this
% person{\textquotesingle}s virtual cell phone is turned on, indicating
% that this person can now receive and make virtual cell phone calls.}

% \bigskip

% {\sffamily\bfseries\itshape\color[rgb]{0.0,0.2784314,1.0}
% 2. Simply select the person{\textquotesingle}s name that you want to
% call in the User area, and click the
% {\textquotedblleft}Call{\textquotedblright} button. \ }

% {\sffamily
% {\textquotedblleft}\texttt{\textbf{I am calling
% [person{\textquotesingle}s name]}}{\textquotedblright} will appear in
% the sender{\textquotesingle}s Chat Area Message Window. }

% {\sffamily
% If the receiver{\textquotesingle}s Virtual Cell Phone is OFF, the sender
% will get an automatic message from the receiver saying
% \ \ {\textquotedblleft}\textbf{[username]:} \texttt{\textbf{sorry, the
% Cell Phone is OFF!}}{\textquotedblright}. }

% \bigskip

% {\sffamily\bfseries \textit{NOTE}: }

% {\sffamily Getting a message from a receiver saying }

% {\sffamily\bfseries
% {\textquotedblleft}\texttt{sorry, the Cell Phone is
% OFF!}{\textquotedblright}}

% {\sffamily can mean two things:}

% {\sffamily
% \textbf{The receiver needs to }\textbf{\textit{Turn On}}\textbf{ \ their
% Virtual Cell Phone} in which case they will answer your call }

% {\sffamily\bfseries
% The receiver is currently u\textit{navailable} to talk to, \textmd{in
% which case they will not answer your call}}

% \bigskip

% {\sffamily\bfseries\itshape\color[rgb]{0.0,0.2784314,1.0}
% 3. If your call is accepted, the following message will appear in
% sender{\textquotesingle}s Message Window:
% {\textquotedblleft}\texttt{[username] has accepted your
% call}{\textquotedblright} . The user will be placed in the
% {\textquotedblleft}Online{\textquotedblright} part of the Virtual Cell
% Phone Organizer screen.}

% \bigskip

% {\sffamily\bfseries
% \textit{NOTE}: \newline
% \textmd{To end a virtual cell phone conversation with any user, select
% their name and click
% {\textquotedblleft}}\textit{Disconnect}\textmd{{\textquotedblright}. }}

% {\sffamily
% To end all conversations by turning off the virtual cell phone, simply
% click {\textquotedblleft}\textbf{\textit{Off}}{\textquotedblright}. }

% {\sffamily
% To Hide the Virtual Cell Phone Organizer screen click
% {\textquotedblleft}\textbf{\textit{Hide}}{\textquotedblright}}

% \bigskip

% {\centering\bfseries\itshape
% Making a virtual cell phone call to multiple users \par}

% {\sffamily
% Making a virtual cell phone call to multiple users is simple, and
% involves the same steps as calling a single user. Instead of selecting
% one person, you select each person and press
% \ {\textquotedblleft}Call{\textquotedblright}.}

% {\sffamily\bfseries\itshape\color[rgb]{0.0,0.2784314,1.0}
% 1. In order to make a multiple user phone call, make sure you have your
% virtual cell phone turned on by clicking on the virtual cell phone
% icon. The Virtual Cell Phone Organizer Screen should now appear.}

% \bigskip

% {\sffamily\bfseries\itshape\color[rgb]{0.0,0.2784314,1.0}
% 2. In the Virtual Cell Phone Organizer Screen, Simply select each user
% in the call screen, and press the
% {\textquotedblleft}Call{\textquotedblright} button after selecting each
% user. \ Each user will appear in the Outgoing calls box in your Virtual
% Cell Phone Organizer Screen to indicate that you are calling them. }

% {\centering\bfseries\itshape
% Answering an incoming call \par}

% \bigskip

% {\sffamily
% When someone tries to call you, a message will appear in your
% \textbf{\textit{Chat Area /Message Windows}} that this person is
% calling you.}

% {\sffamily To answer an incoming call, do the following:}

% {\sffamily\bfseries\itshape\color[rgb]{0.0,0.2784314,1.0}
% 1. Turn on your Virtual Cell Phone (if it{\textquotesingle}s not on
% already) and go to the Virtual Cell Phone Organizer Screen. \newline
% \newline
% }

% \begin{figure}
% \centering
% \includegraphics[width=6.6102in,height=4.5839in]{userguid-img115.jpg}
% \end{figure}

% \bigskip

% {\sffamily\bfseries\itshape\color[rgb]{0.0,0.2784314,1.0}
% 2. In the Incoming Calls box select the person{\textquotesingle}s name
% by left mouse clicking their name, and press \ the
% {\textgreater}{\textgreater} button to talk to the person.\newline
% }

% {\sffamily\bfseries\color{black} \textit{NOTE}: }

% {\sffamily\bfseries
% \textmd{To end a virtual cell phone conversation with any user, select
% their name and click
% {\textquotedblleft}}\textit{Disconnect}\textmd{{\textquotedblright}. }}

% {\sffamily
% To end all conversations by turning off the virtual cell phone, simply
% click {\textquotedblleft}\textbf{\textit{Off}}{\textquotedblright}. }

% {\sffamily\color{black}
% To Hide the Virtual Cell Phone Organizer screen click
% {\textquotedblleft}\textbf{\textit{Hide}}{\textquotedblright}}

% \bigskip

% {\centering\bfseries\itshape
% Putting a current Voice Chat on hold \par}

% {\sffamily
% The Virtual Cell Phone also makes it possible for you to put someone on
% \textbf{\textit{{\textquotedblleft}Hold{\textquotedblright}}} . This
% means that you can then talk to another person while the original
% person waits for you to resume your Voice Chat conversation with them.}

% \bigskip

% {\sffamily To hold someone on Voice Chat, simply do the following:}

% \bigskip

% {\sffamily\bfseries\itshape\color[rgb]{0.0,0.2784314,1.0}
% 1. In the Online box with current users that you are Voice Chatting
% with, select the user that you want to put on hold by left
% mouse-clicking on their name}

% \bigskip

% {\sffamily\bfseries\itshape\color[rgb]{0.0,0.2784314,1.0}
% 2. On the right side of the Online box, press the
% {\textgreater}{\textgreater} \ button to put the user on Hold}

% \bigskip

% {\sffamily This will move the user into the Hold box, indicating that
%  they are on Hold}

% \bigskip

% {\sffamily\bfseries\itshape\color[rgb]{0.0,0.2784314,1.0}
% 3. To unHold a user, press the {\textless}{\textless} \ button so they
% will be moved back into the Online chapter }

% {\centering\bfseries\itshape
% %\newline
% Hiding the Virtual Cell Phone Organizer Screen
% \par}

% {\sffamily\bfseries \textit{NOTE}: }

% {\sffamily\bfseries
% At any time, it is possible to
% {\textquotedblleft}\textit{Hide}{\textquotedblright} the virtual cell
% phone organizer screen so that you can continue performing other tasks.
% }

% {\sffamily
% For instance, while talking to somebody, you want to move your avatar.
% Or, perhaps you \ want to edit a file while engaged in a Voice Chat. }

% {\sffamily In this case it is easy to use
% {\textquotedblleft}\textbf{\textit{Hide}}\textit{{\textquotedblleft}}\textbf{\textit{
% }}\textit{to Hide the Virtual Cell Phone Organizer Screen}\textbf{ }so
% that you can return to the 3d Navigation area. }

% {\sffamily To hide the Virtual Cell Phone Organizer Screen, Simply do the
% following:}

% \bigskip

% {\sffamily\bfseries\itshape\color[rgb]{0.0,0.2784314,1.0}
% 1. Click the {\textquotedblleft}Hide{\textquotedblright} button in the
% Virtual Cell Phone Organizer Screen to hide the Virtual Cell Phone
% Organizer Screen.}

% \begin{figure}
% \centering
% \includegraphics[width=6.7846in,height=4.65in]{userguid-img116.jpg}
% \end{figure}
% {\sffamily\bfseries
% NOTE: \textmd{To return to the }Virtual Cell Phone Organizer
% Screen\textmd{, simply press the Virtual Cell Phone again.}}

% {\centering\bfseries\itshape
% Turning Off the Virtual Cell Phone/ \newline
% Ending All Voice Chat Converations 
% \par}

% \bigskip

% {\sffamily
% At any time, it is possible to turn off the Virtual Cell Phone. If you
% turn off the Virtual Cell Phone, you are ending \textbf{all }Voice Chat
% conversations. \ You can do this by doing the following:}

% \bigskip

% {\sffamily\bfseries\itshape\color[rgb]{0.0,0.2784314,1.0}
% To turn of the Virtual Cell Phone and end all conversations, click the
% {\textquotedblleft}Off{\textquotedblright} button on the bottom of the
% Virtual Cell Phone Organizer Screen. }

% \bigskip

% \subsubsection{Making Room Based Public Conference Calls}
% {\sffamily\color{black}
% In CVE it is also possible to have \textbf{\textit{Room Based Public
% Conference calls}}. A Room Based Public Conference (or Room Based) call
% is different from the Virtual Cell Phone because it enables you have a
% Voice Chat with whoever is present in the current room or joins that
% room. The Room Based call is explicitly public, meaning that anybody
% can join a Voice Chat provided their avatar is in the specific room in
% which the Room Based call is being held. }

% \bigskip

% {\sffamily\bfseries\color{black}
% \textit{NOTE}: The \textit{Room Based Public Conference Call function}
% lets you have a Voice Chat conversation with whoever is in the same
% room as your avatar. }

% \bigskip

% {\sffamily\color{black}
% In essence, the Room Based Call is a way to convene and \ meet with
% other users in your vicinity. It is also a way to have a public meeting
% with various other users. For instance, you could all agree \ to meet
% at a particular room at a particular time, and by enabling the Room
% Based Public Conference Call function, conduct a meeting.}
% 
% {\sffamily\bfseries\itshape\color[rgb]{0.0,0.2784314,1.0}
% To enable a Room Based Public Conference Call, click on the Public
% Conference Call Icon as depicted below.}

% \begin{figure}
% \centering
% \includegraphics[width=5.5756in,height=3.7937in]{userguid-img117.jpg}
% \end{figure}

% \bigskip

% {\sffamily
% The Voice Tab will appear, allowing you to see those currently using the
% Room Based Call option (\textbf{\textit{Active Members}}) as well as
% all users currently present in the room (\textbf{\textit{All Room
% Members}})}

% \liststyleLxxii
% \begin{itemize}
% \item {\sffamily\color{black}
% To see who you are currently speaking to, see the \textbf{\textit{Active
% members}} box. }
% \item {\sffamily\color{black}
% To see \ users who are currently present in the room, but who might or
% might not use the Room Based Call, see the \textbf{\textit{All Room
% Members}} box.}
% \end{itemize}
% {\centering\bfseries\itshape
% Hiding the Room Based Call Screen
% \par}

% \bigskip

% {\sffamily
% At any time, it is also possible to
% {\textquotedblleft}\textbf{Hide{\textquotedblright}} the
% \textbf{\textit{\textcolor{black}{Room Based Public Conference
% call}}}\textbf{\textit{ screen}} so that you can continue performing
% other tasks. For instance, while talking to somebody, you want to move
% your avatar or work on a file. In this case it is easy to \textit{Hide}
% the \textcolor{black}{Room Based Public Conference call Screen} so that
% you can return to the 3d Navigation area.}

% {\sffamily
% To hide the Virtual Cell Phone Organizer Screen, Simply do the
% following:}

% {\sffamily\bfseries\itshape\color[rgb]{0.0,0.2784314,1.0}
% Click the {\textquotedblleft}Hide{\textquotedblright} button in the
% Virtual Cell Phone Organizer Screen to hide the Room Based Call
% screen.}

% \begin{figure}
% \centering
% \includegraphics[width=6.5547in,height=4.2417in]{userguid-img118.jpg}
% \end{figure}

% \bigskip

% \subsubsection[Quicklist Voice Chat]{Quicklist Voice Chat}
% {\centering\bfseries\itshape
% Turn On/Off Voice Chat
% \par}

% \bigskip

% \begin{center}
% \tablehead{\hline
% \centering \sffamily\bfseries\itshape Action &
% \centering\arraybslash \sffamily\bfseries\itshape Action \\\hline}
% \begin{supertabular}{|m{3.07476in}|m{3.06986in}|}
% {\sffamily\bfseries Voice Chat is Turned ON:}

% ~

% \centering
% \includegraphics[width=0.528in,height=0.4583in]{userguid-img119.jpg}
%  &
% {\sffamily\bfseries Voice Chat is Turned OFF:}

% \includegraphics[width=0.5in,height=0.4445in]{userguid-img120.jpg}
% \\
% \hline
% \end{supertabular}
% \end{center}

% \bigskip

% {\centering\bfseries\itshape
% Making Quick Phone Calls
% \par}

% \begin{center}
% \tablehead{\hline
% \centering \sffamily\bfseries\itshape Action 1 &
% \centering \sffamily\bfseries\itshape Action 2 &
% \centering\arraybslash \sffamily\bfseries\itshape Action 3\\\hline}
% \begin{supertabular}{|m{1.9073598in}|m{2.22816in}|m{1.9205599in}|}
% \sffamily\bfseries Select Receiver of Call in Dropbox &
% {\sffamily\bfseries Click {\textquotedblleft}Talk{\textquotedblright} to
% Talk to that User}

% \centering
% \includegraphics[width=0.722in,height=0.4445in]{userguid-img121.jpg}
% \centering
% \includegraphics[width=1.5417in,height=0.528in]{userguid-img122.jpg}
%  &
% \sffamily\bfseries Wait for Receiver to Answer Your Call\\\hline
% \end{supertabular}
% \end{center}

% \bigskip

% {\centering\bfseries\itshape
% Ending Quick Phone Calls
% \par}

% \begin{center}
% \tablehead{\hline
% \centering\arraybslash \sffamily\bfseries\itshape Action\\\hline}
% \begin{supertabular}{|m{1.9212599in}|}
% {\sffamily\bfseries To End any Quick Phone Call Chats, Press
% {\textquotedblleft}End{\textquotedblright}}

% \includegraphics[width=0.722in,height=0.4583in]{userguid-img123.jpg}
% \\
% \hline
% \end{supertabular}
% \end{center}
% {\centering\bfseries\itshape
% Making Virtual Cell Phone Calls to single and multiple users\newline

% \par}

% \begin{center}
% \tablehead{\hline
% \centering \sffamily\bfseries\itshape Action 1 &
% \centering \sffamily\bfseries\itshape Action 2 &
% \centering \sffamily\bfseries\itshape Action 3 &
% \centering\arraybslash \sffamily\bfseries\itshape Action 4\\\hline}
% \begin{supertabular}{|m{1.3247598in}|m{1.6441599in}|m{1.6580598in}|m{1.3323599in}|}
% {\sffamily\bfseries Turn on Virtual Cell Phone}

% ~



% \centering
% \includegraphics[width=0.5555in,height=0.4583in]{userguid-img124.jpg}
% ~
%  &
% {\sffamily\bfseries Select Receiver of Call in Virtual Cell Phone
% Organizer Screen}

% ~

% \sffamily\bfseries [Select Receiver] &
% {\sffamily\bfseries Click {\textquotedblleft}Call{\textquotedblright} to
% talk to user}

% \centering
% \includegraphics[width=0.8472in,height=0.5417in]{userguid-img125.jpg}
%  &
% \sffamily\bfseries Wait for Receiver to Answer Your Call\\\hline
% \end{supertabular}
% \end{center}

% \bigskip

% {\centering\bfseries\itshape
% Ending All Calls/Turning OFF Virtual Cell Phone\newline

% \par}

% \begin{center}
% \tablehead{\hline
% \centering\arraybslash \sffamily\bfseries\itshape Action\\\hline}
% \begin{supertabular}{|m{1.9177599in}|}
% {\sffamily\bfseries Click {\textquotedblleft}OFF{\textquotedblright} to
% end ALL Voice Chats}

% \includegraphics[width=1.639in,height=0.6665in]{userguid-img126.jpg}
% \\
% \hline
% \end{supertabular}
% \end{center}

% \bigskip

% {\centering\bfseries\itshape
% Hiding the Virtual Cell Phone\newline

% \par}

% \begin{center}
% \tablehead{\hline
% \centering \sffamily\bfseries\itshape Action 1 &
% \centering\arraybslash \sffamily\bfseries\itshape Action 2\\\hline}
% \begin{supertabular}{|m{1.7143599in}|m{2.0865598in}|}
% {\sffamily\bfseries To hide the Virtual Cell Phone Organizer Screen
% \ click {\textquotedblleft}Hide{\textquotedblright}}
% 
% \centering
% \includegraphics[width=1.611in,height=0.5693in]{userguid-img127.jpg}
%  &
% \sffamily\bfseries Continue with your other tasks\\\hline
% \end{supertabular}
% \end{center}

% \bigskip

% {\centering\bfseries\itshape
% Making Room Based Calls\newline

% \par}

% \begin{center}
% \tablehead{\hline
% \centering \sffamily\bfseries\itshape Action 1 &
% \centering\arraybslash \sffamily\bfseries\itshape Action 2\\\hline}
% \begin{supertabular}{|m{1.7427598in}|m{2.0962598in}|}
% {\sffamily\bfseries Click on the Room Based Public Conference Function}

% \centering
% \includegraphics[width=0.8492in,height=0.7543in]{userguid-img128.jpg}
%  &
% {\sffamily\bfseries Voice Chat with Current People in the Room who also
% turned on the Room Based Function and are listed under
% {\textquotedblleft}Active Members{\textquotedblright}}

% ~
% \\\hline
% \end{supertabular}
% \end{center}
% 
% \bigskip

\clearpage

\subsection{Chapter 8: Using the File Menu Bar to Create
projects}
\subsubsection{Overview of Chapter}
{\sffamily
The \textbf{\textit{File Menu Bar}} is a key component of working in
CVE. Generally, it lets you create and modify project files, run and
compile these files, as well as change your avatar and account options,
as well as the way various items are displayed in the CVE interface.
The File Menu Bar also lets you create a variety of diffent programs
using C/C++/Java and Unicon, which you can then share and work on with
others using the \textbf{\textit{CVE Integrated Development
Environment}} (for more on this see the next chapter ).}


\bigskip

\liststyleLxxiii
\begin{itemize}
\item[] {\sffamily\color{black}
\textbf{\textit{NOTE:}} The File Menubar lets you do the following
things:}
\item {\sffamily\bfseries\color{black}
Create/Open/Close/Save Files and Projects}
\item {\sffamily\bfseries\color{black}
Exit CVE Configure the size of various elements of the screen}
\item {\sffamily\bfseries\color{black}
Create, Run, Compile Java/C/C++ and Unicon Files}
\item {\sffamily\bfseries\color{black}
Add pre-created Unicon code into your files/projects}
\item {\sffamily\bfseries\color{black}
Connect/Disconnect to the Server}
\item {\sffamily\bfseries\color{black}
Configure Account/Avatar Options}
\item {\sffamily\bfseries\color{black}
Consult Help Guides and Guidelines}
\end{itemize}

\bigskip

\liststyleLxxiv
\begin{itemize}
\item[] 
\begin{figure}
\centering
\includegraphics[width=5.9299in,height=4.2244in]{userguid-img129.jpg}
\end{figure}
\end{itemize}
\liststyleLxxv
\begin{enumerate}
\item[] 
\bigskip
\end{enumerate}
{\sffamily\bfseries\color{black}
For a quick idea of all of the functions see the following
overview:\newline
}

{\sffamily
\textbf{\textcolor{black}{FILE:}}\textcolor{black}{
}\textcolor{black}{Open/ Close/ Save Files / Print/ Save Chat
Transcripts / Take Screenshot/ Exit}}

{\sffamily
\textbf{\textcolor{black}{VIEW: }}Change the size of various screen
windows}

{\sffamily
\textbf{\textcolor{black}{EDIT:}}\textbf{\textcolor{black}{ }}Cut/ Copy/
Paste/ Select All/ Find and Replace/ Go to Line}

{\sffamily
\textbf{\textcolor{black}{INSERT:}} Insert precreated Procedure/ Class/
Method Unicon Code }

{\sffamily
\textbf{\textcolor{black}{COMPILE: }}Make Executable/ Compile
Only/Compiler Options}

{\sffamily
\textbf{\textcolor{black}{RUN: }}Run Program }

{\sffamily
\textbf{\textcolor{black}{PROJECT: }}Create A New/ Open an existing
C/C++/Java/Unicon Project/ Compile a Project / Make Clean / Make Clean
All}

{\sffamily
\textbf{\textcolor{black}{NETWORK: }}Connect/Disconnect From the Server}

{\sffamily
\textbf{\textcolor{black}{ACCOUNT: }}Change
Password/Account/Avatar/Face}

{\sffamily
\textbf{\textcolor{black}{HELP: }}User Guide /Key Commands / About
CVE Version\newline
}

{\sffamily
\textbf{\textit{NOTE:}} Some of the items of the File Menu Bar are also
contained in the IDE toolbar as icons. Next to accessing these
functions in the \textbf{\textit{File Menu Bar}}, some of these
functions can also be entered using the \textbf{\textit{IDE toolbar
}}\textit{(the toolbar that appears when you open a new file)}. Hence,
below these functions are listed using \ the File Menu Bar route. An
extra icon will be given for each function if this function is
\textbf{\textit{also}}\textbf{ }accessible in the IDE toobar. }


\subsubsection{Creating your program files}
{\sffamily
In CVE it is possible to create, save and open your program files. Since
CVE lets you create C/C++, Java and Unicon programs, you can create
files and applications for these programming languages. These steps can
be done by accessing the \textbf{\textit{File Menu Bar.}} }

{\centering\bfseries\itshape
Creating A File
\par}

\begin{center}
\tablehead{\hline
{\sffamily\itshape Action} &
{\sffamily\itshape File Menu Bar Item} &
{\sffamily\itshape IDE icon}\\\hline}
\begin{supertabular}{|m{3.38376in}|m{1.6525599in}|m{1.6525599in}|}
{\sffamily\itshape Create a new file} &
{\sffamily\bfseries\itshape\color[rgb]{0.0,0.2784314,1.0} \ File
{\textgreater} New} \  &
%\centering
\includegraphics[width=0.4307in,height=0.3335in]{userguid-img130.jpg}
\\\hline
\end{supertabular}
\end{center}

\bigskip


\bigskip

{\sffamily\bfseries\itshape\color[rgb]{0.0,0.2784314,1.0}
1. In order to create a new file, access the File Menu Bar Item
{\textquotedblleft}File{\textgreater}New{\textquotedblright}}

{\sffamily
A new Tab \textbf{\textit{{\textquotedblleft}Untitled
1{\textquotedblright}}} will be opened in the \textbf{\textit{3d View
Area}} to indicate that you are creating a new file. \ Contained in
this Tab is the IDE workscreen which lets you program your file.}

\begin{figure}
\centering
\includegraphics[width=5.5028in,height=4.5154in]{userguid-img131.jpg}
\end{figure}

\bigskip

{\sffamily\bfseries\itshape\color[rgb]{0.0,0.2784314,1.0}
2. The IDE workscreen will appear, and you can start writing your
program}


\bigskip

{\sffamily\bfseries\itshape\color[rgb]{0.0,0.2784314,1.0}
3. Save your file by clicking on {\textquotedblleft}File {\textgreater}
Save{\textquotedblright} or {\textquotedblleft}File {\textgreater} Save
As{\textquotedblright} and naming your file}


\bigskip

{\sffamily
In order to save files in either C/C++, Java or Unicon format, you will
need to use the extension associated with that programming language.
Below is an overview of these extensions:}

\begin{center}
\tablehead{}
\begin{supertabular}{|m{3.38376in}|m{3.38376in}|}
\hline
\sffamily Unicon Files &
\ttfamily [Filename].icn\\\hline
\sffamily C\newline
C++ &
\ttfamily [Filename].c\newline
[Filename].cpp\\\hline
\sffamily Java &
\ttfamily [Filename].java\\\hline
\end{supertabular}
\end{center}

\bigskip

{\sffamily
It is also possible to open a new file when working in the IDE by
pressing the icon for {\textquotedblleft}\textbf{New
file}{\textquotedblright} in the \textbf{IDE toolbar}. }

{\centering\bfseries\itshape
Opening a File 
\par}


\bigskip

\begin{center}
\tablehead{\hline
{\sffamily\itshape Action} &
{\sffamily\itshape File Menu Bar Item} &
{\sffamily\itshape IDE icon}\\\hline}
\begin{supertabular}{|m{3.38376in}|m{1.6525599in}|m{1.6525599in}|}
{\sffamily\itshape Open an existing file:}  &
{\sffamily\bfseries\itshape\color[rgb]{0.0,0.2784314,1.0} File
{\textgreater} Open} &
%\centering
\includegraphics[width=0.3752in,height=0.3335in]{userguid-img132.jpg}
\\\hline
\end{supertabular}
\end{center}

\bigskip

{\sffamily
In order to open a file in a directory that you have created, you can
use the {\textquotedblleft}\textbf{File {\textgreater}
Open}{\textquotedblright} or IDE icon.}


\bigskip

{\sffamily\bfseries\itshape\color[rgb]{0.0,0.2784314,1.0}
1. To open a file, click on {\textquotedblleft}File {\textgreater}
Open{\textquotedblright}, select the file, and click
{\textquotedblleft}Open{\textquotedblright} to open the file selected.}


\bigskip

{\centering\bfseries\itshape
Saving a File
\par}

{\sffamily
It is important to save your files often to avoid losing work in case of
a crash. It is possible to save your work by simply clicking on
{\textquotedblleft}File {\textgreater} Save{\textquotedblright} or the
IDE icon for save. If you are saving a new file, you will be asked to
specificy the name and the directory under which you want to save the
file. If saving an existing file, the file will be saved under its
original name.}


\bigskip

{\sffamily\bfseries
Save:}

\begin{center}
\tablehead{\hline
{\sffamily\itshape Action} &
{\sffamily\itshape File Menu Bar Item} &
{\sffamily\itshape IDE Icon}\\\hline}
\begin{supertabular}{|m{3.38376in}|m{1.6525599in}|m{1.6525599in}|}
{\sffamily\itshape Save a file} &
{\sffamily\bfseries\itshape\color[rgb]{0.0,0.2784314,1.0} \ File
{\textgreater} Save}  &
%\centering
\includegraphics[width=0.4165in,height=0.3055in]{userguid-img133.jpg}
\\\hline
\end{supertabular}
\end{center}
{\sffamily\bfseries\itshape\color[rgb]{0.0,0.2784314,1.0}
%\newline
To save your file, click {\textquotedblleft}File {\textgreater}
Save{\textquotedblright} in the File Menu Bar or click the IDE icon}

{\sffamily
It is also possible to use {\textquotedblleft}Save
As{\textquotedblright} to save your file under a specific name in a
specific directory. }

{\sffamily\bfseries
Save As:}

\begin{center}
\tablehead{\hline
{\sffamily\itshape Action} &
{\sffamily\itshape File Menu Bar Item} &
{\sffamily\itshape IDE Icon}\\\hline}
\begin{supertabular}{|m{3.38376in}|m{1.6525599in}|m{1.6525599in}|}
{\sffamily\itshape Save a new file} &
{\sffamily\bfseries\itshape\color[rgb]{0.0,0.2784314,1.0} File
{\textgreater} Save As (specify location and name)} &
%\centering
\includegraphics[width=0.389in,height=0.3335in]{userguid-img134.jpg}
\\\hline
\end{supertabular}
\end{center}

\bigskip

{\sffamily\bfseries\itshape\color[rgb]{0.0,0.2784314,1.0}
To specify the name and location of your file, click
{\textquotedblleft}File {\textgreater} Save As{\textquotedblright} or
click the IDE Icon}


\bigskip

{\centering\bfseries\itshape
Exiting a File
\par}

{\sffamily
At any time, it is possible to close the File Tab and return to the 3d
Area or the Map Tab. It is a good idea to save your work before,
because this will not save your work automatically.}


\bigskip

{\sffamily\bfseries
Save and Close a File:}

\begin{center}
\tablehead{}
\begin{supertabular}{|m{3.38516in}|m{1.6573598in}|m{1.6483599in}|}
\hline
{\sffamily\itshape Action}\centering
\includegraphics[width=0.4165in,height=0.3055in]{userguid-img135.jpg}
 &
{\sffamily\itshape File Menu Bar Item} &
{\sffamily\itshape IDE Icon}\\\hline
{\sffamily\itshape Save and Close a file} &
{\sffamily\bfseries\itshape\color[rgb]{0.0,0.2784314,1.0} 1. File
{\textgreater} Save}

{\sffamily\bfseries\itshape\color[rgb]{0.0,0.2784314,1.0} 2.}

\centering
\includegraphics[width=0.4165in,height=0.3335in]{userguid-img136.jpg}
~
 &
{\sffamily\bfseries\itshape\color[rgb]{0.0,0.2784314,1.0} 1.}

{\sffamily\bfseries\itshape\color[rgb]{0.0,0.2784314,1.0} 2.}
\includegraphics[width=0.4165in,height=0.3335in]{userguid-img137.jpg}
\\\hline
\end{supertabular}
\end{center}
{\sffamily\bfseries\itshape\color[rgb]{0.0,0.2784314,1.0}
%\newline
To close a File Tab and save your file, first save your file, and then
press the IDE icon to close}

{\sffamily
or \newline
\textbf{To Exit without Saving a File:}}

\begin{center}
\tablehead{}
\begin{supertabular}{|m{3.38516in}|m{3.38446in}|}
\hline
\sffamily\itshape Action &
\sffamily\itshape IDE Icon\\\hline
\sffamily\itshape Exit a file without saving &
%\centering
\includegraphics[width=0.4165in,height=0.3335in]{userguid-img138.jpg}
\\\hline
\end{supertabular}
\end{center}

\bigskip

{\sffamily\bfseries\itshape\color[rgb]{0.0,0.2784314,1.0}
To Exit a file without saving it, simply press the IDE icon at any time}


\bigskip

\subsubsection{Exiting CVE}

\bigskip

{\sffamily
At any time, you can also exit CVE by clicking
\textbf{\textit{{\textquotedblleft}File {\textgreater}
Exit{\textquotedblright}}}. Make sure to save your files.}


\bigskip

\begin{center}
\tablehead{}
\begin{supertabular}{|m{3.38516in}|m{3.38446in}|}
\hline
\sffamily\itshape Action &
\sffamily\itshape File Menu Bar Item\\\hline
\sffamily\itshape Exit CVE &
{\sffamily\bfseries\itshape\color[rgb]{0.0,0.2784314,1.0} File
{\textgreater} exit}\\\hline
\end{supertabular}
\end{center}
{\sffamily\bfseries\itshape\color[rgb]{0.0,0.2784314,1.0}
%\newline
To Exit CVE at any time simply click {\textquotedblleft}File
{\textgreater} Exit{\textquotedblright} \ }


\bigskip

\subsubsection{Editing your file}
{\sffamily
CVE also comes equipped with regular file editing capabilities such as
\textbf{\textit{Cut}}, \textbf{\textit{Copy}}, and
\textbf{\textit{Paste }}that will allow you to delete, copy, and paste
selections of your code. In addition, CVE also has
\textbf{\textit{Select All}}, \textbf{\textit{Find}} and
\textbf{\textit{Replace}}, and \textbf{\textit{Go To Line}} functions
that allow you to select all portions of your program, find and replace
specific words and code, and the ability to jump to a particular line
in your code. All of these functions will help you edit your programs
with ease.}

{\centering\bfseries\itshape
Cutting, Copying and Pasting
\par}


\bigskip

\begin{center}
\tablehead{}
\begin{supertabular}{|m{3.38376in}|m{1.6525599in}|m{1.6525599in}|}
\hline
\sffamily\itshape Action &
\sffamily\itshape File Menu Bar Item &
\sffamily\itshape IDE Icon\\\hline
{\sffamily\itshape Cut\newline
\newline
Copy}

~

\sffamily\itshape Paste  &
{\sffamily\bfseries\itshape\color[rgb]{0.0,0.2784314,1.0} Edit
{\textgreater} Cut\newline
\newline
\newline
Edit {\textgreater} Copy}

~

{\sffamily\bfseries\itshape\color[rgb]{0.0,0.2784314,1.0} Edit
{\textgreater} Paste} &
{\centering
\includegraphics[width=0.4165in,height=0.3335in]{userguid-img139.jpg}
\ \\
\includegraphics[width=0.4028in,height=0.3472in]{userguid-img140.jpg}
\ \\
\includegraphics[width=0.4165in,height=0.3335in]{userguid-img141.jpg}
\ \\ }
\\\hline
\end{supertabular}
\end{center}
{\sffamily\bfseries\itshape
%\newline
To Cut:}

{\sffamily\bfseries\itshape\color[rgb]{0.0,0.2784314,1.0}
In order to Cut specific portions of your program, highlight (left mouse
click and drag over) the text you want to cut, and select
{\textquotedblleft}Edit {\textgreater} Cut{\textquotedblright} or the
IDE Icon}

{\sffamily\bfseries\itshape
%\newline
To Copy:}

{\sffamily\bfseries\itshape\color[rgb]{0.0,0.2784314,1.0}
In order to Copy specific portions of your program, highlight (left
mouse click and drag over) the text you want to copy, and select
{\textquotedblleft}Edit {\textgreater} Copy{\textquotedblright} or the
IDE Icon}


\bigskip

{\sffamily\bfseries\itshape
To Paste:}

{\sffamily\bfseries\itshape\color[rgb]{0.0,0.2784314,1.0}
Copy the portion of text you want to paste, and position your curse in
the place where you want to paste \ and select {\textquotedblleft}Edit
{\textgreater} Paste{\textquotedblright} or the IDE Icon for Paste.}


\bigskip


\bigskip

{\centering\bfseries\itshape
Select
\par}


\bigskip

{\sffamily
The \textbf{\textit{{\textquotedblleft}Edit {\textgreater} Select
All{\textquotedblright}}} function lets you select all of the text in
your File Tab. This is handy if you want to copy, cut, or paste all of
your code. To select \ parts of your code, left mouse click hold and
drag over your code until the code is highlighted.\newline
}

{\sffamily\bfseries
Select:}

\begin{center}
\tablehead{}
\begin{supertabular}{|m{3.38376in}|m{3.38376in}|}
\hline
\sffamily\itshape Action &
\sffamily\itshape User Action\\\hline
\sffamily\itshape Select  &
\sffamily Left Mouse Click (Hold) + Drag your code to select parts of
your code\\\hline
\end{supertabular}
\end{center}

\bigskip

{\sffamily\bfseries\itshape\color[rgb]{0.0,0.2784314,1.0}
To select a specific part of your code, Left Mouse Click and Hold + Drag
over your code to select that part\newline
}

{\sffamily
\textbf{Select ALL:}}

\begin{center}
\tablehead{}
\begin{supertabular}{|m{3.38376in}|m{3.38376in}|}
\hline
\sffamily\itshape Action &
\sffamily\itshape File Menu Bar Item\\\hline
\sffamily\itshape Select All  &
{\sffamily\bfseries\itshape\color[rgb]{0.0,0.2784314,1.0} Edit
{\textgreater} Select All}\\\hline
\end{supertabular}
\end{center}

\bigskip

{\sffamily\bfseries\itshape\color[rgb]{0.0,0.2784314,1.0}
\ To select ALL of your code, go to {\textquotedblleft}Edit
{\textgreater} Select All{\textquotedblright} in the File Menu Bar}

{\sffamily
{\textquotedblleft}\textbf{\textit{Select All}}{\textquotedblright} lets
you easily select all of your text in order to copy or cut it.}

{\centering\bfseries\itshape
Find and Replace
\par}

{\sffamily
%\newline
%\newline
The \textbf{Find} and \textbf{Replace} functions let you find pieces of
code quickly in order to alter or replace these. This is ideal for
finding strings in your programs.}

{\sffamily\bfseries
%\newline
%\newline
%\newline
%\newline
Find:}

\begin{center}
\tablehead{}
\begin{supertabular}{|m{3.38376in}|m{1.6525599in}|m{1.6525599in}|}
\hline
\sffamily\itshape Action &
\sffamily\itshape File Menu Bar Item &
\sffamily\itshape IDE Icon\\\hline
\sffamily\itshape Find &
{\sffamily\bfseries\itshape\color[rgb]{0.0,0.2784314,1.0} Edit
{\textgreater} Find} &
{\centering
\includegraphics[width=0.4583in,height=0.3752in]{userguid-img142.jpg}
\ \\}
\\\hline
\end{supertabular}
\end{center}

\bigskip

{\sffamily\bfseries\itshape\color[rgb]{0.0,0.2784314,1.0}
1. To Find a specific piece of code, in the File Menu {\textgreater} Bar
Select {\textquotedblleft}Edit {\textgreater} Find{\textquotedblright}
\ or click the IDE Icon.}



{\sffamily\bfseries\itshape\color[rgb]{0.0,0.2784314,1.0}

2. A new window will open asking you to {\textquotedblleft}Enter String
to Seek{\textquotedblright} \ }


\begin{figure}
\centering
\includegraphics[width=4.4in,height=1.6047in]{userguid-img143.jpg}
\end{figure}
{\sffamily\bfseries\itshape\color[rgb]{0.0,0.2784314,1.0}
3. Fill in your search item and select
{\textquotedblleft}Okay{\textquotedblright} to find this code in your
program}



\begin{figure}
\centering
\includegraphics[width=6.7846in,height=3.4102in]{userguid-img144.jpg}
\end{figure}
{\sffamily
The \textbf{Find} function is handy if you simply want to find a
specific piece of code or if you need to change that code.}


\bigskip

{\sffamily\bfseries
Replace: }

\begin{center}
\tablehead{}
\begin{supertabular}{|m{3.38376in}|m{1.6525599in}|m{1.6525599in}|}
\hline
\sffamily\itshape Action &
\sffamily\itshape File Menu Bar Item &
\sffamily\itshape IDE Icon\\\hline
\sffamily\itshape Replace &
{\sffamily\bfseries\itshape\color[rgb]{0.0,0.2784314,1.0} Edit
{\textgreater} Replace} &
~
\\\hline
\end{supertabular}
\end{center}

\bigskip

{\sffamily\bfseries\itshape\color[rgb]{0.0,0.2784314,1.0}
1. To replace code, select Edit {\textgreater} Replace}

{\sffamily\bfseries\itshape\color[rgb]{0.0,0.2784314,1.0}
2. A new window will open and ask you {\textquotedblleft}Replace what
with what:{\textquotedblright} and ask you to fill in your original
code and the new code}


\bigskip



\begin{figure}
\centering
\includegraphics[width=4.9043in,height=1.9335in]{userguid-img145.jpg}
\end{figure}
{\sffamily\bfseries\itshape\color[rgb]{0.0,0.2784314,1.0}
3. In the first entry box, enter the original code you want to replace}

{\sffamily\bfseries\itshape\color[rgb]{0.0,0.2784314,1.0}
4. In the second entry box, enter the new code}

{\sffamily\bfseries\itshape\color[rgb]{0.0,0.2784314,1.0}
5. Select {\textquotedblleft}Okay{\textquotedblright} to replace the
original code with the new code\newline
}

{\centering\bfseries\itshape
Go to Line
\par}

\begin{center}
\tablehead{}
\begin{supertabular}{|m{3.38376in}|m{3.38376in}|}
\hline
\sffamily\itshape Action &
\sffamily\itshape File Menu Bar Item\\\hline
\sffamily\itshape Go to Specific Line &
{\sffamily\bfseries\itshape\color[rgb]{0.0,0.2784314,1.0} Edit
{\textgreater} Go to Line} \\\hline
\end{supertabular}
\end{center}

\bigskip

{\sffamily
The {\textquotedblleft}Go To Line{\textquotedblright} function on the
File Menu Bar lets you go to a specific line in your code. For
instance, if you want to go to a specific line to fix an error, this
function lets you go directly to that line and sparing you the hassle
of finding it yourself by having to scroll through your code.}


\bigskip

{\sffamily\bfseries\itshape\color[rgb]{0.0,0.2784314,1.0}
1. To go to a specific line in your code, select {\textquotedblleft}Edit
{\textgreater} Go to Line{\textquotedblright}; a new window will
appear}


\bigskip

{\sffamily\bfseries\itshape\color[rgb]{0.0,0.2784314,1.0}
2. In the new window, type in the number of the line you want to go to}



\begin{figure}
\centering
\includegraphics[width=4.7902in,height=2.6445in]{userguid-img146.jpg}
\end{figure}

\bigskip

{\sffamily\bfseries\itshape\color[rgb]{0.0,0.2784314,1.0}
3. Click on {\textquotedblleft}Okay{\textquotedblright} to be taken to
the line you entered}


\bigskip


\bigskip

{\centering\bfseries\itshape
Inserting Unicon Code in your file
\par}

{\sffamily
CVE also lets you insert premade Unicon Code in your file for
\textbf{\textit{Procedure}}, \textbf{\textit{Class}} and
\textbf{\textit{Method}}.}


\bigskip

{\sffamily
\textbf{\textit{NOTE}}: \textbf{The Insert item is specifically geared
towards Unicon}, because it lets you insert standard code in Unicon for
\textbf{Procedure}, \textbf{Class} and \textbf{Method}. Since these are
standard in Unicon code, they will not work when you are creating Java
and C/C++ programs!}

{\sffamily
Using Insert spares you the task of having to write the same lines of
Unicon code, instead letting you focus on your program without needing
to write the standard code for these items. }


\bigskip

{\sffamily
The Insert items lets you add standard code to your program. Hence,
every time you have a file open, and press
\textbf{\textit{Insert{\textgreater}Procedure}}, the following code
will be inserted in your Unicon program:}


\bigskip

{\sffamily
\ \ \ \ \texttt{\textbf{\ \ procedure ()}}}

{\ttfamily\bfseries
\ \ local}

{\ttfamily\bfseries
\ \ end}

{\sffamily
This code is standard code for a procedure in Unicon. Since this code is
Unicon specific, you can only use the Insert function when you are
writing a program in Unicon. }



\begin{figure}
\centering
\includegraphics[width=6.4437in,height=4.5425in]{userguid-img147.jpg}
\end{figure}
{\sffamily\bfseries\itshape\color[rgb]{0.0,0.2784314,1.0}
To insert precreated Unicon Code for Procedure, Class, Method, in the
File Menu Bar Select Inser {\textgreater} Procedure , Class, Method or
the IDE Icon\newline
}

\begin{center}
\tablehead{}
\begin{supertabular}{|m{3.38376in}|m{1.6525599in}|m{1.6525599in}|}
\hline
\sffamily\itshape Action &
\sffamily\itshape File Menu Bar Item &
\sffamily\itshape IDE Icon\\\hline
\sffamily\itshape \ Add precreated Unicon code for Procedure in your
program &
{\sffamily\bfseries\itshape\color[rgb]{0.0,0.2784314,1.0} Insert
{\textgreater} Procedure} &
%\centering
\includegraphics[width=0.4307in,height=0.3472in]{userguid-img148.jpg}
\\
\hline
\sffamily\itshape Add precreated Unicon code for Class in your program 
&
{\sffamily\bfseries\itshape\color[rgb]{0.0,0.2784314,1.0} Insert
{\textgreater} Class} &
%\centering
\includegraphics[width=0.3472in,height=0.3055in]{userguid-img149.jpg}
\\\hline
\sffamily\itshape Add precreated Unicon code for Method in your program
&
{\sffamily\bfseries\itshape\color[rgb]{0.0,0.2784314,1.0} Insert
{\textgreater} Method} &
%\centering
\includegraphics[width=0.3752in,height=0.3193in]{userguid-img150.jpg}
\\\hline
\end{supertabular}
\end{center}

\bigskip


\bigskip


\bigskip


\bigskip

\subsubsection[Running and debugging your files]{Running and debugging
your files}
{\sffamily
After creating and saving your file, you can also run your files to see
if they contain any errors or if they work. The
\textbf{\textit{Compile}} item in the \textbf{\textit{File Menu Bar
}}lets you compile your programs and check them for errors, as well
creating executable files. In addition, you can configure various
options for the compiler to run Java/C/C++ and Unicon code.}

\begin{figure}
\centering
\includegraphics[width=3.139in,height=1.722in]{userguid-img151.jpg}
\end{figure}
{\centering\bfseries\itshape
%\newline
\textsf{Creating executables}
\par}

\begin{center}
\tablehead{}
\begin{supertabular}{|m{3.38376in}|m{1.6525599in}|m{1.6525599in}|}
\hline
\sffamily\itshape Action &
\sffamily\itshape File Menu Bar Item &
\sffamily\itshape IDE Icon\\\hline
\sffamily\itshape Create .exe file so you can run your program &
{\sffamily\bfseries\itshape\color[rgb]{0.0,0.2784314,1.0} Compile
{\textgreater} Make executable} &
%\centering
\includegraphics[width=0.4165in,height=0.3335in]{userguid-img152.jpg}
\\\hline
\end{supertabular}
\end{center}

\bigskip

{\sffamily\bfseries\itshape\color[rgb]{0.0,0.2784314,1.0}
In order to create an executable only, select {\textquotedblleft}Compile
{\textgreater} Make Executable{\textquotedblright} or click the IDE
Icon}

{\sffamily
This will only create a .exe file; it will not run your program. To do
so, follow the instructions for running your program below. }

{\centering\bfseries\itshape
%\newline
\textsf{Compiling your programs:}
\par}


\bigskip

\begin{center}
\tablehead{}
\begin{supertabular}{|m{3.38376in}|m{3.38376in}|}
\hline
\sffamily\itshape Action &
\sffamily\itshape File Menu Bar Item\\\hline
\sffamily\itshape Check for errors in your program and compile \  &
{\sffamily\bfseries\itshape\color[rgb]{0.0,0.2784314,1.0} Compile
{\textgreater} Compile Only} \\\hline
\end{supertabular}
\end{center}

\bigskip

{\sffamily
For checking errors in your program without running it, you can use the
\textbf{\textit{{\textquotedblleft}Compile {\textgreater} Compile
Only{\textquotedblright}}} function}

{\sffamily\bfseries\itshape\color[rgb]{0.0,0.2784314,1.0}
%\newline
1. Save your file by selecting {\textquotedblleft}Files {\textgreater}
Save{\textquotedblright} or {\textquotedblleft}File {\textgreater} Save
As{\textquotedblright}\newline
\newline
2. Select {\textquotedblleft}Compile {\textgreater} Compile
Only{\textquotedblright} to compile your file\newline
\newline
3. Check for errors\newline
}

\begin{figure}
\centering
\includegraphics[width=4.7398in,height=3.7764in]{userguid-img153.jpg}
\end{figure}
{\sffamily\bfseries\itshape\color[rgb]{0.0,0.2784314,1.0}
4. In case of errors, you will automatically be taken to the error line
in your program, with a message of the error in the debug/run window}

{\sffamily
\textbf{\newline
No errors:}\newline
If you have no errors, the following message will be displayed in the
debug / run window saying:}

{\sffamily
{\textquotedblleft}\texttt{0 error(s) \&
warning(s)}{\textquotedblright}\newline
\newline
}

{\sffamily
%\newline
\textbf{Errors:}}

{\sffamily
In case you have any errors, the message will display the number of
errors and take you to the error in the code.}

{\sffamily
For instance:}

{\sffamily
{\textquotedblleft}\texttt{2 error(s) \& warning(s)}{\textquotedblright}
means that there are 2 errors.}


\bigskip

{\centering\bfseries\itshape
% \newline
Running your Programs
\par}

\begin{figure}
\centering
\includegraphics[width=5.4126in,height=3.0665in]{userguid-img154.jpg}
\end{figure}

\bigskip

\begin{center}
\tablehead{}
\begin{supertabular}{|m{3.38376in}|m{1.6525599in}|m{1.6525599in}|}
\hline
\sffamily\itshape Action &
\sffamily\itshape File Menu Bar Item &
\sffamily\itshape IDE Icon\\\hline
\sffamily\itshape Execute your program &
{\sffamily\bfseries\itshape\color[rgb]{0.0,0.2784314,1.0} Run
{\textgreater} Run Program}  &
%\centering
\includegraphics[width=0.5555in,height=0.361in]{userguid-img155.jpg}
\\\hline
\end{supertabular}
\end{center}
{\sffamily
%\newline
After creating your program and compiling and debugging it, you can also
test your program by using the\textbf{\textit{ Run{\textgreater}Run
Program }}\textit{function}}


\bigskip

{\sffamily\bfseries\itshape\color[rgb]{0.0,0.2784314,1.0}
Use Run {\textgreater} Run Program to run your program \newline
}

{\centering\bfseries\itshape
Configuring your compiler: 
\par}

{\sffamily
%\newline
The \textbf{\textit{{\textquotedblleft}Compile {\textgreater} Compile
Options{\textquotedblright} }}\ lets you configure your compiler
options for programs in }

{\sffamily
Unicon/C/C++ and Java. \newline
}

\begin{center}
\tablehead{}
\begin{supertabular}{|m{3.38376in}|m{3.38376in}|}
\hline
\sffamily\itshape Action &
\sffamily\itshape File Menu Bar Item\\\hline
\sffamily\itshape Configure Compiler options for Unicon/C/C++ and Java &
{\sffamily\bfseries\itshape\color[rgb]{0.0,0.2784314,1.0} Compile
{\textgreater} Compile Options} \\\hline
\end{supertabular}
\end{center}
{\sffamily
\textbf{\newline
Unicon Compiler Options:}\newline
The Compiler Options for Unicon lets you configure the Compiler (Unicon
or Wincont), as well as let you configure the \textbf{Compiler} and
\textbf{Linker} Flags and Command. By default, Unicon is selected as
compiler.\newline
}

\begin{figure}
\centering
\includegraphics[width=4.1654in,height=2.2602in]{userguid-img156.jpg}
\end{figure}
{\sffamily\bfseries
Java Compiler Options:}

{\sffamily
The Compiler Options for Java lets you configure the \textbf{Compiler}
and \textbf{Linker} Flags and Command. By default, \textbf{Javac} is
always selected as compiler for Java programs}



\begin{figure}
\centering
\includegraphics[width=3.9409in,height=2.5547in]{userguid-img157.jpg}
\end{figure}
{\sffamily\bfseries
C/C++ Compiler Options:}

{\sffamily
The Compiler Options for C/C++ lets you configure the \textbf{Compiler}
and \textbf{Linker} Flags and Command. By default, \textbf{GNU C++
Compiler (G++)} is selected as compiler for C ++ programs programs;
However, \textbf{GNU C Compiler (GCC) }can also be selected if you are
developing C programs.}



\begin{figure}
\centering
\includegraphics[width=4.0701in,height=2.4453in]{userguid-img158.jpg}
\end{figure}

\bigskip

{\sffamily
Many of these options are for advanced users who want to configure their
compilers. For simple programs, you should keep the default settings.}

\subsubsection{Configuring your workspace}

\bigskip

{\sffamily
In CVE, it is possible to configure your workspace for maximizing your
screen real estate, allowing you to focus on tasks at hand. Using the
\textbf{\textit{{\textquotedblleft}View {\textgreater}
Window{\textquotedblright}}} function, you can select a screen setup
that will let you focus on particular elements on your screen by
minimizing others. Hence, when creating a file, you might want to focus
on the File Workscreen and the Debug / Run window, and thus you select
\textbf{{\textquotedblleft}}\textbf{\textit{View {\textgreater} Window
{\textgreater} Files / Messages}}\textbf{{\textquotedblright}}. Or,
perhaps you want to focus on just your code, so you select
\textbf{\textit{{\textquotedblleft}View {\textgreater} Window
{\textgreater} Files{\textquotedblright}}}\textit{ }\ so that you only
see your File workscreen. }


\bigskip

{\sffamily
\textbf{\textit{NOTE:}}\textbf{ }In CVE, it is possible to configure
your workspace for maximizing your screen real estate, allowing you to
focus on tasks at hand by minimizing or closing non-essential
screens.\textbf{ }At any time, it is possible to switch back to the
original configuration with all workwindows on your screen by selecting
\textbf{\textit{{\textquotedblleft}View {\textgreater} Window
{\textgreater} ALL{\textquotedblright}}}}


\bigskip


\bigskip

{\centering\bfseries\itshape
Changing the Window size
\par}

\begin{center}
\tablehead{}
\begin{supertabular}{|m{3.38376in}|m{1.6525599in}|m{1.6525599in}|}
\hline
\sffamily\itshape Action &
\sffamily\itshape File Menu Bar Item &
\sffamily\itshape IDE Icon\\\hline
\sffamily\itshape Display All Windows &
{\sffamily\bfseries\itshape\color[rgb]{0.0,0.2784314,1.0} View
{\textgreater} Window{\textgreater} \newline
ALL} &
%\centering
\includegraphics[width=0.4165in,height=0.3335in]{userguid-img159.jpg}
\\\hline
\sffamily\itshape Display File workscreen only &
{\sffamily\bfseries\itshape\color[rgb]{0.0,0.2784314,1.0} View
{\textgreater} Window{\textgreater} Files} &
%\centering
\includegraphics[width=0.4028in,height=0.3055in]{userguid-img160.jpg}
\\\hline
\sffamily\itshape Display File workscreen and Run /Debug Window &
{\sffamily\bfseries\itshape\color[rgb]{0.0,0.2784314,1.0} View
{\textgreater} Window{\textgreater} }

{\sffamily\bfseries\itshape\color[rgb]{0.0,0.2784314,1.0} Files /
Messages} &
%\centering
\includegraphics[width=0.4165in,height=0.3335in]{userguid-img161.jpg}
\\\hline
\sffamily\itshape Display File workscreen and Class Browser &
{\sffamily\bfseries\itshape\color[rgb]{0.0,0.2784314,1.0} View
{\textgreater} Window {\textgreater}}

{\sffamily\bfseries\itshape\color[rgb]{0.0,0.2784314,1.0} Files Class /
Browser} &
~
\\\hline
\end{supertabular}
\end{center}


\begin{figure}
\centering
\includegraphics[width=5.989in,height=4.4492in]{userguid-img162.jpg}
\end{figure}
{\sffamily
%\newline
\textbf{Display ALL Windows (default):}}

{\sffamily\bfseries\itshape\color[rgb]{0.0,0.2784314,1.0}
To display all windows, select View {\textgreater} Window {\textgreater}
All}

{\sffamily\bfseries
%\newline
Display File workscreen only:}

{\sffamily\bfseries\itshape\color[rgb]{0.0,0.2784314,1.0}
To display only the File workscreen (pictured above), select View
{\textgreater} Window {\textgreater} Files}

{\sffamily\bfseries
%\newline
Display File and Debug / Run Window only:}

{\sffamily\bfseries\itshape\color[rgb]{0.0,0.2784314,1.0}
To display only the Files and Debug / Run Window, select View
{\textgreater} Window {\textgreater} Files /Messages\newline
}

{\sffamily\bfseries
Display File and Class Browser only:}

{\sffamily\bfseries\itshape\color[rgb]{0.0,0.2784314,1.0}
To display only the File and the Class Browser, select View
{\textgreater} Window {\textgreater} Files / Class Browsers}

{\centering\bfseries\itshape
%\newline
Hiding or Showing the IDE toolbar
\par}

\begin{center}
\tablehead{}
\begin{supertabular}{|m{3.38376in}|m{3.38376in}|}
\hline
\sffamily\itshape Action &
\sffamily\itshape File Menu Bar Item\\\hline
\sffamily\itshape Hide or Show the IDE toolbar above a file that you
opened or created &
{\sffamily\bfseries\itshape\color[rgb]{0.0,0.2784314,1.0} View
{\textgreater} Hide/Show Toolbars} \\\hline
\end{supertabular}
\end{center}

\bigskip

{\sffamily
\textbf{\textit{NOTE:}} By default, the IDE Toolbar is opened
automatically when you open a file. It is possible to hide the IDE
Toolbar, by selecting \textbf{\textit{{\textquotedblleft}View
{\textgreater} Hide Toolbar{\textquotedblright}}} \ so you have more
room for your filescreen. To show the IDE toolbar again, select
\textbf{\textit{{\textquotedblleft}View {\textgreater} Show
Toolbar{\textquotedblright}}}.}



\begin{figure}
\centering
\includegraphics[width=4.5736in,height=3.5827in]{userguid-img163.jpg}
\end{figure}
{\sffamily\bfseries
Hide the IDE toolbar:}

{\sffamily\bfseries\itshape\color[rgb]{0.0,0.2784314,1.0}
To hide the IDE Toolbar, select {\textquotedblleft}View {\textgreater}
Hide Toolbar{\textquotedblright}}

{\sffamily
%\newline
\textbf{Show the IDE Toolbar:}}

{\sffamily\bfseries\itshape\color[rgb]{0.0,0.2784314,1.0}
To show the IDE Toolbar, select {\textquotedblleft}View {\textgreater}
Show Toolbar{\textquotedblright}}


\bigskip

\clearpage{\centering\bfseries\itshape
Changing the size of the Debug/ Run Window
\par}


\bigskip

\begin{center}
\tablehead{}
\begin{supertabular}{|m{3.38376in}|m{3.38376in}|}
\hline
\sffamily\itshape Action &
\sffamily\itshape File Menu Bar Item\\\hline
\sffamily\itshape Change the size of the Debug / Run Window &
{\sffamily\bfseries\itshape\color[rgb]{0.0,0.2784314,1.0} View
{\textgreater} Messages}\\\hline
\end{supertabular}
\end{center}

\bigskip

{\sffamily
\textbf{\textit{NOTE:}}\textit{ }When compiling and debugging your
programs, it is a good idea to expand your \textbf{Debug / Run window},
so you can see your program messages. Whereas normal size only gives
you a couple of lines, expanding the Debug / Run window to
\textbf{medium} or \textbf{maximum} size will allow you a better view
of the messages contained in this window.}

\begin{figure}
\centering
\includegraphics[width=5.6161in,height=4.0535in]{userguid-img164.jpg}
\end{figure}

\bigskip

\clearpage{\centering\bfseries\itshape
Changing the size of the \textsf{\textup{User Class/File Management \&
Collaboration Area}}
\par}


\bigskip


\bigskip

\begin{center}
\tablehead{}
\begin{supertabular}{|m{3.38376in}|m{3.38376in}|}
\hline
\sffamily\itshape Action &
\sffamily\itshape File Menu Bar Item\\\hline
{\sffamily\itshape Expand the User Class / File Management and
Collaboration Area }

~
 &
{\sffamily\bfseries\itshape\color[rgb]{0.0,0.2784314,1.0} View
{\textgreater} Class Browser}

~
\\\hline
\end{supertabular}
\end{center}

\bigskip

{\sffamily
It is also possible to change the size of the Collaboration area by
using {\textquotedblleft}\textbf{View \ {\textgreater} Class
Browser}{\textquotedblright} and selecting either a medium or maximum
size.\newline
}

\begin{figure}
\centering
\includegraphics[width=5.8665in,height=3.9799in]{userguid-img165.jpg}
\end{figure}
\subsubsection[Connecting/Disconnecting to the Network]{\newline
Connecting/Disconnecting to the Network}
{\sffamily
At times, you will probably want to get on with your work in CVE without
getting interrupted by other users. There are two possibilities for
working in CVE without being visible to other users.}

\liststyleLxxvi
\begin{enumerate}
\item[] {\sffamily\bfseries
1. You can use the standalone version of CVE, which lets you log in to
CVE without connecting to the network}

{\sffamily\bfseries\itshape
or }

{\sffamily\bfseries
%\newline
2. You can disconnect while already logged into CVE by going to the
\textit{Network {\textgreater} Disconnect} in the File Menu Bar.}
\end{enumerate}

\bigskip


\bigskip

{\sffamily
\textbf{\textit{NOTE: }}It is possible to do your work uninterrupted in
CVE by using the \textbf{\textit{{\textquotedblleft}Network
{\textgreater} Disconnect{\textquotedblright}}} function from the File
Menu Bar or, when starting CVE, using the\textbf{\textit{ Run
Standalone}} option at the login window. This especially useful when
you want to focus on your own work and do not want to interact with
other users.}


\bigskip


\bigskip

{\sffamily\bfseries
Logging in to CVE without connecting to the Server:}

{\sffamily\bfseries\itshape\color[rgb]{0.0,0.2784314,1.0}
To use CVE without connecting to the server initially, select the
{\textquotedblleft}Run Standalone{\textquotedblright} function at the
login window.}



\begin{figure}
\centering
\includegraphics[width=4.3047in,height=2.6984in]{userguid-img166.jpg}
\end{figure}
{\sffamily
%\newline
Since you are not connected to the network, you will not be able to see
the avatars of other users or be able to chat with them, but you will
be capable of working on your projects in CVE and navigating the
environment. You will also have a default avatar (as opposed to your
own), since standalone mode does not require you to enter your username
and password. }

{\sffamily\bfseries
To Disconnect from the Network when logged in: }

{\sffamily
\textbf{\textit{NOTE:}} By default, when you log in to CVE with your
username and password, you will be connected to the network. At any
time, when you want to disconnect, you can do this by selecting
\textbf{\textit{{\textquotedblleft}Network {\textgreater}
Disconnect{\textquotedblright} }}}


\bigskip

\begin{center}
\tablehead{}
\begin{supertabular}{|m{3.38306in}|m{3.38446in}|}
\hline
\sffamily\itshape Action &
\sffamily\itshape File Menu Bar Item\\\hline
\sffamily\itshape Disconnect to the server &
{\sffamily\bfseries\itshape\color[rgb]{0.0,0.2784314,1.0} Network
{\textgreater} Disconnect}\\\hline
\end{supertabular}
\end{center}
{\sffamily\bfseries\itshape\color[rgb]{0.0,0.2784314,1.0}
%\newline
When you have already logged into CVE using your username and password
and want to disconnect from the network, select
{\textquotedblleft}Network {\textgreater}
Disconnect{\textquotedblright}. }


\bigskip

{\sffamily\bfseries
To Connect to the Network again:}

\begin{center}
\tablehead{}
\begin{supertabular}{|m{3.38376in}|m{3.38376in}|}
\hline
\sffamily\itshape Action &
\sffamily\itshape File Menu Bar Item\\\hline
\sffamily\itshape Connect to the server &
{\sffamily\bfseries\itshape\color[rgb]{0.0,0.2784314,1.0} Network
{\textgreater} Connect}\\\hline
\end{supertabular}
\end{center}
{\sffamily\bfseries\itshape\color[rgb]{0.0,0.2784314,1.0}
%\newline
In order to connect to the network, select {\textquotedblleft}Network
{\textgreater} Connect{\textquotedblright} }

{\centering \par}

\begin{figure}
\centering
\includegraphics[width=2.5693in,height=1.5138in]{userguid-img167.jpg}
\end{figure}

\subsubsection{Changing your Account Information}
{\sffamily
In CVE it is possible to change the password you use to log in to CVE,
as well as your username and personal information such as your screen
name and affiliation.}


\bigskip

{\centering\bfseries\itshape
Changing your Password
\par}


\bigskip

\begin{center}
\tablehead{}
\begin{supertabular}{|m{3.38376in}|m{3.38376in}|}
\hline
\sffamily\itshape Action &
\sffamily\itshape File Menu Bar Item\\\hline
\sffamily\itshape Change your password &
{\sffamily\bfseries\itshape\color[rgb]{0.0,0.2784314,1.0} Account
{\textgreater} Change Password} \\\hline
\end{supertabular}
\end{center}

\bigskip

{\sffamily\bfseries\itshape\color[rgb]{0.0,0.2784314,1.0}
1. To change your password, select the Account {\textgreater} Change
Password option in the file menu bar}



\begin{figure}
\centering
\includegraphics[width=3.0972in,height=1.6807in]{userguid-img168.jpg}
\end{figure}
{\sffamily\bfseries\itshape\color[rgb]{0.0,0.2784314,1.0}
%\newline
2. In the {\textquotedblleft}Edit Your Password{\textquotedblright}
window, enter your username, your old password and your new password
and click {\textquotedblleft}OK{\textquotedblright} }

{\sffamily
%\newline
That{\textquotesingle}s it! If all things have gone well, you should
have a message reading in the Chat Area/ Message Window:}

\begin{figure}
\centering
\includegraphics[width=3.9307in,height=2.9165in]{userguid-img169.jpg}
\end{figure}
{\ttfamily
passwordchange():password change request for user [your
username]\newline
}

{\sffamily
You can now use your new password when logging in to CVE! \newline
}

{\sffamily
\textbf{\textit{NOTE:}} If by any chance you have lost your password, or
you are having trouble logging in to CVE, contact the system
administrator \textbf{jeffery@cs.uidaho.edu}}


\bigskip

{\centering\bfseries\itshape
Changing your Registration Information
\par}

\begin{center}
\tablehead{}
\begin{supertabular}{|m{3.38376in}|m{3.38376in}|}
\hline
\sffamily\itshape Action &
\sffamily\itshape File Menu Bar Item\\\hline
\sffamily\itshape Edit your user{\textquotesingle}s
email/username/first/last name and affiliation &
{\sffamily\bfseries\itshape\color[rgb]{0.0,0.2784314,1.0} Account
{\textgreater} Edit Registration}\\\hline
\end{supertabular}
\end{center}
{\sffamily
%\newline
In CVE it is also possible to change the Registration information you
gave when you first created your avatar. }

{\sffamily\bfseries\itshape\color[rgb]{0.0,0.2784314,1.0}
%\newline
1.To change your Register Information, select {\textquotedblleft}Account
{\textgreater} Edit Registration{\textquotedblright}}



\begin{figure}
\centering
\includegraphics[width=3.878in,height=3.4272in]{userguid-img170.jpg}
\end{figure}
{\sffamily\bfseries\itshape\color[rgb]{0.0,0.2784314,1.0}
2. Fill in your Username and Password, and any information you want to
change, such as FirstName, Lastname, EmailID and Institutional
Affiliation, and click {\textquotedblleft}OK{\textquotedblright}}


\bigskip

{\sffamily
That{\textquotesingle}s it! If all things have gone well, you should
have a message reading in the Chat Area/ Message Window:\newline
}

{\ttfamily
changereg\_info():registration information change requested for user
[your username]\newline
}

\subsubsection[Printing Your File]{Printing Your File}
{\sffamily
The File {\textgreater} Print item also lets you print file and file
code, so that you can go over what you have just created}

\begin{center}
\tablehead{}
\begin{supertabular}{|m{3.38376in}|m{3.38376in}|}
\hline
\sffamily\itshape Action &
\sffamily\itshape File Menu Bar Item\\\hline
\sffamily\itshape Print a file:  &
{\sffamily\bfseries\itshape\color[rgb]{0.0,0.2784314,1.0} File
{\textgreater} Print}\\\hline
\end{supertabular}
\end{center}

\bigskip

\subsubsection{Saving a Chat Session or Screenshot}
{\sffamily
The \textbf{\textit{{\textquotedblleft}File {\textgreater} Save ChatWin
Transcript{\textquotedblright}}} item in the File Menu Bar lets you
save chats as .txt files so that you can refer to them later or remind
yourself what was discussed in that particular chat. You can also Save
a Screenshot by selecting\textbf{\textit{ {\textquotedblleft}File
{\textgreater} Save 3d Screenshot{\textquotedblright}}}. }

{\sffamily
\textbf{\textit{NOTE: }}It is important to save your Chats and take
relevant screenshots of moments in CVE, so that you can refresh your
memory and refer to them later. Do make sure to ask permission from
other users if you want to save their chat with you!!}

{\centering\bfseries\itshape
%\newline
Saving a Chat Session
\par}

\begin{center}
\tablehead{}
\begin{supertabular}{|m{3.38376in}|m{3.38376in}|}
\hline
\sffamily\itshape Action &
\sffamily\itshape File Menu Bar Item\\\hline
\sffamily\itshape Save a chat &
{\sffamily\bfseries\itshape\color[rgb]{0.0,0.2784314,1.0} File
{\textgreater} Save ChatWin}\\\hline
\end{supertabular}
\end{center}

\bigskip

{\sffamily\bfseries\itshape\color[rgb]{0.0,0.2784314,1.0}
1. In order to save a chat, select File {\textgreater} Save Chatwin in
the File Menu Bar}



\begin{figure}
\centering
\includegraphics[width=1.4283in,height=1.8138in]{userguid-img171.jpg}
\end{figure}
{\sffamily\bfseries\itshape\color[rgb]{0.0,0.2784314,1.0}
2. In the {\textquotedblleft}Save Transcript{\textquotedblright} window,
select the directory and the filename (which should end in .txt), and
click {\textquotedblleft}Okay{\textquotedblright} to save the
transcript}



\begin{figure}
\centering
\includegraphics[width=4.9256in,height=3.5654in]{userguid-img172.jpg}
\end{figure}
{\centering\bfseries\itshape
Saving a Screenshot
\par}


\bigskip

\begin{center}
\tablehead{}
\begin{supertabular}{|m{3.38376in}|m{3.38376in}|}
\hline
\sffamily\itshape Action &
\sffamily\itshape File Menu Bar Item\\\hline
{\sffamily\itshape Save a screenshot}

~
 &
{\sffamily\bfseries\itshape\color[rgb]{0.0,0.2784314,1.0}
File{\textgreater} Save Screenshot}

~
\\\hline
\end{supertabular}
\end{center}
{\sffamily
%\newline
It is a good idea to save a screenshot of your interactions in CVE so
you can revisit this later. For instance, when your instructor gives
you important information on the blackboard, you can take a screenshot
to keep this information for reference. \newline
}

{\sffamily\bfseries\itshape\color[rgb]{0.0,0.2784314,1.0}
1. To save a screenshot, select {\textquotedblleft}File {\textgreater}
Save Screenshot{\textquotedblright} in the File Menu Bar}



\begin{figure}
\centering
\includegraphics[width=1.8138in,height=1.9366in]{userguid-img173.jpg}
\end{figure}
{\sffamily\bfseries\itshape\color[rgb]{0.0,0.2784314,1.0}
2. In the {\textquotedblleft}Save Screenshot{\textquotedblright} window,
select the directory, name and extension (.jpg) that you want to save
the screenshot as}

%\newline


\begin{figure}
\centering
\includegraphics[width=3.6071in,height=2.7374in]{userguid-img174.jpg}
\end{figure}
{\sffamily\bfseries\itshape\color[rgb]{0.0,0.2784314,1.0}
3. Click on {\textquotedblleft}Okay{\textquotedblright} to save your
screenshot.}


\bigskip

\subsubsection{Getting Help}
{\sffamily
The Help item in the File Menu Bar provides you with a user guide that
offers helpful tips in using CVE. Next to a manual, there is a command
list for commonly used commands such as avatar movement, teleporting,
and other function keys. }


\bigskip

{\centering\bfseries\itshape
Opening the User Guide\newline

\par}

\begin{center}
\tablehead{}
\begin{supertabular}{|m{3.38376in}|m{3.38376in}|}
\hline
\sffamily\itshape Open User Guide &
{\sffamily\bfseries\itshape\color[rgb]{0.0,0.2784314,1.0} Help
{\textgreater} User Guide} \\\hline
\end{supertabular}
\end{center}
{\sffamily\bfseries\itshape\color[rgb]{0.0,0.2784314,1.0}
%\newline
To open the user guide, select Help {\textgreater} User Guide in the
File Menu Bar}


\bigskip

{\sffamily
The user guide can help you in case you have any issues in CVE.}

{\centering\bfseries\itshape
%\newline
Displaying commonly used commands
\par}

\begin{center}
\tablehead{}
\begin{supertabular}{|m{3.38376in}|m{3.38376in}|}
\hline
\sffamily\itshape Display a short list of commonly used commands in CVE
&
{\sffamily\bfseries\itshape\color[rgb]{0.0,0.2784314,1.0} Help
{\textgreater} Commands}\\\hline
\end{supertabular}
\end{center}
{\sffamily\bfseries\itshape\color[rgb]{0.0,0.2784314,1.0}
%\newline
To see a list of commonly used commands, select Help {\textgreater}
Commands in the File Menu Bar\newline
}

{\sffamily
\textbf{\textit{NOTE: }}Commonly used keys for navigating through the
environment, avatar movement, and social interaction are listed in
\ \textbf{\textit{{\textquotedblleft}Help {\textgreater}
Commands{\textquotedblright}}} in the File Menu Bar.}

{\centering\bfseries\itshape
%\newline
Displaying your version of Unicon
\par}

\liststyleLxxvii
\begin{enumerate}
\item[] 
\bigskip
\end{enumerate}
\begin{center}
\tablehead{}
\begin{supertabular}{|m{3.38376in}|m{3.38376in}|}
\hline
\sffamily\itshape Display your version of CVE &
{\sffamily\bfseries\itshape\color[rgb]{0.0,0.2784314,1.0} Help
{\textgreater} About}\\\hline
\end{supertabular}
\end{center}

\bigskip

{\sffamily\bfseries\itshape\color[rgb]{0.0,0.2784314,1.0}
To display your version of CVE, select Help {\textgreater} About in the
File Menu Bar}


\bigskip

{\sffamily
It is important to know what version of CVE you are running, so that you
can update to the latest version.}

\clearpage\subsection{Chapter 9: Creating a Project using the Integrated
Development Environment}
{\sffamily
The CVE system contains an IDE (integrated development environment)
for creating Unicon, C/C++ and Java projects for Linux and Windows. The
IDE is integrated directly into the CVE virtual environment and is
available anytime, anywhere. This means that, from within the world,
the user can build, compile, and run the program without having to exit
the world and without stopping or upsetting the execution of other
users{\textquoteright} work. }

{\sffamily
This chapter will describe the following elements:}

\liststyleLii
\begin{itemize}
\item {\sffamily\bfseries\color[rgb]{0.0,0.0,0.5019608}
About the Integrated Development Environment}
\item {\sffamily\bfseries\color[rgb]{0.0,0.0,0.5019608}
Creating and Running your Projects Using the IDE Toolbar}
\item {\sffamily\bfseries\color[rgb]{0.0,0.0,0.5019608}
Managing your projects using the Project Manager}
\item {\sffamily\bfseries\color[rgb]{0.0,0.0,0.5019608}
Creating a Project in CVE\ \ }
\item {\sffamily\bfseries\color[rgb]{0.0,0.0,0.5019608}
Creating a New Project }
\item {\sffamily\bfseries\color[rgb]{0.0,0.0,0.5019608}
Opening an existing Project}
\item {\sffamily\bfseries\color[rgb]{0.0,0.0,0.5019608}
Saving a Project}
\item {\sffamily\bfseries\color[rgb]{0.0,0.0,0.5019608}
Running and Debugging a Project}
\end{itemize}
\subsubsection{About the Integrated Development Environment (IDE)}
{\sffamily
An Integrated Development Environment (IDE) is a software tool intended
to make the process of writing programs easier. IDEs typically work as
a component of the compiler environment. It allows editing program
source code, compiling it with the compiler, fixing syntax errors,
running it, and debugging it, all inside a single environment. }

{\sffamily
The CVE IDE is a simple Integrated Development Environment (IDE) for
Unicon, C/C++, and Java on Linux and Microsoft Windows. It features a
number of advanced programming facilities. The idea behind CVE's IDE
is to provide the virtual environment with a flexible and powerful
programming tool with a user-friendly Graphical User Interface (GUI). }

{\sffamily
CVE's IDE 
is integrated directly into the virtual environment and is available
anytime, anywhere. This means that, from within the world, the user can
build, compile, and run the program without having to exit the world
and without stopping or upsetting the execution of other
users{\textquoteright} work. The IDE gives users the ability to work on
a project in the virtual world individually, as well as share these
files with other users for troubleshooting and feedback. This allows
users to work on projects together and read each
other{\textquoteright}s code by using the collaborative function of the
IDE. }

{\sffamily
The CVE IDE is similar to other IDEs (for example Visual Studio;
Borland C++). The CVE IDE features allow the user to:}

\liststyleLxxviii
\begin{itemize}
\item {\sffamily
Open existing files/projects and create new files/projects. }
\item {\sffamily
Compile, execute, debug and run programs }
\item {\sffamily
Perform text editing with an easy-to-use and user-friendly text editor }
\item {\sffamily
Find errors quickly by using the error-line jumping technique }
\item {\sffamily
Manage their project structure by browsing project files/ moving
directly to files/ methods/procedures }
\end{itemize}

\bigskip

{\sffamily
The CVE IDE also includes elements that make it different from other
IDEs. Most importantly, the CVE IDE is part of the
virtual environment, allowing for collaboration and online synchronous
user-to-user interaction through voice chat, text chat, and
screensharing. The other element that makes the CVE IDE different is
that it supports three different languages (C/C++, Java and Unicon),
and is suited to support programming, compiling and debugging programs
written in these languages.}



\begin{figure}
\centering
\includegraphics[width=6.0583in,height=3.6701in]{userguid-img175.jpg}
\end{figure}
{\centering\sffamily\itshape
Figure 1: A view of the CVE IDE.
\par}


\bigskip

{\sffamily
A view of the IDE is provided above. The IDE consists of four areas of
importance:}


\bigskip

\liststyleLxxix
\begin{itemize}
\item {\sffamily
The \textbf{\textit{Code Area}} is where you write your programs in
either C/C++/Java or Unicon.}
\item {\sffamily
The\textbf{ }\textbf{\textit{IDE toolbar}} lets you open, save,
compile,debug and run files, create executable files, and cut, copy,
paste and insert code. Many of these functions are similar to those
found in the File Menu Bar.}
\item {\sffamily
The\textbf{ }\textbf{\textit{Class Browser Tab}} (located in the
Class/File Management and Collaboration Area), has multiple functions.
It lets you see your project including files you have linked to your
project. It also lets you share your files with other users by inviting
other users through the \textbf{Users Tab.}}
\item {\sffamily
The \textbf{\textit{Debug/Run Window}} gives you diagnostic messages
when compiling and running your program. }
\end{itemize}
{\sffamily
More in-depth explanations of how the CVE IDE works are described in
the sections below.}

{\sffamily
\textbf{\textit{NOTE:}} An \textbf{\textit{Integrated Development
Environment }}(IDE) is a software tool intended to make the process of
writing programs easier by letting you edit, compile, fix syntax
errors, run and debug program source code all inside a single
environment. }

{\sffamily
The \textbf{\textit{CVE IDE}} lets people create, compile, run and
debug code for Unicon, C/C++ and Java Projects, and share this code
with others.}


\bigskip

\subsubsection{Creating and Running your Projects Using the IDE Toolbar}
{\sffamily
The\textbf{ }\textbf{\textit{IDE toolbar}} lets you open, save,
compile,debug and run files, create executable files, and cut, copy,
paste and insert code. It also lets you maximize the size of your code
screen by expanding it. \ Many of these functions are similar to those
found in the File Menu Bar, but since they are in the IDE screen, these
functions let you get on with your coding work without having to access
these functions all the way at the top in the File Menu Bar. }


\bigskip

%\newline


\begin{figure}
\centering
\includegraphics[width=6.9252in,height=4.3709in]{userguid-img176.jpg}
\end{figure}
\subsubsection[Managing your projects using the Project
Manager]{\bfseries Managing your projects using the Project Manager}

\bigskip

{\sffamily
A Project is the collection of all the source files and the required
settings for the compiler, assembler, and linker in order to compile
and link a program. CVE's IDE has a simple and easy project
management feature in the \textbf{\textit{Class Browser Tab }}area that
will help you set up your project in no time. }


\bigskip

{\sffamily
The Project Manager helps organize your source code and simplifies the
application development process, by displaying a logical view of your
project that you can refer to at any time. \ Projects consists of
multiple files. It is important to keep track of your project files,
and the CVE IDE lets you organize this in an easy-to-understand
format that you can browse through.}

\begin{figure}
\centering
\includegraphics[width=2.1937in,height=2.5839in]{userguid-img177.jpg}
\end{figure}

\bigskip

{\sffamily
Any application consists of various source files, associated header
files, and required data files. These files (header and data files) are
generally dependent on each other; thus to make the application work a
link to these files is required. By creating a CVE IDE Project for
any application, all the file dependencies are automatically handled
and included within the project Makefile. This means that you can run
your projects without needing to link them separately using command
line strings.}


\bigskip

{\sffamily
The CVE IDE uses the standard tool make for C/C++ files. However,
for Java applications generally, JDK (the Java Development Kit
compiler) is used.}


\bigskip


\bigskip

{\sffamily\bfseries\color{black}
COMPILERS}

\begin{center}
\tablehead{}
\begin{supertabular}{|m{3.38376in}|m{3.38376in}|}
\hline
\sffamily\bfseries Project Types &
\sffamily\bfseries Compiler\\\hline
\sffamily\color{black} C/C++ Projects &
\sffamily\bfseries\color{black} g++ (Default compiler) \\\hline
{\sffamily\color{black} Unicon Projects}

~
 &
\sffamily\bfseries\color{black} Unicon\\\hline
\sffamily\color{black} JAVA Projects &
\sffamily\bfseries\color{black} JDK Compiler\\\hline
\end{supertabular}
\end{center}

\bigskip


\bigskip

\subsubsection{Creating a Project in CVE}

\bigskip

{\sffamily
Generally, when you create a project in CVE, you will need to do the
following:}

\liststyleLxxx
\begin{itemize}
\item {\sffamily
\textbf{Create a new Project file} or \textbf{Open an existing Project
}}
\item {\sffamily
\textbf{Choose type of project you want to create (C/C++, Java or Unicon
Project}) }
\end{itemize}

\bigskip

{\sffamily
The first step to build programs in CVE is to create a new Project or
open an existing project. }


\bigskip

\subsubsection{Creating a New Project}
{\sffamily\bfseries\itshape\color[rgb]{0.0,0.2784314,1.0}
1. In the File Menu Bar, select
{\textquotedblleft}Project{\textgreater}New{\textgreater} and choose
between C/C++, JAVA or Unicon Project}



\begin{figure}
\centering
\includegraphics[width=6.1181in,height=4.5028in]{userguid-img178.jpg}
\end{figure}
{\sffamily\bfseries\itshape\color[rgb]{0.0,0.2784314,1.0}
%\newline
2. Give your project file a name when prompted to Save File As and click
{\textquotedblleft}Save{\textquotedblright}}



\begin{figure}
\centering
\includegraphics[width=2.8854in,height=2.4047in]{userguid-img179.jpg}
\end{figure}
{\sffamily\bfseries\itshape\color[rgb]{0.0,0.2784314,1.0}
3. Define the project files and dependencies for your project}

{\sffamily
Depending on what project (C/C++/Java/Unicon) you are creating, this
will differ. }


\bigskip

{\sffamily\color{red}
[SECTION ON how to handle file dependencies and links for C/C++/Java/
Unicon]}

\subsubsection{Opening an Existing Project}


{\sffamily\bfseries\itshape\color[rgb]{0.0,0.2784314,1.0}
1. In the File Menu Bar, select
{\textquotedblleft}Project{\textgreater}Open{\textquotedblright} and
select the project that you want to open. }



%\begin{figure}
%\centering
\includegraphics[width=4.8189in,height=1.6453in]{userguid-img180.jpg}
%\end{figure}

{\sffamily\bfseries\itshape\color[rgb]{0.0,0.2784314,1.0}
2. Click {\textquotedblleft}Open{\textquotedblright} to open the
Project}

{\sffamily
This will open up a project that you have been working on.}


\bigskip

{\sffamily
Next to Opening Projects, the IDE also lets you Run and Debug your
programs.}


\bigskip

\subsubsection{Running/Debugging a Project}

\bigskip

{\sffamily
After creating your file, it is possible to run and debug your project.
Automatically, when running a file, CVE will save your file for you. }


\bigskip

{\sffamily\bfseries\itshape\color[rgb]{0.0,0.2784314,1.0}
1. To Run your project, press \ {\textquotedblleft}Run {\textgreater}
Run Program{\textquotedblright} (in the File Menu Bar) or click the Run
Icon (in the IDE Toolbar) }



\begin{figure}
\centering
\includegraphics[width=0.5555in,height=0.361in]{userguid-img181.jpg}
\end{figure}
{\sffamily
Since the CVE IDE contains an automatic debugger, you will see any
errors in your code displayed in the Run / Debug Window below your
file.}


\bigskip


\bigskip



\begin{figure}
\centering
\includegraphics[width=6.0291in,height=4.5252in]{userguid-img182.jpg}
\end{figure}
\subsubsection{Saving your project}

\bigskip

{\sffamily
At any time it is possible to save your project. It is a good idea to
back up your work everytime, so that in case of a malfunction, you can
still retrieve your work from the saved file.}

{\sffamily
\textbf{\textit{NOTE:}} Whenever you run a program, your file gets saved
automatically}


\bigskip

{\sffamily\bfseries\itshape\color[rgb]{0.0,0.2784314,1.0}
1. To Save your project, click on {\textquotedblleft}File {\textgreater}
Save{\textquotedblright} or {\textquotedblleft}File {\textgreater} Save
As{\textquotedblright} (in the File Menu Bar) or click on the Save and
Save As icons (in the IDE toolbar)}

\begin{center}
\tablehead{}
\begin{supertabular}{|m{1.4129599in}|m{1.6379598in}|}
\hline
{\sffamily\itshape Save }

\centering
\includegraphics[width=0.4165in,height=0.3055in]{userguid-img183.jpg}
~
 &
\sffamily\itshape Save As %\centering
\includegraphics[width=0.389in,height=0.3335in]{userguid-img184.jpg}
\\\hline
\end{supertabular}
\end{center}
{\sffamily\bfseries\itshape\color[rgb]{0.0,0.2784314,1.0}
%\newline
2. Click on {\textquotedblleft}Save{\textquotedblright} to save your
file}


\bigskip


\bigskip


\bigskip

\clearpage\subsection{Chapter 10: Creating Unicon Projects in CVE}
{\sffamily
This chapter describes the specific steps involved in creating a Unicon
project using the CVE IDE. These steps include:}

\liststyleLii
\begin{itemize}
\item {\sffamily\bfseries\color[rgb]{0.0,0.0,0.5019608}
Creating a Unicon Project}
\item {\sffamily\bfseries\color[rgb]{0.0,0.0,0.5019608}
Saving your Unicon Project}
\item {\sffamily\bfseries\color[rgb]{0.0,0.0,0.5019608}
Running and Debugging a Unicon Project }
\end{itemize}

\bigskip

\subsubsection{Building Unicon Projects}
{\sffamily
The steps of creating a new / opening an existing project in Unicon are
as follows:}

\liststyleLxxxi
\begin{enumerate}
\item {\sffamily\bfseries\color{black}
Open up the Dialog wizard }
\item {\sffamily\bfseries\color{black}
Fill in the target name of the application }
\item {\sffamily\bfseries\color{black}
Add source files }
\item {\sffamily\bfseries\color{black}
Compile your project}
\item {\sffamily\bfseries\color{black}
Save your project}
\item {\sffamily\bfseries\color{black}
Run and Debug your project}
\end{enumerate}

\bigskip

\subsubsection{Creating a Unicon Project}

\bigskip

{\sffamily\bfseries\itshape\color[rgb]{0.0,0.2784314,1.0}
1. Open up the Dialog Wizard by clicking on {\textquotedblleft}Project
{\textgreater} New CVE Project{\textquotedblright}}


\bigskip

{\sffamily
To create a new project, click on the
\textbf{{\textquotedbl}}\textbf{\textit{Project {\textgreater} New
{\textgreater} Unicon Project}}\textbf{{\textquotedbl}} \ File menu
item. After you select a name for your project, automatically a dialog
wizard will be opened; this wizard will let you begin defining a
project. This is a dialog box with various tab-items (Figure 7-1). }


\bigskip

{\sffamily\bfseries\itshape\color[rgb]{0.0,0.2784314,1.0}
2. Fill in the target name of the application }

{\sffamily
%\newline
After starting a new project the first thing to fill in is the name of
the program you are building \textbf{(the Target name).} This target
name depends on the name of the application we are building. }

{\sffamily
For instance, when building a Windows application, one has to name the
file + .exe, whereas Linux only requires the name for the file.}

{\sffamily
For instance, if the project name is \textbf{[test]} on Windows we can
use for example \textbf{{\textquotedbl}test.exe{\textquotedbl}} or
\textbf{{\textquotedbl}test{\textquotedbl}} on Linux. }


\bigskip


\bigskip

\begin{center}
\tablehead{}
\begin{supertabular}{|m{3.38376in}|m{3.38376in}|}
\hline
\sffamily\bfseries Windows Target Name &
\sffamily\bfseries Linux Target name\\\hline
\ttfamily [target name].exe &
\ttfamily [target name]\\\hline
\end{supertabular}
\end{center}

\bigskip

{\centering \par}

\begin{figure}
\centering
\includegraphics[width=4.4425in,height=3.5102in]{userguid-img185.jpg}
\end{figure}

\bigskip

{\centering\sffamily\itshape
\textbf{\ }Unicon Project Names Tab for a Windows application
\par}


\bigskip

{\sffamily
The names tab specifies the compiler and linker default values and their
flags in addition to the target file name.}

{\sffamily\bfseries\itshape\color[rgb]{0.0,0.2784314,1.0}
3. Add source files to your project by clicking on the
{\textquotedblleft}Files{\textquotedblright} tab and clicking on
{\textquotedblleft}Add{\textquotedblright} to select the source files
you want to Add }

{\centering\sffamily\itshape
Unicon Project Files Tab
\par}

\begin{figure}
\centering
\includegraphics[width=4.0965in,height=3.1984in]{userguid-img186.jpg}
\end{figure}

\bigskip

{\sffamily\bfseries\itshape\color[rgb]{0.0,0.2784314,1.0}
4. After adding your files, click
{\textquotedblleft}Create{\textquotedblright} to create your project}

{\sffamily
Once you have added the files needed click
{\textquotedbl}Create{\textquotedbl}. CVE's IDE will create a
Makefile suitable to compile your project with the Unicon compiler. }

{\sffamily
That{\textquotesingle}s it! You can now create your project. After you
are done creating it, you can run it by following the instruction
below. }

\subsubsection{Saving your Unicon project}
{\sffamily
At any time it is possible to save your project. It is a good idea to
back up your work everytime, so that in case of a malfunction, you can
still retrieve your work from the saved file.}

{\sffamily
\textbf{\textit{NOTE:}} Whenever you run a program, your file gets saved
automatically}


\bigskip

{\sffamily\bfseries\itshape\color[rgb]{0.0,0.2784314,1.0}
1. To Save your project, click on {\textquotedblleft}File {\textgreater}
Save{\textquotedblright} or {\textquotedblleft}File {\textgreater} Save
As{\textquotedblright} (in the File Menu Bar) or click on the Save and
Save As icons (in the IDE toolbar)}

\begin{center}
\tablehead{}
\begin{supertabular}{|m{1.4129599in}|m{1.6379598in}|}
\hline
{\sffamily\itshape Save }

\centering
\includegraphics[width=0.4165in,height=0.3055in]{userguid-img187.jpg}
~
 &
\sffamily\itshape Save As % \centering
\includegraphics[width=0.389in,height=0.3335in]{userguid-img188.jpg}
\\\hline
\end{supertabular}
\end{center}
{\sffamily
%\newline
2. Click on {\textquotedblleft}Save{\textquotedblright} to save your
file}

\subsubsection[Running/Debugging a Unicon Project]{Running/Debugging a
Unicon Project}

\bigskip

{\sffamily
After creating your file, it is possible to run and debug your project.
Since you might have some errors in your project, CVE will
automatically check for bugs and take you to the specific line where
the error was made. After correcting these bugs, you will be able to
run it.}

{\sffamily
Automatically, when running a file, CVE will save your file for you. }


\bigskip

{\sffamily\bfseries\itshape\color[rgb]{0.0,0.2784314,1.0}
1. To Run your project, press \ {\textquotedblleft}Run {\textgreater}
Run Program{\textquotedblright} (in the File Menu Bar) or click the Run
Icon (in the IDE Toolbar) }



\begin{figure}
\centering
\includegraphics[width=0.5555in,height=0.361in]{userguid-img189.jpg}
\end{figure}
{\sffamily
Since the CVE IDE contains an automatic debugger, you will see any
errors in your code displayed in the Run / Debug Window below your
file.}



\bigskip


\bigskip

{\sffamily\bfseries\color{red}
[1] File:}


\bigskip

\liststyleLxxxii
\begin{itemize}
\item {\sffamily\color{red}
\textbf{Compiling and running your project. (will follow)} }
\item {\sffamily\color{red}
\textbf{UML Diagram of the IDE files.} }
\end{itemize}

\bigskip

\liststyleLxxxiii
\begin{itemize}
\item {\sffamily\color{red}
\textbf{Unicon Makefile Maker: (umfMaker): foll} }
\end{itemize}

\bigskip

{\centering 
\includegraphics[width=6.6252in,height=4.3126in]{userguid-img190.jpg}
\par}


\bigskip


\bigskip

\clearpage\subsection{Chapter 11: Creating Java Projects in CVE}
{\sffamily
This chapter describes the specific steps involved in creating a Java
project using the CVE IDE. These steps include:}

\liststyleLii
\begin{itemize}
\item {\sffamily\bfseries\color[rgb]{0.0,0.0,0.5019608}
Creating a New Java project }
\item {\sffamily\bfseries\color[rgb]{0.0,0.0,0.5019608}
Creating source code in Java}
\item {\sffamily\bfseries\color[rgb]{0.0,0.0,0.5019608}
Creating specific Java applications}
\item {\sffamily\bfseries\color[rgb]{0.0,0.0,0.5019608}
Saving your Java Project}
\item {\sffamily\bfseries\color[rgb]{0.0,0.0,0.5019608}
Running and Debugging a Java Project }
\end{itemize}

\bigskip

\subsubsection{Building Java Projects}
{\sffamily
The CVE IDE is different than other Java IDEs such as
Borland{\textquotesingle}s JBuilder. The CVE IDE is not designed for
giant Java projects, but it can still be used to compile big projects.}

{\sffamily
The CVE IDE has been designed to work with Sun{\textquotesingle}s
\textbf{\textit{JDK}} (\textbf{\textit{Java Development Kit}}). Instead
of using command lines for the compiler and interpreter, you interact
with these by using the CVE IDE. The entire development process is
simplified by letting you focus on coding without having to write long,
tedious commands to compile and run your programs. }

{\sffamily
Here{\textquotesingle}s the general procedure for building a working
Java application using CVE IDE. Using CVE IDE, the project
creation cycle for a Java application is as follows:}

\liststyleLxxxiv
\begin{enumerate}
\item {\sffamily\bfseries\color{black}
Create a project}
\item {\sffamily\bfseries\color{black}
Create your source code}
\item {\sffamily\bfseries\color{black}
Compile your project}
\item {\sffamily\bfseries
\textcolor{black}{Run your project}\newline
}
\end{enumerate}
\liststyleLxxxv
\begin{enumerate}
\item {\sffamily
\textbf{Create a Project.} \newline
Start with a new Project, even if you don{\textquotesingle}t yet have
any source code entered. Name the project the same name as your main
Java application name. Once the new project has been created, you will
define the \textbf{\textit{application name}}, the \textbf{\textit{name
of the main source file}}, the \textbf{\textit{compiler options}}, and
other information needed to compile your application. }


\bigskip
\item {\sffamily
\textbf{Create your source code.} \newline
CVE's IDE provides an easy editor. The IDE editor is different
than the ordinary generic editor. It has special icons and a lot of
commands appropriate for programming. }


\bigskip
\item {\sffamily
\textbf{Compile your project.} \newline
After the source code has been entered; the next step is to compile your
source to object code. \ You do this by clicking on
{\textquotedblleft}\textbf{\textit{Project{\textgreater} Compile
Java}}\textbf{{\textquotedblright} in the File Menu Bar}. Since it is
certain that the source code will have errors (either syntax or logical
errors); compilation errors will be displayed in the message-box
window. You can simply right-click on the error, and the IDE will open the
source code file, and go to the offending line. After making
corrections, you repeat this step until all compilation and linking
errors are removed and you are ready to run your program.}
\end{enumerate}
\begin{center}
\tablehead{}
\begin{supertabular}{|m{3.38376in}|m{3.38376in}|}
\hline
\liststyleLxxxv
\setcounter{saveenum}{\value{enumi}}
\begin{enumerate}
\setcounter{enumi}{\value{saveenum}}
\item[] \sffamily\itshape Compile your Program \end{enumerate}
 &
{\sffamily In the File Menu Bar, select}

\sffamily\bfseries\itshape Project {\textgreater} Compile Java\\\hline
\end{supertabular}
\end{center}
\liststyleLxxxv
\setcounter{saveenum}{\value{enumi}}
\begin{enumerate}
\setcounter{enumi}{\value{saveenum}}
\item[] 
\bigskip
\end{enumerate}

\bigskip

\liststyleLxxxv
\setcounter{saveenum}{\value{enumi}}
\begin{enumerate}
\setcounter{enumi}{\value{saveenum}}
\item {\sffamily
\textbf{Run your project.} \newline
Starting the program using the CVE IDE will be direct by choosing
{\textquotedblleft}\textbf{Project {\textgreater} Run
Project{\textquotedblright}} from the File Menu Bar or simply by
selecting the Run Icon.\newline
}
\end{enumerate}
\begin{center}
\tablehead{}
\begin{supertabular}{|m{3.38376in}|m{3.38376in}|}
\hline
\sffamily\itshape Run your Program  &


%\centering
\includegraphics[width=0.5555in,height=0.361in]{userguid-img191.jpg}
{\sffamily Or in the File Menu Bar Select}

\sffamily\bfseries\itshape Project {\textgreater} Run\\\hline
\end{supertabular}
\end{center}

\bigskip


\subsubsection{Creating a New Java Project}

\bigskip

{\sffamily\bfseries\itshape\color[rgb]{0.0,0.2784314,1.0}
1. Click on {\textquotedblleft}Project {\textgreater} New {\textgreater}
Java Project{\textquotedblright} on the File Menu Bar}

{\sffamily
To create a new Java project in the CVE IDE, click on the
\textbf{\textit{{\textquotedbl}Project {\textgreater} New
{\textgreater} Java Project{\textquotedbl}}} menu. After you select a
name for your project, automatically a dialog wizard will be opened;
this wizard will let you begin defining a project. This is a dialog box
with various tab-items }


\bigskip

{\sffamily
For most projects you will use just two of those several tabs, namely
the \textbf{Main} and \textbf{Files} tabs. \newline
However, below the function of the \textbf{Compiler},
\textbf{Interpreter} and \textbf{Debugger} tabs are also briefly
discussed. }


\bigskip


\bigskip

{\centering \par}

\begin{figure}
\centering
\includegraphics[width=4.4634in,height=3.3854in]{userguid-img192.jpg}
\end{figure}

\bigskip


\bigskip

{\sffamily\bfseries\itshape\color[rgb]{0.0,0.2784314,1.0}
2. Specify the name of your project by clicking on the
{\textquotedblleft}Main{\textquotedblright} Tab}

{\sffamily
Clicking on \textbf{Main} Tab allows you to enter the name of the
project. This should be the Java Class name of the main file. You
don{\textquotesingle}t need to add .java or any other extensions. When
creating an applet, the HTML file must be specified to allow the applet
to run. It normally uses \texttt{appletviewer.}}

\clearpage{\sffamily\bfseries\itshape\color[rgb]{0.0,0.2784314,1.0}
3. Specify the .Java files source directory by clicking on the
{\textquotedblleft}Files{\textquotedblright} tab}

{\sffamily
The \textbf{Files Tab} \ is where you can specify both of the .java
files source directory and the .class files output directory. It is
better to have all the files in the same directory with the project
file and to leave the default values. For ease of use, you do not need
to fill any of the fields out in the Files tab to keep all of your
files in the same directory.}


\bigskip

{\centering\sffamily\itshape
%\newline
Java Project Files Tab
\par}

\begin{figure}
\centering
\includegraphics[width=4.4634in,height=3.3854in]{userguid-img193.jpg}
\end{figure}

\bigskip

{\sffamily\bfseries\itshape\color[rgb]{0.0,0.2784314,1.0}
3. Click {\textquotedblleft}OK{\textquotedblright} to start creating
your Java Project}


\bigskip

{\sffamily
After you have checked the values for \textbf{Main} and \textbf{Files
tabs}, click on {\textquotedblleft}\textbf{OK{\textquotedblright}} to
start programming your JAVA Project. If you want to specify other
values, below are descriptions of the functions of the other tabs.}


\bigskip


\bigskip

{\centering\bfseries\itshape
Compiler
\par}

{\sffamily
The function of the \textbf{Compiler} tab (Figure 6-3) is to enter the
java compiler and its options (the compiler is always \textbf{javac}).
}

{\sffamily
\textbf{Compiler Switch Pool:} contains all the supported JDK options.}


\bigskip


\bigskip

{\centering \par}

\begin{figure}
\centering
\includegraphics[width=4.4634in,height=3.3854in]{userguid-img194.jpg}
\end{figure}
{\centering\sffamily\itshape
Java Project Compiler Tab
\par}


\bigskip

{\centering\bfseries\itshape
Interpreter
\par}

{\sffamily
The function of the \textbf{Interpreter} tab (Figure 6-4) is to enter
the java interpreter and its options (The default value is
\textstyleTeletype{\textrm{\textbf{java
}}}\textstyleTeletype{\textrm{which can be used for console
applications}}). It is \textbf{javaw }if the applications are GUI.}



\begin{figure}
\centering
\includegraphics[width=4.4634in,height=3.3854in]{userguid-img195.jpg}
\end{figure}
{\centering\sffamily\itshape
Java Project Interpreter Tab
\par}


\bigskip

{\centering\bfseries\itshape
Debugger
\par}

{\sffamily
The function of the Debugger tab is to check for errors in your program
before you run it. }

{\sffamily
Enter the java interpreter and its options (The default value is
\textbf{jdb})}


\bigskip

{\centering\sffamily\itshape
\textrm{\textbf{Java Project Debugger Tab}} 
\par}

\begin{figure}
\centering
\includegraphics[width=4.4634in,height=3.3854in]{userguid-img196.jpg}
\end{figure}
\subsubsection{Creating Source Code in Java}
{\sffamily
Running a Java program in the ordinary method takes several steps. The
first of these steps is creating the source file. Source codes normally
will be compiled by Javac, the Java compiler \textbf{which produces }a
bytecode as output. Bytecode can be executed by using the bytecode
interpreter. Interpreters are different according to the used operating
system and the type of the application.}


\bigskip

{\sffamily
\textbf{\textit{NOTE:}} Interpreters are different according to the used
operating system (Windows or Linux). }

\subsubsection{Creating Specific Java Applications}
{\sffamily
The section below describes the process of creating different programs
in Java}

\liststyleLxxxvi
\begin{itemize}
\item {\sffamily
Console Application}
\item {\sffamily
GUI Window Application}
\item {\sffamily
Applet}


\bigskip
\end{itemize}
{\centering\bfseries\itshape
Java console application:
\par}

{\sffamily\color{red}
What is a Java console application?}

{\sffamily
For creating Java console applications in both Windows and Linux, the
following compiler and interpreter need to be defined when creating
your source code:}

{\ttfamily
Source code -{\textgreater} compiler (javac) -{\textgreater} interpreter
(java)}

{\sffamily
\ (For both Windows and Linux)}


\bigskip

{\centering\bfseries\itshape
Java GUI window application:
\par}

{\sffamily\color{red}
What is a java gui window application?}

{\sffamily
For creating Java graphical user interface (GUI) window applications,
there is a difference between the interpreter used for Windows and
Linux applications: }

{\sffamily
Windows: \texttt{Source code -{\textgreater} compiler (javac)
-{\textgreater} interpreter (javaw)}}

{\sffamily
Linux: \texttt{Source code -{\textgreater} compiler (javac)
-{\textgreater} interpreter (java)}}

{\centering\bfseries\itshape
\ Java applet:
\par}

{\sffamily
\textcolor{red}{What is a java applet?} \newline
When creating a Java applet, a HTML file is required to be able to use
the applet bytecode; then the applet is initiated by using the
appletviewer or using the web browser.}

{\ttfamily\itshape
Source code -{\textgreater} javac -{\textgreater} + HTML -{\textgreater}
browser}

{\sffamily\itshape
{}-or-}

{\ttfamily\itshape
Source code -{\textgreater} javac -{\textgreater} + HTML -{\textgreater}
appletviewer}


\bigskip

{\sffamily\bfseries
CVE IDE Java Project creation cycle}

{\sffamily\color{red}
CVE's IDE benefits from the way that Java compiler (javac) behaves
while compiling the files of the project. Every single file can be
compiled separately by entering {\textquotedbl}javac
ProgramName.java{\textquotedbl} in the command line. This will produce
all the required class files.}


\bigskip

{\sffamily\color{red}
For ease, the new project creations works with only a single file; the
one that has the {\textquotedbl}main{\textquotedbl} method. It is
recommended to name the project by the main-file name. }


\bigskip

\subsubsection[Saving your Java project]{Saving your Java project}
{\sffamily
At any time it is possible to save your project. It is a good idea to
back up your work everytime, so that in case of a malfunction, you can
still retrieve your work from the saved file.}

{\sffamily
\textbf{\textit{NOTE:}} Whenever you run a program, your file gets saved
automatically}


\bigskip

{\sffamily\bfseries\itshape\color[rgb]{0.0,0.2784314,1.0}
1. To Save your project, click on {\textquotedblleft}File {\textgreater}
Save{\textquotedblright} or {\textquotedblleft}File {\textgreater} Save
As{\textquotedblright} (in the File Menu Bar) or click on the Save and
Save As icons (in the IDE toolbar)}

\begin{center}
\tablehead{}
\begin{supertabular}{|m{1.4129599in}|m{1.6379598in}|}
\hline
{\sffamily\itshape Save }

\centering
\includegraphics[width=0.4165in,height=0.3055in]{userguid-img197.jpg}
~
 &
\sffamily\itshape Save As %\centering
\includegraphics[width=0.389in,height=0.3335in]{userguid-img198.jpg}
\\\hline
\end{supertabular}
\end{center}
{\sffamily
%\newline
2. Click on {\textquotedblleft}Save{\textquotedblright} to save your
file}

\subsubsection{Running/Debugging a Java Project}

\bigskip

{\sffamily
After creating your file, it is possible to run and debug your project.
Since you might have some errors in your project, CVE will
automatically check for bugs and take you to the specific line where
the error was made. After correcting these bugs, you will be able to
run it.}

{\sffamily
Automatically, when running a file, CVE will save your file for you. }


\bigskip

{\sffamily\bfseries\itshape\color[rgb]{0.0,0.2784314,1.0}
1. To Run your project, press \ {\textquotedblleft}Run {\textgreater}
Run Program{\textquotedblright} (in the File Menu Bar) or click the Run
Icon (in the IDE Toolbar) }



\begin{figure}
\centering
\includegraphics[width=0.5555in,height=0.361in]{userguid-img199.jpg}
\end{figure}
{\sffamily\bfseries
Since the CVE IDE contains an automatic debugger, you will see any
errors in your code displayed in the Run / Debug Window below your
file.}

\clearpage
\bigskip

\subsection{Chapter 12: Creating C/C++ Projects in CVE}
{\sffamily
This chapter describes the specific steps involved in creating a C/C++
project using the CVE IDE. These steps are:}

\liststyleLii
\begin{itemize}
\item {\sffamily\bfseries\color[rgb]{0.0,0.0,0.5019608}
Creating a New C/C++ project }
\item {\sffamily\bfseries\color[rgb]{0.0,0.0,0.5019608}
Saving your C/C++ Project}
\item {\sffamily\bfseries\color[rgb]{0.0,0.0,0.5019608}
Running and Debugging a C/C++ Project}
\end{itemize}
\subsubsection{Building C/C++ Projects}
{\sffamily
In order to create C/C++ projects you{\textquoteright}ll need to take
the following steps: }

\liststyleLxxxvii
\begin{enumerate}
\item {\sffamily\bfseries\color{black}
Open and create a new project with the dialog wizard}
\item {\sffamily\bfseries\color{black}
Fill in the Name for the project }
\item {\sffamily\bfseries\color{black}
Select the type build and compiler for the project you are creating }
\item {\sffamily\bfseries\color{black}
Check the project attributes }
\item {\sffamily\bfseries\color{black}
Add the source files to your project}
\item {\sffamily\bfseries\color{black}
Save your Project}
\item {\sffamily\bfseries\color{black}
Run and Debug}
\end{enumerate}

\bigskip

\subsubsection{Creating a new C/C++ project with the dialog wizard }

\bigskip

{\sffamily
In order to create new C/C++ projects do the following:}


\bigskip

{\sffamily\bfseries\itshape\color[rgb]{0.0,0.2784314,1.0}
1. Click on {\textquotedblleft}Project {\textgreater} New {\textgreater}
C++ Project{\textquotedblright}}


\bigskip

{\sffamily
After you select a name for your project, automatically a dialog wizard
will be opened; this wizard will let you begin defining a project. }


\bigskip



\begin{figure}
\centering
\includegraphics[width=4.7744in,height=3.561in]{userguid-img200.jpg}
\end{figure}
{\centering\sffamily\itshape
Saving a new project
\par}


\bigskip

{\sffamily\bfseries\itshape\color[rgb]{0.0,0.2784314,1.0}
2. Fill in the name of the project you are building (Target name) and
click {\textquotedblleft}Save{\textquotedblright}}


\bigskip

{\sffamily
This target name depends on the type and the name of the application you
are building; If you are building a Windows project, type in the
[target name] + .exe, if you are writing a Linux project, type in just
the [target name]. }

{\sffamily
For instance, if the project name is \textbf{[test]} on windows we can
use for example \textbf{{\textquotedbl}test.exe{\textquotedbl} }or
\textbf{{\textquotedbl}test{\textquotedbl}} on Linux. Also it may be
{\textquotedblleft}\textbf{test.a{\textquotedblright}} when we build
{\textquotedblleft}\textbf{.a}{\textquotedblright} libraries; or a
\textbf{{\textquotedbl}test.lib{\textquotedbl}} for most windows
compilers. }


\bigskip


\bigskip

\begin{center}
\tablehead{}
\begin{supertabular}{|m{1.7976599in}|m{1.4934598in}|m{1.5622599in}|m{1.7587599in}|}
\hline
\sffamily\bfseries Windows Target Name &
\sffamily\bfseries Linux Target name &
\sffamily\bfseries A Libraries &
\sffamily\bfseries Most windows Compilers\\\hline
\ttfamily [target name].exe &
\ttfamily [target name] &
\ttfamily [target name].a &
\ttfamily [target name].lib\\\hline
\end{supertabular}
\end{center}

\bigskip


\bigskip


\bigskip

{\centering\sffamily\itshape
%\newline
New Project wizard
\par}

\begin{figure}
\centering
\includegraphics[width=4.6374in,height=3.7035in]{userguid-img201.jpg}
\end{figure}

\bigskip

{\sffamily\bfseries\itshape\color[rgb]{0.0,0.2784314,1.0}
3. Select the type of build and compiler for the project you are
creating }

{\sffamily
When starting a new project, you{\textquoteright}ll want to define the
program type of build and the compiler. You can choose from the
following options:}


\bigskip

\liststyleLxxxviii
\begin{itemize}
\item {\sffamily
Select {\textquotedblleft}\textbf{Console
Application{\textquotedblright}} for a shell or console window. }
\item {\sffamily
Select {\textquotedblleft}\textbf{GUI application{\textquotedblright}}
for a graphical interface. }
\item {\sffamily
Select {\textquotedblleft}\textbf{Dynamic Library}
\textbf{(DLL){\textquotedblright} }(dynamically loaded library) }
\end{itemize}
{\sffamily
When you are using a Windows system you will also be able to define the
compiler you are using. }

{\sffamily
%\newline
You can choose between the following compilers:}

\liststyleLxxxix
\begin{itemize}
\item {\sffamily\bfseries
MingGW }
\item {\sffamily\bfseries
Cygwin}
\item {\sffamily
\textbf{Borland} }
\item {\sffamily\bfseries
Visual C++}
\item {\sffamily\bfseries
Other Compilers}
\end{itemize}
{\sffamily
By selecting one of these in the New Project Wizard, you will ensure
that the compiler you chose will be used. }

{\sffamily\bfseries\itshape\color[rgb]{0.0,0.2784314,1.0}
%\newline
4. Click {\textquotedblleft}Next{\textquotedblright} when you are done
defining your project}


\bigskip

{\sffamily\bfseries\itshape\color[rgb]{0.0,0.2784314,1.0}
5. Check the project attributes by clicking on the
{\textquotedblleft}Names{\textquotedblright} and
{\textquotedblleft}Files{\textquotedblright} tabs}


\bigskip

{\sffamily
If you want to check the project attributes, click on the \textbf{Names
}and \textbf{Files }tabs\textbf{. \newline
\newline
}Once you have defined the initial project attributes, a new dialog box
with set of tab-items will appear. Most of the fields will be defined
according to the values you filled in the New Project wizard. Check to
make sure that these contain the values that you entered.}


\bigskip

{\centering 
\includegraphics[width=3.8854in,height=3.1043in]{userguid-img202.jpg}
\par}

{\centering\sffamily\itshape
Project Editor with {\textquotedblleft}Names{\textquotedblright} Tab
selected
\par}


\bigskip

{\sffamily\bfseries\itshape\color[rgb]{0.0,0.2784314,1.0}
6. Add source files to your project by clicking on the Files Tab and
clicking {\textquotedblleft}Add{\textquotedblright} to add specific
source files }

{\sffamily
If you want to add source files click on the
{\textquotedblleft}\textbf{Files}{\textquotedblright} tab. Make sure
that your source files are in the same folder as your project file that
you created. You can set definitions as needed in the other tabs. Once
you have added the files needed, click
\textit{{\textquotedbl}}\textbf{\textit{OK}}\textit{{\textquotedbl}.}
CVE IDE will create a Makefile suitable to compile your project
with gcc or Borland BCC32. }


\bigskip

{\centering 
\includegraphics[width=3.8543in,height=3.0835in]{userguid-img203.jpg}
\par}

{\centering\sffamily\itshape
\ Project Editor with {\textquotedblleft}Files{\textquotedblright} Tab
selected
\par}

{\sffamily
%\newline
If you want to specify other values, below are descriptions of the
functions of the other tabs.}


\bigskip


\bigskip

{\sffamily\bfseries
Paths}

{\sffamily
This tab lets you specify the source files directory, generated object
and binary files directories.}


\bigskip

{\centering 
\includegraphics[width=3.8752in,height=3.0728in]{userguid-img204.jpg}
\par}

{\centering\sffamily\itshape
Project Editor with {\textquotedblleft}Paths{\textquotedblright} Tab
selected
\par}


\bigskip

{\sffamily\bfseries
Libraries}

{\sffamily\color{red}
This tab lets you specify which library files you want to add to your
project.}

{\centering 
\includegraphics[width=3.8335in,height=3.052in]{userguid-img205.jpg}
\par}

{\centering\sffamily\itshape
Project Editor with {\textquotedblleft}Libraries{\textquotedblright} Tab
selected
\par}


\bigskip

{\sffamily\bfseries
Defines}

{\sffamily
This tab allows to define a compile time defines for C and C++
programs.}

{\centering \par}

\begin{figure}
\centering
\includegraphics[width=4.6374in,height=3.7035in]{userguid-img206.jpg}
\end{figure}
{\centering\sffamily\itshape
Project Editor with {\textquotedblleft}Defines{\textquotedblright} Tab
selected
\par}


\bigskip


\bigskip

{\sffamily\bfseries
Advanced}

{\sffamily\color{red}
This tab lets you define Makefile and Target options.}

{\centering\sffamily\itshape
Project Editor with {\textquotedblleft}Advanced{\textquotedblright} Tab
selected
\par}

\begin{figure}
\centering
\includegraphics[width=4.6374in,height=3.7035in]{userguid-img207.jpg}
\end{figure}
\subsubsection{Saving your C/C++ project}
{\sffamily
At any time it is possible to save your project. It is a good idea to
back up your work everytime, so that in case of a malfunction, you can
still retrieve your work from the saved file.}

{\sffamily
\textbf{\textit{NOTE:}} Whenever you run a program, your file gets saved
automatically}


\bigskip

{\sffamily\bfseries\itshape\color[rgb]{0.0,0.2784314,1.0}
1. To Save your project, click on {\textquotedblleft}File {\textgreater}
Save{\textquotedblright} or {\textquotedblleft}File {\textgreater} Save
As{\textquotedblright} (in the File Menu Bar) or click on the Save and
Save As icons (in the IDE toolbar)}

\begin{center}
\tablehead{}
\begin{supertabular}{|m{1.4129599in}|m{1.6379598in}|}
\hline
{\sffamily\itshape Save }

\centering
\includegraphics[width=0.4165in,height=0.3055in]{userguid-img208.jpg}
~
 &
\sffamily\itshape Save As %\centering
\includegraphics[width=0.389in,height=0.3335in]{userguid-img209.jpg}
\\\hline
\end{supertabular}
\end{center}
{\sffamily
%\newline
2. Click on {\textquotedblleft}Save{\textquotedblright} to save your
file}

\subsubsection[Running/Debugging a C/C++ Project]{Running/Debugging a
C/C++ Project}

\bigskip

{\sffamily
After creating your file, it is possible to run and debug your project.
Since you might have some errors in your project, CVE will
automatically check for bugs and take you to the specific line where
the error was made. After correcting these bugs, you will be able to
run it.}

{\sffamily
Automatically, when running a file, CVE will save your file for you. }


\bigskip

{\sffamily\bfseries\itshape\color[rgb]{0.0,0.2784314,1.0}
1. To Run your project, press \ {\textquotedblleft}Run {\textgreater}
Run Program{\textquotedblright} (in the File Menu Bar) or click the Run
Icon (in the IDE Toolbar) }



\begin{figure}
\centering
\includegraphics[width=0.5555in,height=0.361in]{userguid-img210.jpg}
\end{figure}
{\sffamily
Since the CVE IDE contains an automatic debugger, you will see any
errors in your code displayed in the Run / Debug Window below your
file.}

\clearpage
\bigskip

\subsection{Chapter 13: Using the Collaborative IDE to share Documents
with others }



\bigskip

This chapter details how you can develop a program code and share your
code in CVE with others from inside the Collaboration Area.


\bigskip

In CVE it is possible to share documents such as computer code that you
have written either in Java, C/C++ or Unicon by clicking on the Share
button shown in \textbf{\textit{Figure 13.1}}. As mentioned in the
previous chapter, CVE{\textquotesingle}s Integrated Development
Environment (IDE) lets you write, edit, run, compile and debug code
inside the virtual environment.

\bigskip

{\centering 
\includegraphics[width=6in,height=4in]{Chapter13-img1.jpg}
\par}

{\centering
\textcolor[rgb]{0.0,0.0,0.039215688}{Figure 13.1:
}{\textbf{\textcolor[rgb]{0.0,0.0,0.039215688} Collaboration Features}
\par}}

\bigskip

The collaboration area can also be used to share your files in a
collaborative session. This makes the IDE especially suited for
education purposes, because students can debug their code with the help
of other students or their instructor by sharing their code with them.

NOTE: It is possible to share your code with others by you initiating a
collaborative session with another user (other users can be either
available user{\textquoteright}s friends or users who are experts in
the shared file programming language). This area lets you organize your
projects, connect with other users and share your code by inviting
another user to share your screen.

Sharing code will lead to a greater sharing of knowledge


\bigskip

This chapter will describe the following items:

\begin{itemize}
\item Sharing your code with others through the Collaborative IDE
\item Inviting a Slave user to collaborate
\item Collaborative Session Options
\item Taking a Turn on a Session
\item Closing a Collaborative Session
\end{itemize}
The IDE gives users the ability to work on a project in the virtual
world individually, as well as share these files with other users for
troubleshooting. This lets users work on projects together and read
each other{\textquotesingle}s code by using the collaborative function
of the IDE (see Figure 13-2).

\bigskip

{\centering 
\includegraphics[width=3.1252in,height=2.3437in]{Chapter13-img2.jpg}
\par}

{\centering
\textcolor[rgb]{0.0,0.0,0.039215688}{Figure 13.2:
}{\textbf{\textcolor[rgb]{0.0,0.0,0.039215688} CVE connects students with each other and the instructor}
\par}}



\bigskip

\textbf{Sharing your code with others through the Collaborative IDE}


\bigskip

In the collaborative IDE a student can ask the instructor to see his/her
computer code, so the instructor can help the student in debugging the
code. In the collaborative IDE there are two roles that can be played
by users in collaborating with each other: Master and Slave.


\bigskip

The Master user is the person currently working on a program and shares
access to their screen, whereas the Slave is witness to what the Master
user is doing on their screen. However, a Slave user can ask for
permission from a Master user by Taking a Turn to work on a file if
they want to try their hand at it. This then switches the roles around
between Master and Slave users.

\bigskip

{\centering 
\includegraphics[width=3.2602in,height=2.4453in]{Chapter13-img3.jpg}
\par}

{\centering
\textcolor[rgb]{0.0,0.0,0.039215688}{Figure 13.3:
}{\textbf{\textcolor[rgb]{0.0,0.0,0.039215688} Master and Slave roles
 during real-time collaboration.}
\par}}



\bigskip

\textbf{\textit{NOTE:}}


\bigskip

\textbf{\textit{Master user:}} The Master user invites another person in
sharing their workscreen, and is the person who controls all the events
in the shared editor screen.


\bigskip

\textbf{\textit{Slave user:}} The Slave user accepts the invitation from
the Master user, but can only see what actions the Master user
performed in the editor screen. The Slave user can ask for permission
to Take a Turn and become a Master user who can edit the file.


\bigskip

Anybody can be a \textbf{\textit{Master}} or \textbf{\textit{Slave}}; it
just depends on which person initiates a collaborative session. For
instance, if a student wants to debug his code with the help of an
instructor, the student (as \textbf{\textit{Master}} user) will need to
invite the instructor (\textbf{\textit{Slave}} user), so that the
instructor can see the student{\textquotesingle}s interactions on
screen. The instructor (the Slave user) can then give feedback through
chat or Take a Turn to help the student debug their code.


\bigskip

In order to use the \textbf{\textit{Collaborative IDE}}, a Master user
will always need to take the following steps:


\bigskip

\textbf{\textit{Step 1}}\textbf{\textit{: }}Open or create a filele to
collaborate on

\textbf{\textit{Step 2: }}\ Invite a Slave user to collaborate

Above is an overview of the process, followed by the individual steps to
opening a file and inviting a user to collaborate.

\bigskip

{\centering 
\includegraphics[width=6.328in,height=3.6701in]{Chapter13-img4.jpg} 
\par}

{\centering
\textcolor[rgb]{0.0,0.0,0.039215688}{Figure 13.4:
}{\textbf{\textcolor[rgb]{0.0,0.0,0.039215688} Initiating a Collaboration Session}
\par}}



\bigskip

\textbf{\textit{NOTE:}}


\bigskip

In order to collaborate on a file, a Master user will always need to
take these two steps:

1. Open an existing or new file to work on

2. Invite a Slave user to a collaborative session


\bigskip

\textbf{\textit{Inviting another User to Collaborate}}


\bigskip

NOTE: In order to invite another user, a file owner will need to first
have opened a file (as described above) prior to initiating
collaboration with a guest user.


\bigskip

Inviting a guest user in CVE is easy:


\bigskip


\begin{enumerate}
\item With a file opened, click the
{\textquotedblleft}Share{\textquotedblright} button (Figure 13.4 (B)).
\item From the generated drop down menu, choose one of the users to do a
collaborative IDE session with.
\end{enumerate}

\bigskip

A notification will be listed in a popup menu under the notification count
(see Figure 13.5) in the selected
Slave user{\textquotesingle}s client window saying that User [Master
User Name] is asking to open collaborative editor session. This signals
to the Slave user that the Master user sent him/her a collaborative
editing session invitation.

\bigskip

{\centering 
\includegraphics[width=3.1484in,height=1.139in]{Chapter13-img5.jpg}
\par}

{\centering
\textcolor[rgb]{0.0,0.0,0.039215688}{Figure 13.5:
}{\textbf{\textcolor[rgb]{0.0,0.0,0.039215688} Pending invitations browsed
from the popup menu.}
\par}}


\bigskip

The slave user can accept, reject, or forward the invitation by choosing
the invitation item appears in Figure 13.5, and a notification dialog
will appear (See Figure 13.6).

{\centering 
\includegraphics[width=1.7252in,height=1.9965in]{Chapter13-img6.jpg}
\par}

{\centering
\textcolor[rgb]{0.0,0.0,0.039215688}{Figure 13.6
}{\textbf{\textcolor[rgb]{0.0,0.0,0.039215688} pending invitation details in
the notification dialog.}
\par}}



\bigskip

\setcounter{saveenum}{\value{enumi}}
\begin{enumerate}
\setcounter{enumi}{\value{saveenum}}
\item Modify the shared file in the editor screen.
\end{enumerate}

\includegraphics[width=6.328in,height=3.611in]{Chapter13-img7.jpg} 

{\centering
\textcolor[rgb]{0.0,0.0,0.039215688}{Figure 13.7:
}{\textbf{\textcolor[rgb]{0.0,0.0,0.039215688} collaboration windows have a yellow background}
\par}}


\bigskip

The invited user will see the inviting user{\textquotesingle}s name in
the \textbf{\textit{Session Tree}} (See Figure 13.7) with a
\textbf{\textit{*(owner)}} next to the name to indicate that this is
the owner of the session and the one who currently have the permission
to edit the file (*). 


\bigskip

\textbf{Collaborative Session Options}


\bigskip

As mention earlier, and shown in Figure 13.6, when invited by a Master
user, the Slave user can Accept, Reject, or Forward the Master
user{\textquotesingle}s request for a collaborative editing session.


\bigskip

{\centering
\textit{When a collaborative session is rejected}
\par}


\bigskip

If the Slave user rejects the invitation, the new editor tab will not
open on the Slave user{\textquotesingle}s screen, and the Master
user{\textquotesingle}s request for a collaborative session is
rejected. A message from the user invited will appear in the Master
user{\textquotesingle}s screen indicating that the session invitation
is rejected.


\bigskip

{\centering
\textit{When a collaborative session is accepted:}
\par}


\bigskip

If the slave user accepts the invitation, a new editor tab will open in
the slave user. A chat message will appear in the Master
user{\textquotesingle}s screen indicating that
{\textquotedblleft}\textit{[User Name]} is accepted to start the
collaborative IDE{\textquotedbl}, which indicates to them that their
request for a collaborative session is accepted by the Slave user. 

As displayed in Figure 13.7, a collaborative session changes by the
background color to light yellow in both the Master{\textquoteright}s
and Slave{\textquotesingle}s side (which indicates that the tab is
currently used for a collaborative session).


\bigskip

\textbf{\textit{NOTE:}}


\bigskip

The Slave user has no ability to type or make any changes in the editor,
since the Master user can only make changes. Hence, during the
collaborative session, the tab for the Master user is unlocked, whereas
the Slave user{\textquotesingle}s tab is locked.

The Slave user can give feedback to the Master user by using voice chat
or text chat. Moreover, the Slave user can ask to \textit{Take a Turn},
which will allow the Slave user to become the owner of the lock and
edit the file while the Master becomes their Slave user.


\bigskip

When a Collaborative Session has started, anything the Master user does
in his/her editor will appear in the Slave user{\textquotesingle}s
editor screen:


\bigskip

%\liststyleWWNumvii
\begin{itemize}
\item Typing from the keyboard
\item Selecting text using mouse or keyboard
\item Using the scrollbar
\item Mouse Click, PgUp, PgDn keys.
\item Selecting all (Ctrl-A), Copy (Ctrl-C), Cut (Ctrl-X), Paste
(Ctrl-V), Undo (Ctrl-Z) and Redo (Ctrl-Y)
\end{itemize}

\bigskip

During the collaborative session, the Slave user can chat with the
Master user and give feedback to the Master user about the
collaborative session. A Slave user can also ask to
\textbf{\textit{Take A Turn}} to edit the file themselves.


\bigskip

{\centering
\textit{Taking a Turn on a Collaborative Session}
\par}


\bigskip

In order to Take a Turn and work on a program, a guest user must ask for
permission for this from the Master user. This ensures that the Master
user is finished with their work and is ready to share their program
with the Slave user.


\bigskip

In order for a Slave user to \textbf{\textit{Take Turn}}, the following
will need to be done:


\bigskip

%\liststyleWWNumviii
\begin{enumerate}
\item The Slave user will need to click on the Take Turn button (See
Figure 13.4(C)), and notification will be sent to the owner of the lock
(Master) (See Figure 13.8).
\end{enumerate}
{\centering 
\includegraphics[width=3.5307in,height=1.1154in]{Chapter13-img8.jpg}
\par}

{\centering
\textcolor[rgb]{0.0,0.0,0.039215688}{Figure 13.8:
}{\textbf{\textcolor[rgb]{0.0,0.0,0.039215688} a Pending Take Turn Request}
\par}}

\bigskip

\setcounter{saveenum}{\value{enumi}}
\begin{enumerate}
\setcounter{enumi}{\value{saveenum}}
\item The Master can either \textit{accept} or \textit{ignore} the
notification when asked if the Slave user can Take a Turn and start
editing the Master user{\textquotesingle}s file.
\end{enumerate}

\bigskip

If a Master user is ready, they can accept to switch roles with the
Slave user.

If a Master user is not ready to give the Slave user a turn at editing
the file, they can choose to ignore the invitation.


\bigskip

{\centering
\textit{Closing (Leaving) a Collaborative Session}
\par}


\bigskip

At any time the user can leave the collaborative session, by clicking on
the Leave button (see Figure 13.4(D)) 


\bigskip

The Shared editor area will return from the collaborative light yellow
color to a regular white background to indicate that the collaborative
session has ended. Closing (Leaving) a session will automatically save
the last version of the file on the owner and guest
user{\textquotesingle}s side.



\pagebreak

\subsection[Chapter 14: How do
I....]{\textstyleStrongEmphasis{\textmd{Chapter 14: How do I....}}}

\bigskip


\bigskip

\subsection[References]{\textstyleStrongEmphasis{\textmd{References}}}

\bigskip
\end{document}
